% Typeset/compile with pdfLaTeX or XeLaTeX
\documentclass{article}
%\DeclareUnicodeCharacter{1E41}{\d{m}} %works in pdfLaTeX, but not LuaLaTeX
\DeclareUnicodeCharacter{1E41}{$\dot{m}$} %works in pdfLaTeX, but not LuaLaTeX
\usepackage[calc,useregional]{datetime2}

% TODO:
% check all dashes are -- not just -.
% check spacing around dashes
% check for more diacritics??
% update the comemntary and opponent sections that don't quite make sense
% get feedback on changes to commentary and opponent section and if they make more sense with new changes
% FIX HEADERS!!
% make smaller margins?
% possible to use subsection X.n as variable?
% use uneven columns? https://www.overleaf.com/learn/latex/Multiple_columns#Unbalanced_columns
% italicize titles

%%%%%%%%%%%%%%%%%%%%%%%%%%%%%%%%%%
%%
%%         Edit these values
%%

\title{Eastern Classics Reading List 2024-2025}
\author{St. John's Santa Fe Graduate Institute}
\date{Updated: 2025-01-25}

\DTMsavedate{fallStart}{2024-09-02}
\DTMsavedate{springStart}{2025-01-20}
\DTMsavedate{summerStart}{2025-06-16}
\DTMsavedate{fallPrecept1Start}{2024-09-03}
\DTMsavedate{fallPrecept2Start}{2024-10-29}
\DTMsavedate{summerPreceptStart}{2025-06-16}

%date for break are inclusive, i.e. there are no classes on start or end dates
\DTMsavedate{thanksgivingBreakStart}{2024-11-27}
\DTMsavedate{thanksgivingBreakEnd}{2024-12-01}
\DTMsavedate{springBreakStart}{2025-03-15}
\DTMsavedate{springBreakEnd}{2025-03-30}
% 2024-12-20 no classes, end of fall semester
% 2025-05-23 no classes, end of spring semester

%NOTE: trying to save one date as another doesn't work, need to do this little dance to get it into the right format:
%\DTMsavedate{testDate}{\DTMfetchyear{springStart}-\DTMfetchmonth{springStart}-\DTMfetchday{springStart}}

%%
%%
%%        END: Edit these values
%%
%%%%%%%%%%%%%%%%%%%%%%%%%%%%%%%%%%

% From http://tex.stackexchange.com/questions/34040/graphics-logo-in-headers
\usepackage{geometry}
\geometry{letterpaper, margin=0.75in}

%\usepackage{layout}
%\usepackage{blindtext}
\usepackage{multicol}
\usepackage{comment}

% Insert header image
% \usepackage{lipsum}
\usepackage{graphicx}
\usepackage{fancyhdr} % https://www.overleaf.com/learn/latex/Headers_and_footers#Using_the_fancyhdr_package
\pagestyle{fancy}
\fancyhead{} % This is the text to show in the header, right now we want none
\renewcommand{\headrulewidth}{0pt}
\renewcommand{\footrulewidth}{0pt}
\setlength\headheight{92.0pt}
\addtolength{\textheight}{-92.0pt}
% \chead{\includegraphics[width=\textwidth]{MAEC-header.png}}
\chead{\includegraphics[width=\textwidth]{header-blank.png}}


% Tex automaticaly adds numbers to the beginning of section, subsections etc (e.g. "1.3 Session 3") but we don't want this, so setting it to 0
\setcounter{secnumdepth}{0}

% only show first level of sections in Table of Contents (i.e. only `\section{}` commands)
\setcounter{tocdepth}{1}

% how far apart do we want the columns?
\setlength{\columnsep}{1cm}

% run this at begin of document, so they exist. overwrite currentClassDate later, prob with one of the fall/spring/summer-start dates.
\newcount\myNextClassRegister
\newcounter{dateChanged}
\DTMsavenow{currentClassDate}
\DTMsavenow{lastPrintedClassDate}


% NOTE: "<dow> is the day of the week number starting from 0 for Monday"


%% code/section flow: 
%pass in date of last class (that was just printed), 
%get the next class day with \getNextBiweeklyClassDate, 
%then check to make sure the next class day is not during a break
%	if no, cool, print it.
%	if yes, ugh:
%		hand back last day of break for which the previously computed 'next class' date was. We say that the break dates are inclusive, so just pass the last day of break to the \getNextBiweeklyClassDate command and we shoudl be good. then print that new date.

% function - \printHeaderForNextSeminarClassDate: helper function that just runs `\checkIfHoliday{#1}` and then with output, runs `\getNextBiweeklyClassDate{#1}{0}{3}` so that user doesn't need to know about the 0=Monday and 3=Thursday nomenclature.
\newcommand{\printHeaderForNextSeminarClassDate}[1]{%
  % save input to tempDate variable, don't really need to but looks prettier for code?
  \DTMsavedate{tempDate}{\DTMfetchyear{#1}-\DTMfetchmonth{#1}-\DTMfetchday{#1}}%
  % make sure dateChanged counter = 0 to properly record if any changes happen
  \setcounter{dateChanged}{0}%

  %test%\DTMusedate{currentClassDate}%
  \getNextBiweeklyClassDate{0}{3}{tempDate}%
  % \getNextBiweeklyClassDate{0}{3}{#1}%

  \checkIfDateIsBetweenSchoolBreaks{tempDate}%

  % check if date was changed
  \ifnum \value{dateChanged}>0%
    %date was changed, so now get next class day after
    \getNextBiweeklyClassDate{0}{3}{tempDate}%
  \fi%

  %now pass that date on to be printed out
  \printHeaderWithSessionAndDate{tempDate}%
}


% function - \printHeaderForNextPreceptClassDate: helper function that just runs `\checkIfHoliday{#1}` and then with output, runs `\getNextBiweeklyClassDate{#1}{1}{3}` so that user doesn't need to know about the 1=Tuesday and 3=Thursday nomenclature.
\newcommand{\printHeaderForNextPreceptClassDate}[1]{%
  % save input to tempDate variable, don't really need to but looks prettier for code?
  \DTMsavedate{tempDate}{\DTMfetchyear{#1}-\DTMfetchmonth{#1}-\DTMfetchday{#1}}%
  % make sure dateChanged counter = 0 to properly record if any changes happen
  \setcounter{dateChanged}{0}%

  \getNextBiweeklyClassDate{1}{3}{tempDate}%

  \checkIfDateIsBetweenSchoolBreaks{tempDate}%

  % check if date was changed
  \ifnum \value{dateChanged}>0%
    %date was changed, so now get next class day after
    \getNextBiweeklyClassDate{1}{3}{tempDate}%
  \fi%

  %now pass that date on to be printed out
  \printHeaderWithSessionAndDate{tempDate}%
}


%\getNextBiweeklyClassDate: pass in <DoW of 1st class day>, <DoW of 2nd class day>, and <a Date for which you want to know the next class Date on the appropriate DoW>
% e.g. :
%    \DTMsavedate{fallStart}{2024-09-02} % Note: a monday
%    Usage: type in document `\getNextBiweeklyClassDate{0}{3}{fallStart}`
%        Notes: 0 and 3 indicate classes are Monday (0), and Thursday (3).tempDate
%    Function Result: In above example, saves the date of "2024-09-05" (because that is the Thursday after the Monday provided) to the variable `currentClassDate` (equivalent to running `\DTMsavedate{currentClassDate}{2024-09-05}`)
\newcounter{DoWfirst}%
\newcounter{DoWsecond}%
\newcounter{daysToAdvance}%
\newcommand{\getNextBiweeklyClassDate}[3]{%
  % put DoW for classes in order, in case user inputs {3}{0}{date}
  \ifnum #1<#2%
    \setcounter{DoWfirst}{#1}%
    \setcounter{DoWsecond}{#2}%
  \else%
    \setcounter{DoWfirst}{#2}%
    \setcounter{DoWsecond}{#1}%
  \fi%

  %figure out how many days to advance any date to get the next biweekly class DoW date
  \ifnum \DTMfetchdow{#3}<\value{DoWfirst}%
    % increment date to DoW of DoWfirst
    \DTMsaveddateoffsettojulianday{#3}{\numexpr\value{DoWfirst}-\DTMfetchdow{#3}}{\myNextClassRegister}%
    \DTMsavejulianday{tempDate}{\number\myNextClassRegister}%
  \else%
    \ifnum \DTMfetchdow{#3}<\value{DoWsecond}% %i.e. DoW of currentClassDate is NOT less than DoWfirst, and IS less than DoWsecond
      % increment date to DoW of DoWsecond
      \DTMsaveddateoffsettojulianday{#3}{\numexpr\value{DoWsecond}-\DTMfetchdow{#3}}{\myNextClassRegister}%
      \DTMsavejulianday{tempDate}{\number\myNextClassRegister}%
    \else% %i.e. DoW of currentClassDate is NOT less than DoWfirst, and is NOT less than DoWsecond
      %advance the date to next DoWfirst after date's DoW
      %\the\numexpr(7+\value{DoWfirst}-\DTMfetchdow{#3})
      \setcounter{daysToAdvance}{\numexpr(7+\value{DoWfirst}-\DTMfetchdow{#3})}%
      \DTMsaveddateoffsettojulianday{#3}{\value{daysToAdvance}}{\myNextClassRegister}%
      \DTMsavejulianday{tempDate}{\number\myNextClassRegister}%
    \fi%
  \fi%
}


%function - \checkBreakRanges: takes in a date and the two Days of the week that a class is held on, check if date is between the defined school breaks.
% returns nothing, but will change value of `tempDate` and `dateChanged` if needed.
\newcommand{\checkIfDateIsBetweenSchoolBreaks}[1]{%
  % check if date passed in is between two defined schools breaks
  \checkIfDateBetweenDates{tempDate}{thanksgivingBreakStart}{thanksgivingBreakEnd}%
  \checkIfDateBetweenDates{tempDate}{springBreakStart}{springBreakEnd}%
}


%function - \checkIfDateBetweenDates: this command will check if a date is between two dates (inclusively). If it is not, functiondoes nothing. If it is between the two dates, this function will write a header of "No Classes from <first-date-in-range> - <last-date-in-range>" and then set a counter to mark that the date is change, and change the value of `tempDate` to <last-date-in-range>.
% parameters: \checkIfDateBetweenDates{<date-to-check>}{<first-date-in-range>}{<last-date-in-range>}
\newcommand{\checkIfDateBetweenDates}[3]{%
  \DTMifdate
  {#1}
  {between=
  \DTMfetchyear{#2}-\DTMfetchmonth{#2}-\DTMfetchday{#2}
  and
  \DTMfetchyear{#3}-\DTMfetchmonth{#3}-\DTMfetchday{#3}
  } {\DTMsavedate{tempDate}{\DTMfetchyear{#3}-\DTMfetchmonth{#3}-\DTMfetchday{#3}} \setcounter{dateChanged}{1} \subsection{No Classes from \DTMusedate{#2} -- \DTMusedate{#3}}}{}
}


%function - \printHeaderWithSessionAndDate: increments cntSession (since it starts at 0), then prints text for Session <cntSession> - <currentClassDate> 
\newcommand{\printHeaderWithSessionAndDate}[1]{%
	\stepcounter{cntSession}%
	\subsection{Session \arabic{cntSession} - \DTMusedate{#1}}%

  % save date we just printed into lastPrintedClassDate to pass on to next header printing
  \DTMsavedate{lastPrintedClassDate}{\DTMfetchyear{#1}-\DTMfetchmonth{#1}-\DTMfetchday{#1}}
}


%function - \printHeaderWithSession: increments cntSession (since it starts at 0), then prints text for Session <cntSession>
% just in case one doesn't want the dates that I wrote so much code to calculate... Don't use this function and \printHeaderWithSessionAndDate in same doc, it will prob mess up counters/registerss
\newcommand{\printHeaderWithSession}{%
	\stepcounter{cntSession}%
	\subsection{Session \arabic{cntSession}}%
}


% set counter/variable to keep track of which week it is
\newcounter{cntSemester} % mostly used to reset class session counter
\newcounter{cntSession}
\counterwithin{cntSession}{cntSemester}

% Load this package last?!
\usepackage{hyperref} % make clickable links, e.g. for table of contents https://www.overleaf.com/learn/latex/Hyperlinks

%% make links the color
%\hypersetup{
%  colorlinks   = true, %Colours links instead of ugly boxes
%  urlcolor     = blue, %Colour for external hyperlinks
%  linkcolor    = blue, %Colour of internal links
%  citecolor   = red %Colour of citations
%}

%% underline the black text links with colored line
\usepackage{xcolor}  
\hypersetup{
    pdfborderstyle={/S/U/W 1}, % underline links instead of boxes
    linkbordercolor=red,       % color of internal links
    citebordercolor=green,     % color of links to bibliography
    filebordercolor=magenta,   % color of file links
    urlbordercolor=cyan        % color of external links
}

% End - Introductory typesetting configuration


\begin{document}

\maketitle

\tableofcontents

\thispagestyle{fancy} % instead of a `plain` pagestyle this set it to a style which includes the header on the Table of Contents page


%%%%%%%%%%%%%%%%%%%%%%%%%%%%%%%%%%
%%                              %%
%%         Fall Seminar         %%
%%                              %%
%%%%%%%%%%%%%%%%%%%%%%%%%%%%%%%%%%

%page break
\clearpage

\begin{center}
	\section{Fall Seminar}
\end{center}
\stepcounter{cntSemester}

Chinese names are given in two transliteration systems.
The Pinyin system is the one in use today both for ordinary life and academic publications, but the older and obsolete Wade-Giles system can be found in some of our available editions, many of which were published more than fifty years ago.
Both systems are attempts to render the same pronunciations.
\textbf{Please use translations carried by the campus bookstore and/or those specifically mentioned.}

\begin{multicols}{2}

	\printHeaderWithSessionAndDate{fallStart}
	Confucius, \emph{Analects}, \#1-7. Skip commentary where present.

	\printHeaderForNextSeminarClassDate{lastPrintedClassDate}
	Confucius, \emph{Analects}, \#8-13. Skip commentary where present.

	\printHeaderForNextSeminarClassDate{lastPrintedClassDate}
	Confucius, \emph{Analects}, \#14-20. Skip commentary where present.

	\printHeaderForNextSeminarClassDate{lastPrintedClassDate}
	Mozi (Mo Tzu), Fascicles 11, 14-17, 19, 20, 26, 27, 31, 32, 35, 39, and 50.

	Translations by Burton Watson (\emph{Mo Tzu Basic Writings}), Ian Johnston (\emph{The Book of Master Mo}) or Chris Fraser (\emph{The Essential Mozi}) are acceptable.

	\printHeaderForNextSeminarClassDate{lastPrintedClassDate}
	\emph{Mengzi (Meng Tzu, Mencius)}, Books I -- II (1A, 1B, 2A, 2B).

	\printHeaderForNextSeminarClassDate{lastPrintedClassDate}
	\emph{Mengzi (Meng Tzu, Mencius)}, Books III-IV (3A, 3B, 4A, 4B).

	\printHeaderForNextSeminarClassDate{lastPrintedClassDate}
	\emph{Mengzi (Meng Tzu, Mencius)}, Books V -VII (5A, 5B, 6A, 6B, 7A, 7B).

	\printHeaderForNextSeminarClassDate{lastPrintedClassDate}
	Xunzi (Hsun Tzu), Sections 1, 2, 9, 17. Use translations by Burton Watson or Eric L. Hutton.

	\printHeaderForNextSeminarClassDate{lastPrintedClassDate}
	Xunzi (Hsun Tzu), Sections 19-23.

	\printHeaderForNextSeminarClassDate{lastPrintedClassDate}
	Zhuangzi (Chuang Tzu), Chapters 1-3

	\printHeaderForNextSeminarClassDate{lastPrintedClassDate}
	Zhuangzi (Chuang Tzu), Chapters 4-7

	\printHeaderForNextSeminarClassDate{lastPrintedClassDate}
	Zhuangzi (Chuang Tzu), Chapters 17-19, 26

	\printHeaderForNextSeminarClassDate{lastPrintedClassDate}
	Laozi (Lao Tzu), \emph{Dao De Jing (Tao Te Ching),} Chapters 1-37. Commentaries are not necessary.

	\printHeaderForNextSeminarClassDate{lastPrintedClassDate}
	Laozi (Lao Tzu), \emph{Dao De Jing (Tao Te Ching),} Chapters 38-81. Commentaries are not necessary.

	\printHeaderForNextSeminarClassDate{lastPrintedClassDate}
	Han Feizi (Han Fei Tzu), Fascicles 20, 21. Available in EC Manual.

	\printHeaderForNextSeminarClassDate{lastPrintedClassDate}
	\emph{Han Feizi~: Basic Writings,} Translated by Burton Watson. Sections 5-10.
	%FIXME: the ~ after Feizi? accent mark intended?

	\printHeaderForNextSeminarClassDate{lastPrintedClassDate}
	\emph{Han Feizi~: Basic Writings,} Translated by Burton Watson. Sections 12, 13, 17, 18, 49, 50.

	\printHeaderForNextSeminarClassDate{lastPrintedClassDate}
	\emph{The Rig Veda~: An Anthology} Translated by Wendy Doniger. Sections 10.129, 10.121, 10.90, 10.130, 10.190, 10.81-82, 10.72, 10.14, 10.16, 10.18, 10.154, 10.135, 10.58, 10.71, 10.125, 10.101, 10.151, 1.164, 1.163, 1.162, 10.56

	\printHeaderForNextSeminarClassDate{lastPrintedClassDate}
	\emph{The Rig Veda~: An Anthology}. Sections 1.1, 1.26, 5.2, 2.35, 10.51, 10.124, 10.5, 8.79, 9.74, 4.58, 8.48, 10.136, 4.18, 10.28, 1.32, 2.12, 5.83, 7.101, 1.50, 1.160, 1.185, 6.70; 10.10, 1.179, 10.95, 10.85.

	\printHeaderForNextSeminarClassDate{lastPrintedClassDate}
	\emph{Bṛhadāraṇyaka Upaniṣad,''} Parts I \& II

	\printHeaderForNextSeminarClassDate{lastPrintedClassDate}
	\emph{Bṛhadāraṇyaka Upaniṣad,''} Parts III \& IV

	\printHeaderForNextSeminarClassDate{lastPrintedClassDate}
	\emph{Bṛhadāraṇyaka Upaniṣad,''} Parts V \& VI

	\printHeaderForNextSeminarClassDate{lastPrintedClassDate}
	\begin{center}
		\textbf{\textit{[ESSAYS DUE]}}
	\end{center}
	``Katha Upanisad''

	\printHeaderForNextSeminarClassDate{lastPrintedClassDate}
	% FIXME: what is this from?
	``Kena Upaniṣad'' and ``Mundaka Upanisad''

	\printHeaderForNextSeminarClassDate{lastPrintedClassDate}
	%FIXME: don't italicize?
	\emph{``The Nyāya Sūtra,''} in \emph{A Sourcebook in Indian Philosophy}, edited by Sarvepalli Radhakrishnan and Charles A. Moore pp. 358 -- 379.24.

	\printHeaderForNextSeminarClassDate{lastPrintedClassDate}
	%FIXME: don't italicize?
	% FIXME: ṁ Missing character: There is no ṁ (U+1E41) in font [lmroman10-italic]:+tlig;!
	\emph{``The Vaiśeṣika Sūtra''} and \emph{``The Padārthadharmasaṁgraha,'' in A Sourcebook in Indian Philosophy}, pp. 387 -- 423.

	\printHeaderForNextSeminarClassDate{lastPrintedClassDate}
	Tattva-Kaumudî, karikas 1 -- 29. Including commentary by Vacaspati Misra. Available in EC Manual.

	\printHeaderForNextSeminarClassDate{lastPrintedClassDate}
	Tattva-Kaumudî, karikas 30 -- End.

	\printHeaderForNextSeminarClassDate{lastPrintedClassDate}
	``The Yoga Philosophy of Patañjali'' in \emph{A Sourcebook in Indian Philosophy}, pp. 454 - 485

	\printHeaderForNextSeminarClassDate{lastPrintedClassDate}
	\emph{The Bhagavadgītā in the Mahābhārata} translated by J.A.B. van Buitenen. Pp. 39-107

	\printHeaderForNextSeminarClassDate{lastPrintedClassDate}
	\emph{The Bhagavadgītā in the Mahābhārata} translated by J.A.B. van Buitenen. Pp. 107 - 157

\end{multicols}


%%%%%%%%%%%%%%%%%%%%%%%%%%%%%%%%%%
%%                              %%
%%       Spring Seminar         %%
%%                              %%
%%%%%%%%%%%%%%%%%%%%%%%%%%%%%%%%%%

%page break
\clearpage

% increment semester count
\stepcounter{cntSemester}


\begin{center}
	\section{Spring Seminar}
\end{center}

\begin{multicols}{2}
	%\raggedcolumns


	\printHeaderWithSessionAndDate{springStart}
	Kālidāsa, \emph{Kumārasaṃbhava}, in \emph{The Origin of the Young God}, translated by Hank Hifetz. Available in EC Manual.

	\printHeaderForNextSeminarClassDate{lastPrintedClassDate}
	Kālidāsa, \emph{The Recognition of Śhakuntalā,} translated by W.J. Johnson.

	\printHeaderForNextSeminarClassDate{lastPrintedClassDate}
	``The \emph{Dhvanyāloka} of Ānandavardhana with the \emph{Locana} of Abhinavagupta'' selections with supplemental material by Keith and Perry. Through pp. 119. Available in EC Manual.

	\printHeaderForNextSeminarClassDate{lastPrintedClassDate}
	``The \emph{Dhvanyāloka} of Ānandavardhana with the \emph{Locana} of Abhinavagupta'' selections with supplemental material by Keith and Perry. Sections through pp. 696. Available in EC Manual.

	\printHeaderForNextSeminarClassDate{lastPrintedClassDate}
	% Missing character: There is no ṁ (U+1E41) in font [lmroman10-italic]:+tlig;!
	% Missing character: There is no ṁ (U+1E41) in font [lmroman10-regular]:+tlig;!
	%FIXME: ṁ doesn't print right
	``Pūrva Mīmāṁsā'' in \emph{A Sourcebook in Indian Philosophy} edited by Sarvepalli Radhakrishnan and Charles A. Moore. Pp. 486 -- 505.

	\printHeaderForNextSeminarClassDate{lastPrintedClassDate}
	``Cārvāka'' in \emph{A Sourcebook in Indian Philosophy.} Pp. 227 -- 249.
	%FIXME: pp vs Pp 

	\printHeaderForNextSeminarClassDate{lastPrintedClassDate}
	``Discourses on the Nobel Quest,'' ``Discourse to Kālāmas,'' and ``The Greater Discourse on Cause'' from \emph{Early Buddhist Discourses} by John Holder.

	\printHeaderForNextSeminarClassDate{lastPrintedClassDate}
	``The Greater Discourse on the Foundations of Mindfulness,'' ``The Greater Discourse on the Destruction of Craving,'' ``Discourse of the Honeyball'' from \emph{Early Buddhist Discourses.}

	\printHeaderForNextSeminarClassDate{lastPrintedClassDate}
	``Short Discourses from the Saṃyutta Nikāya,'' ``The Shorter Discourse to Māluṅkyaputta,'' ``Discourse on the Parable of the Water Snake,'' ``Discourse to Vacchagotta on Fire,'' from \emph{Early Buddhist Discourses.}

	\printHeaderForNextSeminarClassDate{lastPrintedClassDate}
	``Discourse to Prince Abhaya,'' ``Discourse to Poṭṭhapāda,'' ``Discourse on the Threefold Knowledge,'' from \emph{Early Buddhist Discourses.}

	\printHeaderForNextSeminarClassDate{lastPrintedClassDate}
	``Discourse to Assalāyana,'' ``Discourse to the Layman Sigāla,'' from \emph{Early Buddhist Discourses. Samannaphala Sutta} (``Fruits of the Homeless Life''), available in EC Manual.

	\printHeaderForNextSeminarClassDate{lastPrintedClassDate}
	\emph{The Lotus Sutra}, Translated by Burton Watson. Chapters 1 (prose portion), 2-7 (only the parable in the last nine paragraphs of the final prose section, beginning with ``Monks, you must understand this.''), and 8.

	\printHeaderForNextSeminarClassDate{lastPrintedClassDate}
	\emph{The Lotus Sutra}, Translated by Burton Watson. Chapters 10 (prose portion), 11 (verse only), 12-14, 16, 20, 21, 23, 25, 28 (all prose portion).

	\printHeaderForNextSeminarClassDate{lastPrintedClassDate}
	Nāgārjuna, Chapters 1, 2, 7.

	Translation options:

	(A)\emph{Mūlamadhyamakakārikā of Nāgārjuna: The
		Fundamental Wisdom of the Middle Way}, translated by Jay L. Garfield OR

	(B)\emph{The Philosophy of the Middle Way} translation by David J.
	Kalupahana.

	\printHeaderForNextSeminarClassDate{lastPrintedClassDate}
	Nāgārjuna, Chapters 10, 12, 17, 18

	\printHeaderForNextSeminarClassDate{lastPrintedClassDate}
	Nāgārjuna, Chapters 24, 25, 27

	\printHeaderForNextSeminarClassDate{lastPrintedClassDate}
	Vimalakīrti Sūtra, \emph{The Holy Teaching of Vimalakīrti}, translated by Thurman. Sections 1 -- 6.

	\printHeaderForNextSeminarClassDate{lastPrintedClassDate}
	Vimalakīrti Sūtra, \emph{The Holy Teaching of Vimalakīrti}, translated by Thurman. Sections 7 -- End.

	\printHeaderForNextSeminarClassDate{lastPrintedClassDate}
	Gauḍapāda, \emph{``Māndukya} Upanishad with the Kārikā of Gauḍapāda\emph{''} in Vol. II of \emph{The Upanishads} translated by Swami Nikhilananda (Ramakrishna-Vivekananda Center), pp. 223 -- 368.
	Read both the Māndukya Upanishad and Gauḍapāda commentary. Ignore other commentaries.
	In Chapter 1, after the Hari Om paragraph, read the paragraph following each heading numeral, following the woven threads of Upanishad and Karika (Gaudapada's commentary). For example, ``Now the Upanishad is resumed,'' or ``Now the Karika is resumed.'' From Chapter 2 on, the paragraph following the number is Gaudapada.

	\printHeaderForNextSeminarClassDate{lastPrintedClassDate}
	Saṅkarâkârya's commentary on the Vedânta Sūtras. pp. 2 -- 67, \emph{Sacred Books of the East, Vol. 34,} translated by George Thibaut.
	%FIXME: use same language in everyplace for things in EC Manual. this made me think it was a seperate purchases
	Available as a photocopy in the bookstore.
	% FIXME: add diacritics to purvapakshin and names?
	In reading Sankara and Rāmānujā pay close attention to whether you are hearing their own (Siddhanta) positions or their opponents' (Purvapaksha) arguments. The latter can be lengthy and detailed (and plausible).

	\printHeaderForNextSeminarClassDate{lastPrintedClassDate}
	Saṅkara, pp. 183 -- 191, 199 -- 216, 283 -- 289, 299 -- 308, 312 -- 320, 355 -- 362.

	\printHeaderForNextSeminarClassDate{lastPrintedClassDate}
	Rāmānujā's commentary on the Vedânta Sūtras, pp. 3 -- 73, \emph{Sacred Books of the East, Vol. 48}, translated by George Thibaut. Available in EC Manual

	\printHeaderForNextSeminarClassDate{lastPrintedClassDate}
	\begin{center}
		\textbf{\textit{[ESSAYS DUE]}}
	\end{center}
	Rāmānujā's commentary on the Vedânta Sūtras, pp. 102 -119, 129 -- 138, 255- 273, 296- 308.

	\printHeaderForNextSeminarClassDate{lastPrintedClassDate}
	Jayadeva's ``The \emph{Gītagovinda}'' in \emph{Love Song of the Dark Lord}, edited and translated by Barbara Stoler Miller, pp. 69 -- 125.

	\printHeaderForNextSeminarClassDate{lastPrintedClassDate}
	``The \emph{Diamond Sūtra}'' in \emph{The Diamond Sūtra \& the Sūtra of Hui-Neng}, translated by A.F. Price and Wong Mou-lam (Shambhala).

	``The \emph{Heart Sūtra''} pp. 17 -- 53. Available in EC Manual.

	\printHeaderForNextSeminarClassDate{lastPrintedClassDate}
	Hui-Neng, ``Commentary on the Diamond Sūtra,'' in \emph{The Sutra of Hui-neng, Grand Master of Zen: With Hui-neng\textquotesingle s Commentary on the Diamond Sutra}, translated by Thomas Cleary (Shambhala), pp. 85-144.

	\printHeaderForNextSeminarClassDate{lastPrintedClassDate}
	Hui-Neng, \emph{The Platform Sutra of the Sixth Patriarch}, translated by Philip Yampolsky (Columbia University Press), paragraphs 1 -- 30.

	\printHeaderForNextSeminarClassDate{lastPrintedClassDate}
	Hui-Neng, \emph{The Platform Sutra of the Sixth Patriarch}, 31 -- 57.

	\printHeaderForNextSeminarClassDate{lastPrintedClassDate}
	``The \emph{Great Learning}'' and ``The \emph{Doctrine of the Mean},'' in \emph{A Sourcebook in Chinese Philosophy}, translated and edited by Wing-tsit Chan, pp. 84 -- 114.

	\printHeaderForNextSeminarClassDate{lastPrintedClassDate}
	Zhu Xi (Chu Hsi), ``The Complete Works of Chu Hsi'' in \emph{A Sourcebook in Chinese} \emph{Philosophy}, pp. 605 -- 632 (stop at \emph{Jen}).

	\printHeaderForNextSeminarClassDate{lastPrintedClassDate}
	Zhu Xi (Chu Hsi), ``The Complete Works of Chu Hsi'' in \emph{A Sourcebook in Chinese Philosophy}, pp. 632-653, 693-604.

	\columnbreak %this is an unfortunate neccessity, otherwise the last few numbers of the previous section get pushe dover into the next column by themselves

	\printHeaderForNextSeminarClassDate{lastPrintedClassDate}
	Wang Yang-Ming, ``Inquiry on the Great Learning'' and selections from ``Instructions for Practical Living,'' in \emph{A Sourcebook in Chinese Philosophy}, pp. 659 -- 691.

\end{multicols}

%%%%%%%%%%%%%%%%%%%%%%%%%%%%%%%%%%
%%                              %%
%%       Summer Seminar         %%
%%                              %%
%%%%%%%%%%%%%%%%%%%%%%%%%%%%%%%%%%

%page break
\clearpage

% increment semester count
\stepcounter{cntSemester}

\begin{center}
	\section{Summer Seminar}
\end{center}


\begin{multicols}{2}


	\printHeaderWithSessionAndDate{summerStart}
	\emph{The Tale of the Heike,} translated by Helen Craig McCullough. Chapters 1 -- 8.

	\printHeaderForNextSeminarClassDate{lastPrintedClassDate}
	\emph{The Tale of the Heike,} Chapters 9 -- 10.

	\printHeaderForNextSeminarClassDate{lastPrintedClassDate}
	\emph{The Tale of the Heike,} Chapters 11 -- end.

	\printHeaderForNextSeminarClassDate{lastPrintedClassDate}
	Kūkai \emph{Major Works} translated by Yoshito S. Hakeda (Columbia University Press), ``The Difference Between Exoteric and Esoteric Buddhism'', ``Attaining Enlightenment in This Very Existence, ``The Meanings of Sound, Word, and Reality.'' Available in EC Manual.

	\printHeaderForNextSeminarClassDate{lastPrintedClassDate}
	Sei Shōnagon, \emph{The Pillow Book}, translated by Meredith McKinney (Penguin Classics), sections 1-100, pp. 3-113

	\printHeaderForNextSeminarClassDate{lastPrintedClassDate}
	\emph{The Pillow Book}, sections 101-206, 243, 258, 273, pp. 113-190

	\printHeaderForNextSeminarClassDate{lastPrintedClassDate}
	Kamo no Chōmei, ``Hōjōki or the Record of the Ten-Foot Square Hut''

	\emph{Kenkō and Chōmei: Essays in Idleness and Hōjōki,} translated by
	Meredith McKinney (Penguin) OR

	\emph{Four Huts: Asian Writings on the Simple Life}, translated by
	Burton Watson (Shambhala)

	\printHeaderForNextSeminarClassDate{lastPrintedClassDate}
	Dōgen, ``Bendōwa,'' in \emph{The Heart of Dōgen's Shōbōgenzō}, translated by Waddell and Abe (SUNY Press).

	\printHeaderForNextSeminarClassDate{lastPrintedClassDate}
	Dōgen, ``Busshō,'' in \emph{The Heart of Dōgen's Shōbōgenzō}, pp. 59 -- bottom of 84.

	\printHeaderForNextSeminarClassDate{lastPrintedClassDate}
	Dōgen, ``Busshō,'' in \emph{The Heart of Dōgen's Shōbōgenzō}, pp. 84 -- 98.

	\printHeaderForNextSeminarClassDate{lastPrintedClassDate}
	Dōgen, ``Genjōkōan,'' in \emph{The Heart of Dōgen's Shōbōgenzō}

	\printHeaderForNextSeminarClassDate{lastPrintedClassDate}
	Dōgen, ``Uji,'' in \emph{The Heart of Dōgen's Shōbōgenzō}

	\printHeaderForNextSeminarClassDate{lastPrintedClassDate}
	Kenkō, sections 1 -- 38, 43 -- 49, 52, 53, 58 -- 60, 66. In \emph{Kenkō and Chōmei: Essays in Idleness and Hōjōki,} translated by Meredith McKinney (Penguin)

	OR

	\emph{Essays in Idleness}, translated by Donald Keene (Columbia University Press)

	\printHeaderForNextSeminarClassDate{lastPrintedClassDate}
	Kenkō, sections 69 -- 75, 81 -- 85, 89, 92, 97 -- 98, 104 -- 122, 127 -- 130, 133, 137, 154, 162, 166, 184, 188, 190,191, 235 -- 237.

	\printHeaderForNextSeminarClassDate{lastPrintedClassDate}
	Bashō, ``Journey of Bleached Bones in a Field,'' ``Kashima Journal,'' ``Knapsack Notebook, ``Sarashina Journal'' in Bashō's Journey: The Literary Prose of Matsuo Bashō, translated by David Landis Barnhill (SUNY Press)

	\printHeaderForNextSeminarClassDate{lastPrintedClassDate}
	Bashō, ``The Narrow Road to the Deep North.'' in Bashō's Journey: The Literary Prose of Matsuo Bashō, translated by David Landis Barnhill (SUNY Press)

\end{multicols}


%%%%%%%%%%%%%%%%%%%%%%%%%%%%%%%%%%
%%                              %%
%%      Fall Preceptorial 1     %%
%%                              %%
%%%%%%%%%%%%%%%%%%%%%%%%%%%%%%%%%%

%page break
\clearpage

% increment semester count
\stepcounter{cntSemester}

\begin{center}
	\section{Fall Preceptorial – Sima Qian, \textit{Records of the Grand Historian (Shi Ji)}}
	 (First Eight Weeks)
\end{center}

% TODO: make list below
\textbf{Required Texts:}
\begin{itemize}
	\item (In EC Manual): “The Grand Scribes Records” Vol 1, translated by Neinhauser, Jr. (pp 1-86)
	\item \textit{Records of the Grand Historian}, translated by Burton Watson (Columbia University Press), Volumes:
	      \begin{itemize}
		      \item \textit{Quin Dynasty} (ISBN: 0231081693)
		      \item \textit{Han Dynasty I} (ISBN: 0231081650)
		      \item \textit{Han Dynasty II} (ISBN: 0231081677)
	      \end{itemize}
\end{itemize}


\begin{multicols}{2}

	\printHeaderWithSessionAndDate{fallPrecept1Start}
	% FIXME: add detail that it is in EC manual?
	\emph{Shi Ji} Ch 1 -- 2 (Nienhauser, Vol 1)

	\printHeaderForNextPreceptClassDate{lastPrintedClassDate}
	\emph{Shi Ji} Ch 3 - 4 (Nienhauser, Vol 1)

	\printHeaderForNextPreceptClassDate{lastPrintedClassDate}
	% FIXME: italicize titles
	\emph{Shi Ji} Ch 6 (Quin Dynasty, pp. 35-83)

	\printHeaderForNextPreceptClassDate{lastPrintedClassDate}
	\emph{Shi Ji} Ch 15, 61, 68, 79 (Quin Dynasty, pp. 85-99, 131-157, photocopy of section 61)

	\printHeaderForNextPreceptClassDate{lastPrintedClassDate}
	% FIXME: add note location letter, pages? preface?
	\emph{Shi Ji} Ch 85-88, 126 (Quin Dynasty, pp. 159-216, plus Sima Qian's Letter to Ren An)

	\printHeaderForNextPreceptClassDate{lastPrintedClassDate}
	\emph{Shi Ji} Ch 48, 7 (Han Dynasty I, pp. 1-38)

	\printHeaderForNextPreceptClassDate{lastPrintedClassDate}
	\emph{Shi Ji} Ch 8, 16 (Han Dynasty I, pp. 51-88)

	\printHeaderForNextPreceptClassDate{lastPrintedClassDate}
	\emph{Shi Ji} Ch 53, 55, 56 (Han Dynasty I, pp 91-128)

	\printHeaderForNextPreceptClassDate{lastPrintedClassDate}
	\emph{Shi Ji} Ch 89-91 (Han Dynasty I, pp 131-162)

	\printHeaderForNextPreceptClassDate{lastPrintedClassDate}
	\emph{Shi Ji} Ch 92-94 (Han Dynasty I, pp 163-202)

	\printHeaderForNextPreceptClassDate{lastPrintedClassDate}
	\emph{Shi Ji} Ch 9-12, 106 (Han Dynasty I, 267-319, 403-422)

	\printHeaderForNextPreceptClassDate{lastPrintedClassDate}
	\emph{Shi Ji} Ch 28 (Han Dynasty II, 3-52)

	\printHeaderForNextPreceptClassDate{lastPrintedClassDate}
	\emph{Shi Ji} Ch 110 (Han Dynasty II, 129-162)

	\printHeaderForNextPreceptClassDate{lastPrintedClassDate}
	\emph{Shi Ji} Ch 117 (Han Dynasty II, 259-306)

	\printHeaderForNextPreceptClassDate{lastPrintedClassDate}
	\emph{Shi Ji} Ch 121, 119, 122 (Han Dynasty II, 355-407)

	\printHeaderForNextPreceptClassDate{lastPrintedClassDate}
	\emph{Shi Ji} Ch 124, 125, 127, 129 (Han Dynasty II, 409-454)

\end{multicols}


%%%%%%%%%%%%%%%%%%%%%%%%%%%%%%%%%%
%%                              %%
%%      Fall Preceptorial 2     %%
%%                              %%
%%%%%%%%%%%%%%%%%%%%%%%%%%%%%%%%%%

%page break
\clearpage

% increment semester count
\stepcounter{cntSemester}

\begin{center}
	\section{Fall Preceptorial - Mahābhārata}
	 (Second Eight Weeks of Semester)
\end{center}

\textbf{Required Texts}:
\begin{itemize}
	\item \textit{Mahābhārata} translated by J.A.B. van Buitenen (University of Chicago Press)
	      \begin{itemize}
		      \item Volume 1, \textit{The Book of the Beginning} (ISBN: 9780226846637)
		      \item Volume 2, \textit{The Book of the Assembly Hall, The Book of the Forest} (ISBN: 9780226846644)
		      \item Volume 3, \textit{The Book of the Virata, The Book of the Effort} (ISBN: 9780226846651)
	      \end{itemize}
	\item \textit{The Sauptikaparvan of the Mahbharata} translated by W.J. Johnson (Oxford World’s Classics) (ISBN: 978-0192823618)
	      \begin{itemize}
		      \item Out of print, but photocopies are available in the EC Manual sold in the bookstore
	      \end{itemize}
\end{itemize}

\begin{multicols}{2}


	\printHeaderWithSessionAndDate{fallPrecept2Start}
	\emph{The Book of the Beginning}, pp. 19-63

	(Lists -- Puloman)

	\printHeaderForNextPreceptClassDate{lastPrintedClassDate}
	\emph{The Book of the Beginning}, pp. 63-123

	(Astika)

	\printHeaderForNextPreceptClassDate{lastPrintedClassDate}
	\emph{The Book of the Beginning}, pp. 123-210

	(The Descent -- Latter Days of Yayati)

	\small{\emph{(pp 145-154 may be skimmed)}}

	\printHeaderForNextPreceptClassDate{lastPrintedClassDate}
	\emph{The Book of the Beginning}, pp. 210-273

	(Latter Days of Yayati -- The Origins)

	\printHeaderForNextPreceptClassDate{lastPrintedClassDate}
	\emph{The Book of the Beginning}, pp. 274-344

	(The Fire in the Lacquer House - Citraratha)

	\printHeaderForNextPreceptClassDate{lastPrintedClassDate}
	\emph{The Book of the Beginning}, pp. 344-405

	(Draupadi's Bridegroom Choice -- Arjuna's Sojourn in the Forest)

	\printHeaderForNextPreceptClassDate{lastPrintedClassDate}
	\emph{The Book of the Beginning}, pp. 405-431

	(Abduction of Subhadra -- Burning of the Khandava Forest). Discussion of whole of Book 1

	\printHeaderForNextPreceptClassDate{lastPrintedClassDate}

	\emph{The Book of the Assembly Hall:} Read italicized summaries pp.33-106, then read whole pp.106-69 (The Dicing -- The Sequel to the Dicing).

	At this point note that the \emph{Sauptikaparvan} volume has narrative summaries in the back: read summary for Book 4\emph{, The Book of Virata.}

	\printHeaderForNextPreceptClassDate{lastPrintedClassDate}
	\emph{The Book of the Effort}, pp 187-254

	(Embassy of Samjaya)

	\printHeaderForNextPreceptClassDate{lastPrintedClassDate}
	\emph{The Book of the Effort}, pp 254-294

	(Dhrtarastra's Vigil -- Sanatsujata)

	\printHeaderForNextPreceptClassDate{lastPrintedClassDate}
	\emph{The Book of the Effort}, pp 294-339

	(The Suing for Peace)

	\columnbreak
	%\raggedcolumns

	\printHeaderForNextPreceptClassDate{lastPrintedClassDate}
	\emph{The Book of the Effort} pp 344-382

	(The Coming of the Lord);

	\emph{The Book of the Effort} pp 415-442

	(The Coming of the Lord, continued)

	\printHeaderForNextPreceptClassDate{lastPrintedClassDate}
	In Volume 3\emph{, The Book of the Forest,} pp. 780-795

	(The Robbing of the Earrings), then

	% FIXME: make a blank line and don't indent this next line
	\emph{The Book of the Effort}, pp. 442-461

	(The Temptation of Karna)

	\printHeaderForNextPreceptClassDate{lastPrintedClassDate}
	\emph{The Book of the Effort}, pp.461-532

	(The Marching Out -- Amba)

	\printHeaderForNextPreceptClassDate{lastPrintedClassDate}
	\emph{The Sauptikaparvan}, read Summaries in Appendix, then main text pp 5 -- 86

\end{multicols}


%%%%%%%%%%%%%%%%%%%%%%%%%%%%%%%%%%
%%                              %%
%%      Summer Preceptorial     %%
%%                              %%
%%%%%%%%%%%%%%%%%%%%%%%%%%%%%%%%%%

%page break
\clearpage

% increment semester count
\stepcounter{cntSemester}

\begin{center}
	\section{Summer Preceptorial – The Tale of Genji}
\end{center}

\textbf{Required Text:}

\emph{The Tale of Genji} by Murasaki Shikibu, translated by Edward G. Seidensticker (ISBN: 978-0679417385)

\begin{multicols}{2}

	\printHeaderWithSessionAndDate{summerPreceptStart}
	Chapters 1-4 (80 pages)

	\printHeaderForNextPreceptClassDate{lastPrintedClassDate}
	Chapters 5-6 (47 pages)

	\printHeaderForNextPreceptClassDate{lastPrintedClassDate}
	Chapters 7-10 (82 pages)

	\printHeaderForNextPreceptClassDate{lastPrintedClassDate}
	Chapters 11-13 (55 pages)

	\printHeaderForNextPreceptClassDate{lastPrintedClassDate}
	Chapters 14-19 (76 pages)

	\printHeaderForNextPreceptClassDate{lastPrintedClassDate}
	Chapters 20-22 (60 pages)

	\printHeaderForNextPreceptClassDate{lastPrintedClassDate}
	Chapters 23-29 (72 pages)

	\printHeaderForNextPreceptClassDate{lastPrintedClassDate}
	Chapters 30-33 (54 pages)

	\printHeaderForNextPreceptClassDate{lastPrintedClassDate}
	Chapters 34-35 (98 pages)

	\printHeaderForNextPreceptClassDate{lastPrintedClassDate}
	Chapters 36-38 (39 pages)

	\printHeaderForNextPreceptClassDate{lastPrintedClassDate}
	Chapters 39-43 (74 pages)

	\printHeaderForNextPreceptClassDate{lastPrintedClassDate}
	Chapters 44-46 (69 pages)

	\printHeaderForNextPreceptClassDate{lastPrintedClassDate}
	Chapters 47-48 (63 pages)

	\printHeaderForNextPreceptClassDate{lastPrintedClassDate}
	Chapters 49-50 (87 pages)

	\printHeaderForNextPreceptClassDate{lastPrintedClassDate}
	Chapters 51-52 (70 pages)

	\printHeaderForNextPreceptClassDate{lastPrintedClassDate}
	Chapters 53-54 (47 pages)

\end{multicols}

\textit{Note: During the spring term students will be provided with a list of options for preceptorials.}

%%%%%%%%%%%%%%%%%%%%%%%%%%%%%%%%%%
%%                              %%
%%          Book List           %%
%%                              %%
%%%%%%%%%%%%%%%%%%%%%%%%%%%%%%%%%%

%page break
\clearpage
\section{Booklist}

Books for Seminar are listed in order they are encountered in class.

TODO...?

% TODO: add QR code to some survey/form through which people can easily submit changes/errors/etc?

\end{document}
