\documentclass[12pt]{book}

\usepackage{indentfirst}

% TODO: don't have blank pages at beginning, don't start new chapter on right side/skip pages

% if set to zero, do not print numbers before section headers
\setcounter{secnumdepth}{0}

\title{Dhvanyāloka of Ānandavardhana \\ `The Light of Suggestion'}
\author{Critically edited with Introduction, Translation \& Notes \\ By Dr.\ K. Krishnamoorthy}
\date{1982}

% NOTE: 
% Section of IMP text provided in Eastern Classics Manual 
% Translator’s Introduction (15–17);  
% 1.1 K, A, L (47–49 partial);  
% –skipping–;  
% 1.1 L (partial 51–52 partial);  
% 1.1a A & L (54–58);  
% –skipping–;  
% 1.1e A & L (partial 67–73);  
% –skipping–;  
% 1.4 K, A, L, 1.4a A & L, 1.4b A (78–84);  
% 1.4b L partial text (84);  
% 1.4b L partial text & footnotes (92);  
% 1.4c A & L (98–99);  
% –skipping–;  
% 1.4g A & L, 1.5 K, A, L (105–119);  
% –skipping–;  
% 1.18 K, A, L (188–196);  
% –skipping–;  
% 2.3–2.5 K, A, L {for all in this range} (214–233);  
% –skipping–;  
% 2.7–2.10 K, A, L {for all in this range} (251–260);  
% –skipping–;  
% 4.5 K, A (690–696);  
% 4.5 L partial text (696) 

\begin{document}

\maketitle
% \title{THE LIGHT OF SUGGESTION}


\chapter{THE FIRST FLASH}


% TODO: center block of text, not each line - box?
\begin{quotation}
\begin{em}
May Lord Hari's claws preserve you

In his Lion's form self-adopted;

They outshine the moon in clear hue

And destroy the woes of the devoted.
\end{em}
\end{quotation}

% NOTE: Line breaks kept from original document, except where they would introduce formatting errors.


\section{I, Kārikā 1}

Though the learned men of yore have declared time and
again that the soul of poetry is suggestion, some would aver its
non-existence, some would regard it as something (logically)
implied and some others would speak of its essence as lying
beyond the scope of words. We propose, therefore, to explain
its nature and bring delight to the hearts of perceptive critics.


\section{I, Vṛtti 1}

The word learned men has the sense of those who know
the truth about poetry. Through an unbroken tradition, these
have taught that the soul of poetry has been named Suggestion.
Although it is felt so by cultured critics in their minds (even
to-day), others affirmed its non-existence. The following are
the different views of those who believe in its non-existence:


\subsection{I, Vṛtti 1 continued --- 1.1a A in IMP}

According to some (of the objectors): ``Poetry is but
that whose body is constituted by sound (or word) and meaning.
Sources of charm through sound such as `alliteration' are well-known; and so are the sources of charm through meaning such
as `simile'. Merits or qualities of composition like `sweetness'
are also familiar to us. Also we have heard of dictions such as
the `cultured' propounded by some, though in truth their features
are no different from qualities of style. We have further heard
of styles like \textit{Vaidarbhī}. But what could this concept of DHVANI
(suggestion) be which is different from any of these?”


\subsection{I, Vṛtti 1 continued --- 1.1b A in IMP (skipped in EC Manual)}

Others assert thus: `` `Suggestion' does not exist indeed;
for a species of poetry opposed to all well-known canons will
necessarily cease to be poetry. Poetry can only be defined as
that which is made up of such words and meanings as will delight
the mind of the critic. This will not be achieved by a route which
excludes all the well-known canons mentioned. Even if the
designation of poetry were to be accepted as applying to DHVANI on
the unanimous support of a coterie of self-styled critics, it would
fail to win the acceptance of all the learned.''


\subsection{I, Vṛtti 1 continued --- 1.1c A in IMP (skipped in EC Manual)}

Yet another opinion about its non-existence is: ``It is
indeed impossible that `suggestion' can be something unknown
before. Since it is not distinct from a source of charm, it gets
naturally included in the causes of charm already enunciated.
By coining a novel designation to just one of them nothing
profound will have been stated. Moreover, since the ways
of speech are endless, even if there should be an insignificant
element left unexplained by the famous framers of the rules of
poetry, we cannot understand the reason why persons should
close their eyes under the self-assumed illusion of being `perceptive
critics' and dance about with joy saying that they have discovered DHVANI therein. Thousands of other great men
have expounded, and are still expounding, figurative elements
(of speech). But we do not hear of any such over-excitement
on their part. Therefore, DHVANI is but a fabrication; and it
would be impossible to demonstrate any truth about it which can
bear scrutiny. In fact, a gentleman has already composed a
verse to this effect:

% TODO: indent some, as it is a quote?
\begin{em}
Poetry, wherein there is nothing to delight the mind and no
embellishment, which is destitute of felicitous words and
artful turns, is praised so warmly by the dunce as being
endowed with 
\end{em}
DHVANI
\begin{em}
(Suggestion). But we are at a loss to
imagine what answer he would give when faced with
a straight question by an intelligent critic about the nature of
\end{em}
DHVANI
\begin{em}
itself!
\end{em}


\subsection{I, Vṛtti 1 continued --- 1.1d A in IMP (skipped in EC Manual)}

Some others mention it as something (logically) implied.
(To put it differently,) others declare that the soul of poetry,
designated by the term Suggestion, is the same as a secondary
usage of words. Although it is true that no literary theorist has
ever shown any element like a secondary usage of words as being
specifically identical with Suggestion by mentioning the word
`Suggestion' itself, we have noted here such a view because we
can conclude that one who points out the secondary usage of
words in poetry has slightly touched the fringe of the doctrine
of Suggestion, though one does not define it.


\subsection{I, Vṛtti 1 continued --- 1.1e A in IMP}

Still others, not astute enough to frame a definition, rest
content with saying that the true nature of suggestion is beyond
all words and that it is discernible only to the minds of
perceptive critics.

In view of the prevalence of so many conflicting opinions,
we propose to elucidate the nature of Suggestion for the delight
of the perceptive critics.

Suggestion itself is both the quintessence of the works of all
first-rate poets and the most beautiful principle of poetry though it
remained unnoticed even by the subtlest of the rhetoricians
of the past. However, refined critics are certainly alive to its
primary presence in literary works like the \textit{Rāmāyaṇa} and the
\textit{Mahābhārata}; and with a view to placing their delight on a
secure footing, we shall explain its nature (in detail).


\subsection{I, Vṛtti 1 continued --- 1.1f A in IMP}

The following is meant to serve as a groundwork for the
theory of Suggestion which has been taken up for a detailed
study:---


\section{I, Kārikā 2 (skipped in EC Manual)}

That meaning which wins the admiration of refined critics
is decided to be the soul of poetry. The `explicit' and the
`implicit' are regarded as its two aspects.


\section{I, Vṛtti 2 (skipped in EC Manual)}

That meaning which wins the admiration of perceptive
critics and which is of the very essence of poetry---even as the
soul is of a body which is naturally handsome by the union of
graceful and proper limbs---has two aspects, viz., the explicit
and the implicit.


\section{I, Kārikā 3 (skipped in EC Manual)}

Of these, the explicit is commonly known and it has been
already set forth in many ways through figures of speech such
as the simile by other writers; hence it need not be discussed
here at length.


% TODO: change from book, to put ` inside the parens?
\section{I, Vṛtti 3 (skipped in EC Manual)}

The expression, `\textit{other writers}', alludes to writers on poetics
(like Bhaṭṭa Udbhaṭa). `(\textit{Hence it need not be discussed here at
length}' should be taken to imply that) the conclusions of earlier
writers will be freely quoted whenever a need arises for them.


\section{I, Kārikā 4}

But the implicit aspect is quite different from this. In the
words of first-rate poets it shines supreme and towers above
the beauty of the striking external constituents even as charm
in ladies.


\section{I, Vṛtti 4}

The implicit aspect is entirely different from the explicit
aspect and it is found in the words of first-rate poets. It is
most familiar to the minds of refined critics and it shines forth as
being over and above the `striking external constituents'. The
expression `striking' connotes not only what is `adorned with
figures' but also what is `perceptible to the senses'. Charm in
ladies is a simile in point. Just as charm in ladies exceeds the
beauty of all the individual limbs observed separately, and
delights like ambrosia the eye of the admirer in a most unique
fashion, so also does this implicit meaning. 


\subsection{I, Vṛtti 4 continued --- 1.4a A in IMP}

It will be shown in
the sequel that this meaning embraces various divisions such as
the bare idea, figures and sentiments, all implied by the inner
power of the explicit. In all these varieties, it will be seen to
differ from the explicit. To illustrate: even the first variety itself
differs very widely from the explicit. Sometimes the implicit
meaning will be of the nature of a prohibition when the explicit
is of the nature of positive proposal; 


\subsection{I, Vṛtti 4 continued --- 1.4b A in IMP}
e.g.,

\begin{quotation}
\begin{em}
Ramble freely, pious man!

That dog to-day is killed

By the fierce lion that dwells

In Godā river dells.
\end{em}
\end{quotation}


\subsection{I, Vṛtti 4 continued --- 1.4c A in IMP}

Sometimes, though the explicit meaning is of the nature of
a prohibition, the implicit will be of the nature of a positive
proposal; e.g,.

\begin{quotation}
\begin{em}
Mother-in-law lies here, lost in sleep;

And I here; thou shouldst mark

These before it is dark.

O traveller, blinded by night,

Tumble not into our beds aright.
\end{em}
\end{quotation}


\subsection{I, Vṛtti 4 continued --- 1.4d A in IMP (skipped in EC Manual)}

Sometimes, though the explicit meaning may be of the nature
of a positive proposal, the implicit may be neither a definite
prohibition nor a definite proposal; e.g.,

\begin{quotation}
\begin{em}
Get thee gone! I pray,

May all sighs and tears be mine, I say.

Let them not be thine again

In false courtesy to me.
\end{em}
\end{quotation}


\subsection{I, Vṛtti 4 continued --- 1.4e A in IMP (skipped in EC Manual)}

Sometimes, though the explicit meaning is of the nature of a
prohibition, the implicit is neither a prohibition nor a proposal;
e.g,,

\begin{quotation}
\begin{em}
Humbly I beg thee, please go back;

Indeed, O sweet, thou drivest all gloom

By the bright light of your moon-face.

A pity it is that thou dost harm

The journeys of other wantons

Seeking their lovers' arms.
\end{em}
\end{quotation}


\subsection{I, Vṛtti 4 continued --- 1.4f A in IMP (skipped in EC Manual)}

Sometimes, the implicit meaning will relate to something
entirely different from that to which the explicit is related. e.g.,
% NOTE: "related. e.g.," is as it is in the book.

\begin{quotation}
\begin{em}
Who will not rise in rage

Seeing his beloved's lip wounded?

You heeded not my warning

And kissed the lotus hiding a bee.

Now rightly pay the penalty!
\end{em}
\end{quotation}


\subsection{I, Vṛtti 4 continued --- 1.4g A in IMP}

There are various other forms besides these in which the
varieties of the implicit meaning appear distinct from the explicit.
What has been demonstrated above should be taken only as a
pointer in that direction.

That the second class of the implicit (viz.\ figures of speech)
too differs from the explicit will be demonstrated in detail
later on.

But the third class of the implicit, viz., sentiments etc., is
seen to shine forth as a result of the latent power in the explicit.
It never becomes an object of direct verbal denotation and hence
it is decidedly distinct from the explicit. If at all it could be an
object of the explicit, it might be so alleged either as being
directly denoted by its proper names or as being denoted through
the delineation of characters in a setting, etc. If the first alternative were true, there would be no possibility of an experience
of sentiments etc., in instances where their proper names are not
employed. Never are they so denoted directly by their proper
names. Even when proper names are present, the experience of
sentiments etc.\ is not due to them but only due to the delineation
of characters in a proper setting etc. The experience of sentiments
etc.\ is only given a designation by the proper name and is not at
all conditioned by it. In fact we do not have the experience (of
sentiments etc.) in all the instances where proper names are used.
Indeed, there is not even the slightest experience of the presence
of sentiments in a composition-which contains only their proper
names such as the \textit{Erotic} and which is destitute of all delineation
of the characters in a setting and so forth. Since we can have
the experience of sentiments etc.\ only through the characters in
a setting etc.\ irrespective of their proper names and since we
cannot have the experience only by the use of proper names, we
may conclude on the basis of these considerations, both positive
and negative, that sentiments etc.\ are only implied by the latent
power of the explicit and in no way denoted explicitly. Thus
it is established that even the third class of the implicit meaning
is quite distinct from the explicit. It will be shown in the sequel,
however, that its experience will appear to be almost simultaneous
with the explicit.

\section{I, Kārikā 5}

That meaning alone is the soul of poetry; and so it was
that, of yore, the sorrow of the First Poet (i.e. Vālmīki) at
the separation of the curlew couple took the form of a
distich.

\section{I, Vṛtti 5}

That meaning alone happens to be the quintessence of poetry
whose outward charm is secured by the combination of varied
and uncommon explicit meanings, expressions and art of arrangement. That is why the sorrow of the First Poet, on hearing the
wail of the he-curlew afflicted with separation from its close mate,
`transformed itself into a distich. Sorrow indeed is the abiding
emotion which is at the basis of the sentiment of pathos. As
already explained, it is only of the nature of the implicit. Though
one can discern other sub-species of the implicit, they can all
be understood by the synecdoche of sentiments and emotions
since these happen to be the most important representatives of
the rest.



\section{(I, Kārikā 6--17 are skipped in EC Manual)}

Therefore---


\section{I, Kārikā 18}

The fact is that indication is grounded on the primary
denotative force of words. How can it ever be a definition
of suggestion whose sole support is suggestivity?

\section{I, Vṛtti 18}

Hence suggestion is one thing and indication another.

The definition (that suggestion is indication) contains the
fallacy of Too Narrow also. Indication does not cover instances
of suggestion like `that with intended but further-extending
literal import' as also numerous other instances. Hence, indication cannot be a definition of suggestion.


% \section{I, Kārikā 19a}

% (At the most), it might serve as a pointer to one of the
% species of suggestion. 19 (a)


% \section{I, Vṛtti 19a}

% One could only fancy that indication might serve, if at all,
% as a pointer to just a single species from among the numerous
% varieties of suggestion pointed out in the sequel. With all this,
% if one were to assort dogmatically that suggestion is indeed
% defined by indication, that way, one might say that the act of
% defining individual figures is an utter waste since the primary
% denotation of words defines the entire group of all individual
% figures. What is more---


% \section{I, Kārikā 19b}

% If one were to say that the definition of suggestion has
% already been propounded by others, it would only substantiate
% our own position. 19 (b)


% \section{I, Vṛtti 19b}

% Even if it be true that the definition of suggestion has already
% been propounded by earlier writers, it would only mean a
% substantiation of our own position. For, our position is that
% suggestion exists; and in case it has been established already, we
% should consider ourselves to be extremely fortunate inasmuch as
% our object has already been realised without any labour at all
% on our part.



\chapter{THE SECOND FLASH}

So far, two varieties of suggestion, viz., `that with unintended
literal import' and `that with intended but further-extending
literal import' have been mentioned. Now the sub-varieties of
the first are set forth in what follows:---


\section{II, Kārikā 1 (skipped in EC Manual)}

`Merged in the other meaning' and `Completely lost'---these
are the two kinds of the expressed in `Suggestion with intended
literal import'.


\section{II, Vṛtti 1 (skipped in EC Manual)}

The first, viz., `Merged in the other meaning' is instanced in
the following:---

\begin{quotation}
\begin{em}
The quarters all are painted deep

With the glistening black of clouds,

And the cranes in circles fly (with excitement);

The breezes are moisture-laden

And these friends of clouds, the peacocks,

Send their joyous notes in the wind.

Let them all confront me!

I shall bear them all, as I am Rāma

Whose heart is adamant to be sure;

But how will Sitā fare!

Alas! Alas! My dear queen!

Be bold, I beseech thee.
\end{em}
\end{quotation}

The word Rāma in this example carries the suggestive force
mentioned. The word does not merely denote an individual with
that proper name but conveys the sense of a person endowed with
various qualities by the force of suggestion. An illustration is
also found in my own work, \textit{Viṣamabāṇalīlā}:---

\begin{quotation}
\begin{em}
Merits become merits indeed

When critics of culture hold them so.

Lotuses will be lotuses

Only when sunshine shelters them.
\end{em}
\end{quotation}

Here the word `lotuses' repeated a second time is an instance
in point.

We can cite the following verse of the First Poet, Vālmīki,
to illustrate the second variety, viz., `Suggestion with completely
lost literal import':---

\begin{quotation}
\begin{em}
All the charm to the sun hath fled
And the orb is hid in snow;

Like a mirror by breath blinded,
The moon now does not glow.
\end{em}
\end{quotation}

Here the word `\textit{blinded}' contains the said suggestion. So
also the following verse:---

\begin{quotation}
\begin{em}
The sky with dizzy cloud,

The Arjun woods with rain-drops dripping loud,

And nights with moons not proud,

Though black in hue,

They capture you.
\end{em}
\end{quotation}

Here the words `\textit{dizzy}' and `\textit{not proud}' are full of suggestion.


\section{II, Kārikā 2 (skipped in EC Manual)}

The nature of suggestion `with intended literal import' is
also two-fold: (i) `of discernible sequentiality' and (ii) `of
undiscernible sequentiality'. 


\section{II, Vṛtti 2 (skipped in EC Manual)}

The nature of suggestion is the implied sense which is communicated prominently. A variety of it is grasped simultaneously with
the expressed, since the sequentiality existing between the two is
not discernible. Another variety of the same comes about when
the sequentiality is discernible. Of these two,


\section{II, Kārikā 3}

% TODO: add in sanskrit terms from IMP?? here and other places??
% rasa, bhāva, rasābhāsa, bhāvābhāsa, bhāvapraśānti,
% dhvani
Sentiment, emotion, the semblance of sentiment or mood
and their (rise and) cessation etc., are all of `undiscerned
sequentiality'. It is decided that when we have the prominent
presence of this variety, we are having the very soul of suggestion.


\section{II, Vṛtti 3}

Categories like sentiment shine forth along with the literal
import. If they shine also with prominence we have the very
soul of suggestion.

\subsection{II, Vṛtti 3 continued --- 2.4 Introduction A in IMP}

It will be shown in what follows that the sphere of this
suggestion `of undiscerned sequentiality' is quite distinct from
that of the figure of speech called \textit{Rasavadalankāra} or Figurative
Sentiment:---

% TODO: keep going with OCR checks through gImageReader, stopped after image20


\section{II, Kārikā 4}

Only that, wherein all the several beautifiers of the expressed
sense and the expression exist with the single purpose of conveying sentiment and so on, is to be regarded as coming under
the scope of suggestion.


\section{II, Vṛtti 4}

The poem in which the chief category is of the nature of
sentiment, emotion, their semblance or cessation and wherein all
figures, both of sound and sense, and qualities come in only as
handmaids of the chief category and remain as much distinct
from what is suggested as from one another, gets the designation
of Suggestive Poetry.


\section{II, Kārikā 5}

But if in a poem the chief purport of the sentence should
relate to something else, and if sentiment and so on should
come in only as auxiliaries to it, it is my opinion that sentiment
and so on are figures of speech in such a poem.


\section{II, Vṛtti 5}

Although others have explained the scope of Figurative
Sentiment (in quite a different way), still it is my view that only
such sentiments etc.\ as become auxiliaries to some other purport
of the sentence which happens to be much more important are to
be regarded as figures. 


\subsection{II, Vṛtti 5 continued --- 2.5a A in IMP (skipped in EC Manual)}

For instance, one can easily see how in
hymns of praise, sentiments etc., appear as auxiliaries though they
are generally regarded as instances of the figure of Affectionate
Praise.


\subsection{II, Vṛtti 5 continued --- 2.5b A in IMP (skipped in EC Manual)}

The Figurative Sentiment according to our view may be either
pure or mixed. The following is an illustration of the first kind:---

\begin{quotation}
\begin{em}
`Why this jest?

Thou shalt not certainly part again from me,

Having returned after so long,

O ruthless one! whence this flair for travel? ---

Thus in dreams do the wives of your enemy speak

Clasping fast the necks of their beloved lords;

But soon they awake

To find empty their embraces

And to lament loud.
\end{em}
\end{quotation}


In this example the pure sentiment of pathos is an auxiliary
(to the praise of the king) and hence it is clearly Figurative
Sentiment.


\subsection{II, Vṛtti 5 continued --- 2.5c A in IMP (skipped in EC Manual)}

The mixed variety of auxiliary sentiment is instanced in the
following:---

\begin{quotation}
\begin{em}
Let the fire of Siva's shaft burn down

our sins; a shaft that conducted itself

in the manner of a lover who has given

offence afresh to his beloved:---Though

shaken off by the wives of Tripura with

fearful eye-lilies, it would cling fast to

their hands; though forcibly pushed out,

it would ho Id on to the ends of t' eir skirts;

though violently thrust aside by the hair

(of its feather), it would fall at their

feet and yet remain unnoticed because of

their agitation; and though pushed back,

it would hug them verily.
\end{em}
\end{quotation}


Here, the main purport of the sentence is the extraordinary
glory of Siva. The sentiment of love-in-separation due to jealousy
is conveyed by \textit{double entendre} and this is made auxiliary to it
(i.e.\ praise of glory.) Only such instances are proper illustrations
of Figurative Sentiment. Hence it is that though the sentiments
of love in-separation due to jealousy and of pathos are mutually
opposed, since they have been both rendered auxiliary (to the
main purport, their inclusion in the same place does not become
a defect. But in instances where sentiment itself happens to be the
main purport) how can it ever be a figure? It is well known that a
figure is (nothing but) an enhancer of charm. It is indeed impossible that a thing can become an enhancer of its own charm.


\subsection{II, Vṛtti 5 continued --- 2.5d A in IMP (skipped in EC Manual)}

The following verse sums up the position:---

\begin{quotation}
\begin{em}
It is only the employment of figures, one and all, in
view of the main purport of sentiment, emotion, etc., that
really justifies their being regarded as sources of charm.
\end{em}
\end{quotation}

Therefore, none of those cases where sentiment etc.\ happen
to be the main purport, become instances of Figurative Sentiment.
On the other hand they will only form a species of suggestion.
Simile etc.\ are all enhancers of its charm alone. But in cases
where the main purport happens to be some other meaning and
when its beauty is enhanced by sentiment etc., we get proper
instances of Figurative Sentiment.

Thus understood, the distinct spheres of suggestion, figures
like simile and Figurative Sentiment become clearly demarcated.


\subsection{II, Vṛtti 5 continued --- 2.5e A in IMP (skipped in EC Manual)}

If one were to assert that the treatment of sentient subjects
alone serves to exemplify Figurative Sentiment, it would mean
that figures like simile would either be left with very little scope
or no scope at all. For even when the theme happens to be
the behaviour of an insentient object, the behaviour of a sentient
object also will in one way or another be superimposed upon
it. Again, even when such a superimposition is present, one
would have to say that it is not an instance of Figurative
Sentiment in case the insentient ones alone form the main theme
of description. And this would be tantamount to an assertion
that the vast bulk of literature which happens to be really the
golden treasury of sentiments is without any sentiment. 


\subsection{II, Vṛtti 5 continued --- 2.5f A in IMP (skipped in EC Manual)}

Here
is an example:

\begin{quotation}
\begin{em}
Frowning with its waves as with brows,

Girdled with the line of fluttering birds,

Throwing off its foam as a garment slipped in anger,

Hurrying in devious ways with far too tumbling steps

Surely, here is my jealous beloved,

Changed into the form of the stream.*
\end{em}
\end{quotation}

This is another example:

\begin{quotation}
\begin{em}
There standest thou creeper,

All slender, thy poor sad leaves are moist with rain

Thou silent, with no voice of honey-bees

Upon the drooping boughs; as from thy lord

The season separated, leaving off

Thy habit of bloom. Why I might think I saw

My passionate darling penitent

With tear-stained face and body unadorned

Thinking in silence how she spurned my love.*
\end{em}
\end{quotation}
* Translation Sri Aurobindo's.

Or? to take a still another example:

\begin{quotation}
\begin{em}
How do they do, those bower-huts, O friend,

On the bank of the river Jamunā?

Those companions of the sports of cowherdesses

And those witnesses of Rādhā's amours?

Now that none will pluck them soft

To turn them into beds of love,

I am afraid that all those fresh green leaves

Do lose their greenness and become old.
\end{em}
\end{quotation}


\subsection{II, Vṛtti 5 continued --- 2.5g A in IMP (skipped in EC Manual)}

In these examples, though insentient objects happen to be
themes of description, the attribution of sentient behaviour to
them is quite obvious. Perhaps it might be argued that one
may accept the presence of Figurative Sentiment in instances of
this type wherein one finds attribution of sentient behaviour.
At that rate, figures like simile will be left with no scope at all
or, at the most, with very little scope. For, there is no such
insentient theme at all in poetry in which the attribution of
sentient behaviour is wholly absent; it will be found at least in
the form of (a sentimental description of) the setting or situation.
Hence, only sentiments that are secondary in importance should be
regarded as figures. If one finds a sentiment or emotion with
paramount importance, it will serve only as the object for beautification by other figures etc.\ and is of the very essence of
suggestion. 


\subsection{II, Vṛtti 5 continued --- 2.5h A in IMP (skipped in EC Manual)}

Furthermore---


\section{(II, Kārikā 6 is skipped in EC Manual)}


\section{II, Kārikā 7}

The Erotic indeed, is the sweetest and the most delectable
of all sentiments. The quality of sweetness is grounded
securely on poetry which is full of this sentiment.


\section{II, Vṛtti 7}

The Erotic shines indeed as sweeter and more delectable
than every other sentiment. `Sweetness' is a quality which
relates to word and meaning of compositions (imbued with this
sentiment) and not to mere sound-harmony. For, sound-harmony is found alike in forcefulness too (and is not a differentia
of sweetness).


\section{II, Kārikā 8}

In sentiments viz., Love-in-separation and the Pathetic,
sweetness will be uppermost. It is so because the mind is
moved very much in such instances.


\section{II, Vṛtti 8}

The quality of sweetness alone is uppermost in the sentiments
of Love-in\-separation and the Pathetic as it causes great
delectation in the minds of refined critics.


\section{II, Kārikā 9}

Sentiments like the Furious are characterised by great
exciting power in poetry. The quality of forcefulness is that
which inheres in sound and sense which produce this effect.


\section{II, Vṛtti 9}

Indeed sentiments like the Furious produce excessive excitement (in the readers). Hence, by secondary usage, one might
refer to the sentiments themselves by the term `excitement'.
The sound which produces this effect is none other than a
sentence adorned by lengthy compound constructions. As for
instance:

\begin{quotation}
\begin{em}
cañcad-bhuja-bhramita-caṇḍa-gadābhighāta-

saṁcūrṇitoru-yugalasya suyodhanasya |

styānāvabaddha-ghana-śonita-śoṇa-pāṇi-

ruttaṁsayiṣyati kacāṁstava devi bhīmaḥ ||

(O Queen! this Bhīma shall himself bind

Your scattered curls with his hands

Reddened by the profuse and coagulated puddles

Of Suyodhana's blood as he lies low

With thighs pulverised by hard blows

From this terrible mace

Swung by these redoubtable arms of mine.)
\end{em}
\end{quotation}

The sense which produces this effect of forcefulness does not
stand in need of lengthy compound constructions; it may contain
simple constructions only. As for instance:---

\begin{quotation}
\begin{em}
yo yaḥ śastraṁ bibharti svabhujagurumadaḥ pāṇḍavīnāṁ camūnāṁ

yo yaḥ pāñcālagotre śiśuradhikavayā garbhaśayyāṁ gato vā

yo yastatkarmasākṣī carati mayi raṇe yaśca yaśca pratīpaḥ

krodhāndhastasya tasya svayamapi jagatāmantakasyāntako'haṁ

(Whosoever with over weening pride of his strong arms

Bears weapons in Pāṇḍavas' battalions,

Whosoever is sprung from Pāñcāla clan---

Whether child, grown-up or embryo,

Whosoever has been a silent witness of that ghastly deed

And whosoever will cross my way as I move in the battle-field,

I, in my blinding rage, shall prove a Destroyer

Of each one of them and of even Yama himself!)
\end{em}
\end{quotation}

Thus sound and sense can both become imbued-- with
forcefulness.

\section{II, Kārikā 10}

That quality in poetry by which poetry throws itself open
to the entry of all sentiments may he taken as perspicuity.
Its applicability is universal.


\section{II, Vṛtti 10}

Perspicuity is just the lucidity in sound as well as in sense.
It is a quality common to all sentiments and all kinds of composition. Hence this quality should be understood as primarily
relating to the suggested sense only.




\chapter{THE THIRD FLASH}

(The third chapter is skipped entirely in the EC Manual)



\chapter{THE FOURTH FLASH}


\section{(IIII, Kārikā 1--4 are skipped in EC Manual)}

\section{IIII, Kārikā 5}

Though several varieties of the suggested-suggester relationship are possible, the poet should be most intent upon one of
them in particular, viz., that relating to the delineation of
sentiments etc.

\section{IIII, Vṛtti 5}

Though words involving the relation of suggested-suggester
are possible in various ways, the poet desirous of securing novel
poetic themes should be most intent upon one of them only,
viz., suggestion of sentiments, etc. So long as the poet exercises
undeflected concentration regarding the suggested contents, viz.,
sentiment, emotion, its semblance, and the suggesters previously
explained, viz., letter, word, sentence, and texture, and the work
as a whole, the poet's entire work will become strikingly novel.
That is why in epics like the \textit{Rāmāyaṇa} and the \textit{Mahābhārata},
the subjects of `battle' etc., appear quite new though they are
described again and again.

In a work as a whole the delineation of a single sentiment
as the predominant one will endow not only novelty of content
but also abundance of charm. If one should ask for examples,
we would say in reply that the \textit{Rāmāyaṇa} is itself one example
and that the \textit{Mahābhārata} is another. Into the \textit{Rāmāyaṇa} indeed,
the First Poet himself has incorporated the sentiment of pathos
as is clear by his own declaration---``\textit{Sorrow has taken the turn
of a stanza}''. It has in fact been kept up as predominant till
the very end of the work in view of his concluding the work
at the point of the eternal loss of Sīta by Rāma.

In the \textit{Mahābhārata} too, which combines both the elements
of instruction and poetry in one, it will be seen that its conclusion
in a note of despair consequent on the miserable deaths of
Vṛṣṇis as well as Pāṇḍavas, as constructed by the great sage,
reveals his primary intention of preaching the moral of renunciation through his work and throws light upon the fact that he
intended final emancipation as the foremost of human values
and Peace as the most predominant sentiment in the whole
work. This has been partially brought out even by the other
commentators on the \textit{Mahābhārata}. Even the reverend sage
himself whose foremost desire was the rescue of his fellowmen
from the deep abyss of ignorance in which they were weltering
by vouchsafing to them the light of supreme knowledge, has
declared in no uncertain terms:---

\begin{quotation}
\begin{em}
`Just (as much) as wordly pursuits

Turn out to be unavailing,

One's sense of aversion to them will become firm;

There is no doubt at all'.
\end{em}
\end{quotation}

and so on in the same strain more than once. It stands out
most clearly that the main purport of the \textit{Mahābhārata} is the
communication of the fact that Peace is to be regarded as
the most prominent sentiment, the others being secondary to it
and that final emancipation is the most prominent of human
values, the other values beng only subsidiary to it. The principal-subordinate relationship of sentiments has been set forth already.

Just as the body might be invested with prominence though
it is only secondary when the really prominent soul is not taken
into consideration, so also a secondary sentiment as well as a
secondary value might be justly regarded as prominent in itself
and beautiful.
% NOTE: "consideration. so also" is as in the book

It might be urged by some that all the points sought to be
conveyed in the \textit{Mahābhārata} have been enumerated exhaustively
in the Introduction itself while the above point is conspicuous
by its absence in the Introduction. They might add that, on the
other hand, Vyāsa expressly claims in the Introduction that his
work throws light on all the human values and that it contains
all the sentiments.

Here is our answer to the objection: though it is true that
nowhere in the Introduction we come across an express statement
to the effect that in the \textit{Mahābhārata}, Peace is intended to be
the most prominent of all sentiments and that final emancipation
is intended to be the most prominent of all human values, it is
also true that this has been conveyed in a suggestive way through
the sentence—

\begin{quotation}
\begin{em}
Herein, forsooth, will be glorified

Lord Vāsudeva too, the Eternal.
\end{em}
\end{quotation}

The idea implied in this sentence is that all the other
subjects described in the \textit{Mahābhārata} such as the exploits of the
Pāṇḍavas end only in tragedy and belong only to the realm of
ignorance while the only eternal and truly abiding subject
glorified here is Lord Vāsudeva. Therefore (suggests Vyāsa),
`be devoted in heart only to that supreme Lord; don't remain
attached to empty pleasures and don't be too intent upon
excellences even like statesmanship, modesty, and valour just for
their own sakes'! It is to suggest the utter futility of worldly
existence indeed that \textit{ca} (`too') figures last in that sentence
The verses that immediately follow, viz., `He alone is Real', are
also imbued with this very significance.

By appending \textit{Harivaṁśa} at the end of the \textit{Mahābhārata},
the great Poet-Creator Kṛṣṇa Dvaipāyana has made this inner
and beautiful significance abundantly clear. By propagating the
cause of whole-hearted devotion in that Absolute Reality beyond
the realm of worldly existence, he appears definitely to have
regarded the entire activities of worldly existence to be of the
nature of a prima facie case (deserving refutation). He indulges
in lengthy descriptions of the greatness of deities, sacred spots,
asceticism, etc., only because they serve, in his opinion, as the
channels of realising that Supreme Reality; other particular
gods also are glorified only as so many manifestations of His
Supreme glory. The description of the exploits of Pāṇḍavas
etc.\ is also meant to produce a sense of renunciation; renunciation, in its turn, is the very basic instrument of final emancipation; and final emancipation itself has been shown in the
\textit{Bhagavadgītā} and other works to be a sure means towards
the attainment of the Supreme Reality. Thus, indirectly, even the
description of the exploits of Pāṇḍavas etc., might be regarded
as a means towards the attainment of the Supreme Reality.
Instead of referring to the Supreme Reality by the very word
Supreme Reality, Vyāsa uses a synonym, viz. Vāsudeva. By this
word Vāsudeva we should consider as intended the meaning of
Supreme Reality only, which is the abode of boundless power,
because in several contexts like the Gītā, this word has been
widely used to convey the meaning of Supreme Reality itself.
It should not be understood to mean only the human form born
as son of Vāsudeva in Mathurā but wholly the Supreme Reality
itself imitating in every way the nature of such a human being
born in Mathurā; because, the word under consideration, viz.,
Vāsudeva, is qualified by the adjective, viz., Eternal. And in
other works like the Rāmāyaṇa, we find this word used as a
proper name for other incarnations also of the Supreme Lord.
This fact has indeed been established even by grammarians
themselves.

Hence we are quite justified in saying that the purport
implied by the sage for the sentence in question of the Introductory
chapter is the perishable nature of everything with the single
exception of the Supreme Lord and that the \textit{Mahābhārata} as
a whole is intended by him to convey the highest human value,
viz., final emancipation, when the work is regarded as a scripture,
and to delineate the sentiment of Quietude---whose nature is of
heightened tranquillity and happiness at the cessation of desire---as
the predominant sentiment in the work when it is regarded as
a poem. As this purport happens to be the most essential one,
it has not been stated expressly but conveyed by way of suggestion.
An intrinsically essential idea acquires beauty only when it is
revealed in a way other than the expressed. In polished literary
circles it has indeed become a convention of wits to communicate
their best ideas only through suggestion and not at all by express
words.

Therefore, the conclusion is irresistible that novelty in
poetic theme as well as great beauty of construction is achieved
by adopting a single sentiment as predominant in any poem as a
whole. That is why compositions containing only themes in
keeping with sentiment are often seen to hold out abundant
charm although they may be lacking in different figures of
speech. For instance---

\begin{quotation}
\begin{em}
`Victory to that sage, the foremost of ascetics,

The mighty soul who was pitcher-horn;

For he could behold both the divine Fish and Tortoise

In the hollow of his single hand'.
\end{em}
\end{quotation}

and such others might be cited. The idea that, in the hollow
of a single hand, one could see both the divine Fish and Tortoise
heightens the sentiment of Wonderment. Further, though the idea
of the whole ocean being comprised'in the hollow of the sage's
hand is traditional, the idea that both the divine forms of the
Lord, viz., the Fish and the Tortoise were seen by him there
at the same time, is most original and it heightens the sentiment
mentioned, all the more. An idea which is already familiar
due to wide currency among people will not cause surprise,
though it might embody an element of wonder.

An original poetic theme conforms not only to the sentiment
of wonder but also to the other sentiments as In the following
example:---

\begin{quotation}
\begin{em}
That part of her body

Which touched the end of your balance

As you passed her along the street

Still sweats, horripilates and trembles;

O, thou handsome youth!
\end{em}
\end{quotation}

The sentiment apprehended by contemplating upon the
verse as it is, will never be had by any express statement to the
effect, `By your touch, she sweats, horripilates and trembles'.

Thus we have demonstrated how poetic themes are rendered
new by the contact of one or another major variety of \textit{Dhvani}
or principal suggestion. In the same way, the element of subordinated suggestion too which is threefold from the point of
view of the suggested content will ensure novelty to poetic themes
if adopted in the work. But we have refrained from illustrating
it here in detail for fear of making the work unduly prolix.
Refined critics should fancy it for themselves.

\end{document}
