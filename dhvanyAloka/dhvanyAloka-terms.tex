\documentclass[12pt]{article}

\usepackage[top=1in, bottom=1.25in, left=1.25in, right=1.25in]{geometry}

\usepackage{indentfirst}
\usepackage{multicol}
\usepackage{enumitem}



% Don't print page numbers - assuming some overall page number is added for EC Manual printing
\pagenumbering{gobble}

% if set to zero, do not print numbers before section headers
\setcounter{secnumdepth}{0}

\title{Some Notes on the Dhvanyāloka}
\author{Bruce M. Perry \and Updated by Justin K. Lowe}
\date{October 10, 2002, Updated: \today}




\begin{document}

\maketitle

\section{Introduction}

One of the principal achievements of Indian aesthetics (\textit{alaṅkāra}) is its analysis of the nature of the poetic experience.
According to its two foremost theorists, poetry consists in the power of suggestion (\textit{dhvani}), and, conversely, produces in the audience an experience of relishing (\textit{rasa}) that is not emotional but transcendental (\textit{alaukika}).
The crucial texts of these theorists, however, do not lend themselves to a straightforward study.
Anandavardhana wrote verses (\textit{kārikā}), upon which he composed a commentary (\textit{vṛtti}), the \textit{Dhvanyāloka}.
Abhinavagupta subsequently composed a commentary on both the verses and commentary of Ānandavardhana, called the \textit{Locana}.
Given the length, complexity, and seeming randomness of topics taken up in these works, a simplifying approach seems in order.
To provide a roadmap, as it were, of the rasa-theory, a chapter from A. B. Keith's \textit{The Sanskrit Drama}, ``The Sentiments'', is first provided.
Select passages from Ānandavardhana and Abhinavagupta, from the excellent translation of Ingalls, Masson, and Patwardhan, are then presented; these selections contain the most direct, if not exhaustive, formulations or discussions of \textit{dhvani} and \textit{rasa}.

Please note that the scholarly footnotes to the selections are supplied for completeness' sake only: they are not crucial to understanding the texts.
The complete translation will be on reserve in the library.


Bruce M. Perry

10/23/02

% insert page break
\newpage


\section{Terms}

Some helpful terms from Ingalls, Masson, and Patwardhan's translation. Additional explanation is attached, from pages 15--17 of their Introduction to this text.

% \vspace{-6pt}
\setlength{\columnsep}{-3em}
\begin{multicols}{2}
	\begin{description}
		\setlength{\itemsep}{-0.25em}

		\item[anubhāva] consequent

		\item[bhāva] emotion

		\item[vibhāva] determinant

		      \vspace{-6pt}
		      \begin{description}[
				      itemindent=-3em
			      ]
			      \setlength{\itemsep}{-0.25em}
			      \item[ālambanavibhāva] objective \\ determinant

			      \item[uddīpanavibhāva] stimulative \\ determinant
		      \end{description}

		      \vspace{-4pt}
		\item[vyabhicārin/vyabhicāribhāva] temp\-orary/transient state of mind

		\item[sthāyibhāva] abiding emotion

		\item[rasa] flavor/sentiment -- based on the sthāyibhāvas respectively, as shown in following table.

	\end{description}
\end{multicols}



%TODO center table better
\makebox[\textwidth][c]{
	% \noindent{
	\begin{tabular}{|c||c|c||c|c|}
		\hline
		  & \textit{Sanskrit}    & \textit{English Term}    & \textit{Sanskrit} & \textit{English Term}      \\
		\hline
		  & \textbf{sthāyibhāva} & \textbf{abiding emotion} & \textbf{rasa}     & \textbf{flavor, sentiment} \\
		\hline
		1 & rati                 & sexual desire            & śṛṅgāra           & erotic                     \\
		\hline
		2 & hāsa                 & laughter                 & hāsya             & comic                      \\
		\hline
		3 & śoka                 & grief                    & karuṇa            & tragic                     \\
		\hline
		4 & krodha               & anger                    & raudra            & furious,                   \\
		  &                      &                          &                   & cruel                      \\
		\hline
		5 & utsāha               & heroic energy            & vīra              & heroic                     \\
		\hline
		6 & bhaya                & fear                     & bhayānaka         & fearsome,                  \\
		  &                      &                          &                   & timorous                   \\
		\hline
		7 & jugupsā              & disgust                  & bībhatsā          & gruesome,                  \\
		  &                      &                          &                   & loathsome                  \\
		\hline
		8 & vismaya              & wonder, amazement        & adbhuta           & wondrous                   \\
		\hline
		9 &                      &                          & śānta             & peace (\textit{this term}  \\
		  &                      &                          &                   & \textit{ added later})     \\
		\hline
	\end{tabular}
	% TODO: intro of Dhvanyāloka says "To these Ānanda adds a ninth, the rasa of peace (śānta)." p16, EC p74
	% TODO: or do something like: NB: the XYZ and peace were added later by SOEMONE in SOME TEXT. but yeah have to find that first...
	% TODO: check sthāyibhāva sanskrit term tṛṣṇākṣayasukha and translation for peace - EC page 159, text p695, 4.5 A.16 footnote. have 'cessation of desire' from EC p156
	% Masson, J L; Patwardhan, M V (1969), SantaRasa and Abhinavagupta's Philosophy of Aesthetics, Bhandarkar Oriental research Institute, OCLC 844547798 p92 says: its stable emotion (sthāyibhāva) impassivity (sama) which culminates in detachment (Vairāgya) arising from knowledge of truth and purity of mind
	% S K. De, Sushil Kumar (1960). History of Sanskrit Poetics- - Vol2. Firma K. L. Mukhopadhyay. pp30, 343 says tranquility (Sama) or Disenchantment (Nirveda) 
	% }
}


\subsection{Abbreviations Used}

\begin{description}
	\setlength{\itemsep}{-0.25em}

	\item[K] \textit{\textbf{K}ārikā} (verse) of the \textit{Dhvanyāloka} (`Light on Suggestion'),
	      by Rājānaka \\ Ānandavardhana, Kashmiri [9th AD]% and\dots

	\item[A] \textbf{Ā}nandavardhana's \textit{vṛtti} (commentary) on the \textit{Kārikā} of their \textit{Dhvanyāloka}

	\item[L] \textit{\textbf{L}ocana} (`The Eye'), commentary by Abhinavagupta, Kashmiri [10th AD]
\end{description}
% TODO: maybe reword this so it makes more obvious sense that guy wrote K and A at same time? Or was there a time gap??


\subsection{Some Authors Mentioned}

\begin{description}
	\setlength{\itemsep}{-0.25em}

	\item[Bhāmaha] author of works on literary criticism, largely concerned with the figures of speech [8th AD]

	\item[Bharata] (legendary) author of the Nāṭyaśāstra (\textit{BhNŚ}), an ancient manual on dramaturgy, source of first eight sthāyibhāva and rasa noted above.
	      % TODO: does 'legendary' mean 'reputed' or some such? Am I the only one who was confused by this?

	\item[Bhattanayaka] critic of Ānandavardhana, apparently a Mīmāṃsaka, answered by Abhinavagupta
\end{description}


\end{document}


% Delete this section later, used for now for an easy palce to copy letters from for updates to works
%
% ś
% 
% Ś
% 
% ṛ
% 
% ṅ
% 
% ṇ
% 
% ā
% 
% Ā
% 
% ī
% 
% ṭ
% 
% ṃ
