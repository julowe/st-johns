\documentclass[10pt]{article}
% \documentclass[12pt]{article}

\usepackage[top=0.75in, bottom=0.75in, left=0.75in, right=0.75in]{geometry}

\usepackage{indentfirst}
\usepackage{multicol}
\usepackage{enumitem}
% \usepackage{hyperref}

%\DeclareUnicodeCharacter{1E41}{\(\dot{m}\)} %works in pdfLaTeX, but not LuaLaTeX
% TODO: decide if i want to stick with text's conventions, or update to IAST. already updated n with macron to n with tilde on top... but those at least look close - vs m with dot below and m with dot above...




% Don't print page numbers - assuming some overall page number is added for EC Manual printing
\pagenumbering{gobble}

% if set to zero, do not print numbers before section headers
\setcounter{secnumdepth}{0}

\title{Some Notes on the Dhvanyāloka}
\author{Bruce M. Perry \and Updated by Justin K. Lowe\footnote{Available at: https://github.com/julowe/st-johns/blob/main/dhvanyAloka/dhvanyAloka-terms-additional.tex} }
\date{October 10, 2002, Updated: \today}




\begin{document}

% TODO uncomment following part if this is to be used more widely. or no matter what, since it did start from their work?

\maketitle

\vspace{-8mm}
\section{Introduction}

 % {\fontsize{12pt}{14pt}\selectfont
 {\fontsize{10pt}{14pt}\selectfont
  One of the principal achievements of Indian aesthetics (\textit{alaṅkāra}) is its analysis of the nature of the poetic experience.
  According to its two foremost theorists, poetry consists in the power of suggestion (\textit{dhvani}), and, conversely, produces in the audience an experience of relishing (\textit{rasa}) that is not emotional but transcendental (\textit{alaukika}).
  The crucial texts of these theorists, however, do not lend themselves to a straightforward study.
  Ānandavardhana wrote verses (\textit{kārikā}), upon which he composed a commentary (\textit{vṛtti}), the \textit{Dhvanyāloka}.
  Abhinavagupta subsequently composed a commentary on both the verses and commentary of Ānandavardhana, called the \textit{Locana}.
  Given the length, complexity, and seeming randomness of topics taken up in these works, a simplifying approach seems in order.
  To provide a roadmap, as it were, of the rasa-theory, a chapter from A. B. Keith's \textit{The Sanskrit Drama}, ``The Sentiments'', is first provided.
  Select passages from Ānandavardhana and Abhinavagupta, from the excellent translation of Ingalls, Masson, and Patwardhan, are then presented; these selections contain the most direct, if not exhaustive, formulations or discussions of \textit{dhvani} and \textit{rasa}.

  Please note that the scholarly footnotes to the selections are supplied for completeness' sake only: they are not crucial to understanding the texts.
  The complete translation will be on reserve in the library.


  Bruce M. Perry

  10/23/02
 }

% % insert page break
% \newpage


% Some helpful terms from Ingalls, Masson, and Patwardhan's translation. Additional explanation is attached, from pages 15--17 of their Introduction to this text.


\subsection{Abbreviations Used}

\begin{description}
	\setlength{\itemsep}{-0.25em}

	\item[K] \textit{\textbf{K}ārikā} (verse) of the \textit{Dhvanyāloka} (`Light on Suggestion'),
	      by Rājānaka \\ Ānandavardhana, Kashmiri [9th AD] (also referred to as `Ānanda')% and\dots

	\item[A] \textbf{Ā}nandavardhana's \textit{vṛtti} (commentary) on the \textit{Kārikā} of their \textit{Dhvanyāloka}. NB: The \textit{vṛtti} (commentary) following each \textit{Kārikā} was not originally broken into sections. The a, b, c, etc.\ lettering was done by the translators Ingalls, Masson, and Patwardhan (IMP).

	\item[L] \textit{\textbf{L}ocana} (`The Eye'), commentary by Abhinavagupta, Kashmiri [10th AD]  (also referred to as `Abhinava')
\end{description}
% TODO: maybe reword this so it makes more obvious sense that guy wrote K and A at same time? Or was there a time gap??


\subsection{Some Authors Mentioned}

% TODO: add more authors??

\begin{description}
	\setlength{\itemsep}{-0.25em}

	\item[Bhāmaha] author of works on literary criticism, largely concerned with the figures of speech [8th AD]

	\item[Bharata] (legendary) author of the Nāṭyaśāstra (\textit{BhNŚ}), an ancient manual on dramaturgy, source of first eight sthāyibhāva and rasa noted above.
	      % TODO: does 'legendary' mean 'reputed' or some such? Am I the only one who was confused by this?

	\item[Bhaṭṭanāyaka] critic of Ānandavardhana, apparently a Mīmāṃsaka, answered by Abhinavagupta
	      % Mīmāṁsaka is now IAST, but not in this text - it uses m with dot below
	      % TODO: check diacritics
\end{description}

\subsection{Sections of text included in EC Reading}

% NOTE: one line break is not oucnted as a new paragraph, all below is typeset/rendered as one continuous line
Translator's Introduction (15--17);
\textbf{1.1} K, A, L (47--49 partial);
\textit{--skipping--};
\textbf{1.1} L (partial 51--52 partial);
\textbf{1.1a} A \& L (54--58);
\textit{--skipping--};
\textbf{1.1e} A \& L (partial 67--73);
\textit{--skipping--};
\textbf{1.4} K, A, L, \textbf{1.4a} A \& L, \textbf{1.4b} A (78--84);
\textbf{1.4b} L partial text (84);
\textbf{1.4b} L partial text \& footnotes (92);
\textbf{1.4c} A \& L (98--99);
\textit{--skipping--};
\textbf{1.4g} A \& L, \textbf{1.5} K, A, L (105--119);
\textit{--skipping--};
\textbf{1.18} K, A, L (188--196);
\textit{--skipping--};
% \textbf{2.3} K, A, L, \textbf{2.4} K, A, L, \textbf{2.5} K, A, L (214--233); 
\textbf{2.3--2.5} K, A, L \{for all in this range\} (214--233);
\textit{--skipping--};
\textbf{2.7--2.10} K, A, L \{for all in this range\} (251--260);
\textit{--skipping--};
\textbf{4.5} K, A (690--696);
\textbf{4.5} L partial text (696)



% insert page break
\newpage


\section{Expanded List of Terms}

Translation of Sanskrit terms in ``quotes'' are from IMP translation, unless otherwise noted.
I have attempted to provide the reference from where each term is first encountered.
Notation follows section numbering in IMP translation. Page numbers are given when translation of K. Kris.\ is used.
% TODO: spell out K. Kris.
Alternate translations are separated by a comma (whether from a subsequent section of IMP text, or from K. Kris.).
Significantly differing definitions or usages are separated by a semicolon.
Footnotes are marked by a dot, e.g. ``1.1 L.1'' is footnote 1 for the Locana of 1.1.
``(M.W. p.\#)'' refers to a page from Monier-Williams Sanskrit---English Dictionary, 1899 printing.


% \newcommand\litem[1]{\item{\bfseries #1}}

\newcommand\litem[1]{\item{\bfseries #1\label{#1}}}


% TODO: ?? i forget...
\setlength{\columnsep}{3em}
\begin{multicols}{2}
	% \makebox[\columnwidth][c]{%
	\hspace*{-9mm}
	\begin{tabular}{|c||c|c||c|c|}
		% \begin{tabular}{lrlrl}
		%\hline
		%& \textit{Sanskrit}    & \textit{English Term}    & \textit{Sanskrit} & \textit{English Term}      \\
		\hline
		  & \textbf{sthāyi-} & \textbf{abiding} & \textbf{rasa} & \textbf{flavor,}   \\
		  & \textbf{bhāva}   & \textbf{emotion} &               & \textbf{sentiment} \\
		\hline
		1 & rati             & sexual           & śṛṅgāra       & erotic             \\
		  &                  & desire,          &               &                    \\
		  &                  & love             &               &                    \\
		\hline
		2 & hāsa             & laughter         & hāsya         & comic              \\
		\hline
		3 & śoka             & grief            & karuṇa        & tragic,            \\
		  &                  &                  &               & compassion         \\
		\hline
		4 & krodha           & anger            & raudra        & furious,           \\
		  &                  &                  &               & cruel              \\
		\hline
		5 & utsāha           & heroic           & vīra          & heroic             \\
		  &                  & energy           &               &                    \\
		\hline
		6 & bhaya            & fear             & bhayānaka     & fearsome,          \\
		  &                  &                  &               & timorous           \\
		\hline
		7 & jugupsā          & disgust          & bībhatsā      & gruesome,          \\
		  &                  &                  &               & loathsome          \\
		\hline
		8 & vismaya          & wonder,          & adbhuta       & wondrous           \\
		  &                  & amaze-           &               &                    \\
		  &                  & ment             &               &                    \\
		\hline
		9 &                  &                  & śānta         & peace              \\
		\hline
	\end{tabular}

	\vspace{1mm}
	\textbf{NB}: The introduction to the IMP translation of the Dhvanyāloka provides that the first 8 sthāyibhāva and rasa are given by BhNŚ 6.17 and ``To these Ānanda adds a ninth, the rasa of peace (\textit{śānta}).'' (p.16)
	% TODO: add something like...? These \textit{rasa}s can have more specific subtypes, e.g. \textit{vipralambhaśṛṅgāra} ``the \textit{rasa} of love in separation''?? this isn't right translation... (2.8 K?)
	% The \textit{rasa} of \textit{śānta} (peace) was added later.
	% TODO: or do something like: NB: the XYZ and peace were added later by SOEMONE in SOME TEXT. but yeah have to find that first...
	% TODO: check sthāyibhāva sanskrit term tṛṣṇākṣayasukha and translation for peace - EC page 159, text p695, 4.5 A.16 footnote. have 'cessation of desire' from EC p156
	% TODO: or 'liberation/moks.a' (4.5 A, EC p.155? 2nd line. also 3rd P, line 3)
	% Masson, J L; Patwardhan, M V (1969), SantaRasa and Abhinavagupta's Philosophy of Aesthetics, Bhandarkar Oriental research Institute, OCLC 844547798 p92 says: its stable emotion (sthāyibhāva) impassivity (sama) which culminates in detachment (Vairāgya) arising from knowledge of truth and purity of mind
	% S K. De, Sushil Kumar (1960). History of Sanskrit Poetics- - Vol2. Firma K. L. Mukhopadhyay. pp30, 343 says tranquility (Sama) or Disenchantment (Nirveda) 
	% from 2.9 L: comic, fearsome, loathsome, peace
	% compassion from 1.5 A
	% rati - love from 2.7 L
	% tṛṣṇākṣayasukha - from footnote 4.5 A.16 - same phrase is used at 3.26a A Abhinava makes it clear that --- is the sthāyibhāva of śāntarasa - 
	% TODO: look up tṛṣṇākṣayasukha in 3.26a A in K. Kris.
	% }


	% TODO: check comma vs semicolon between alt translations vs alt usage/def
	\begin{enumerate}[
			leftmargin=0em,
			% rightmargin=-1.5em,
			rightmargin=0em,
		]
		% \setlength{\itemsep}{0.25em}
		\setlength{\itemsep}{0.15em}

		% TODO: move to new column? rearrange so it isn't split over 2.
		\litem{dhvani} ``suggestion'' (1.1 K, 1.1 A) Also ``suggested meaning'' (1.4g A). The three types of \textit{dhvani} per 1.4a A are:
		\begin{enumerate}

			\litem{vastudhvani} ``the suggestion of a fact'' (first seen 1.4a A, defined 1.1e L) of the ``sequential variety'' of \textit{dhvani} (2.3 L.11).
			From:
			% TODO: check definition
			\begin{itemize}
				\litem{vastumātra} ``a simple thing'' (1.4a A)
			\end{itemize}

			\litem{alaṅkāradhvani} ``the suggestion of a figure of speech'' (1.1e L);
			``second variety of suggested meaning'' (1.4g A).
			From:
			% TODO: maybe wrong? see 1.1 K.2, EC p.77. Connected with bhākta?
			\begin{itemize}
				\litem{alaṅkāra} ``figures of speech'' (1.1 L, 1.1a A, 1.4g L), ``ornaments of sound'' (2.4 A)
			\end{itemize}

			\litem{rasadhvani} ``suggested \textit{rasa}'' (1.5 L),
			% TODO: get diff or better def of rasadhvani - don't love this one either: ``suggested rasa'' (1.5 L)
			it is the ``soul of poetry'' (1.5 K).
			From: ``not only \textit{rasa},
			but \textit{bhāva}, % page 125 or 127?
			\textit{rasābhāsa}, % page 127?
			\textit{bhāvābhāsa}, % page ?
			\textit{bhāvodaya}, % page ?
			\textit{bhāvasandhi}, % page ?
			\textit{bhāvaśabala}, % page ?
			\textit{bhāvapraśama}. % page 126?
			For definitions see 1.4g and for examples 2.3 L''
			and 2.3 K. % this note prob form footnote 1.18 L.7?
			%%% TODO: decide - alternate definition, straight from text tho...
			% ``which consists of \textit{rasa}, 
			% \textit{bhāva}, 
			% \textit{rasābhāsa}, 
			% \textit{bhāvābhāsa}, 
			% or \textit{bhāvapraśānti}'' (2.4 A)
			%%% TODO: decide - alternate definition, straight from text tho...
			% ``A \textit{rasa}, 
			% \textit{bhāva}, 
			% \textit{rasābhāsa}, 
			% \textit{bhāvābhāsa}, 
			% \textit{bhāvapraśānti}, etc.'' (2.3 K)
			This is also referred to by:
			% TODO: what is this 2.3 K note? it's mine, but why?
			\begin{itemize}
				\litem{rasādi} rendered as ``rasa, etc.'' in several places in the IMP translation.
				Explained as: ``The term refers to all elements that belong to \textit{rasadhvani}'' (1.4a A, defined 2.5 L) NB: \textit{ādi}: ``etc.'' (2.3 L)
				% EC p.143
				% TODO: get 1st ref, check ref of definition
				%\item \textit{rasadhvani} is the ``soul of poetry'' (1.5 K) % TODO: delete? moved this to 1st def block.
			\end{itemize}



			% \litem{three types of suggested meanings} ``vastumātra (simple thing), figure of speech, a rasa etc.'' (1.4a A) 
		\end{enumerate}
		% See \#~\ref{rasadhvani},
		% \ref{vastudhvani}, 
		% and \ref{alaṅkāradhvani} (per 1.1e L)
		% % TODO: keep  3 types of dhvani from 1.1e L? check these are correct.

		% TODO: add references/pages for below few terms. in text, or just Intro?
		\litem{rasa} ``flavor'', ``sentiment'' (TODO REF) --- based on the sthāyibhāvas respectively, as shown in the table above.
		% TODO: 'following' table, or above?
		% TODO: rasa is referenced in (2.3 K), but not defined
		% see 1.4g L - EC p.104 for good words on RASA

		% ``A rasa, bhāva, rasābhāsa, bhāvabhāsa, bhāvapraśānti, etc., \dots'' (2.3 K)
		% above, we get bhāvapraśānti. that is not in list for rasadhvani term
		% and then, ``For these technical terms see Introduction, pp. 17--20, 37, and Abhinava's remarks on this section and on 2.4 and its various subdivisions. For bhāvadhvani Abhinava invents a whole new meaning.''


		\litem{dhvanana} ``hinting'' (2.4 L)
		% TODO: move to bottom or elsewhere??

		\litem{vyañjana} ``suggesting'' (2.4 L)


		\litem{bhāva} ``emotion'' (1.4g L, 2.3 K) [K. Kris. Also B. Perry's translation, from IMP's Introduction]; ``realization'' (1.1e L, quote from \textit{BhNŚ})
		% TODO: check 2.3 K reference..., none earlier??
		% TODO: is emotion from the IMP text itself, or into?

		\litem{sthāyibhāva} ``abiding emotion'' (TODO REF), ``basic emotion'' (1.5 L, 1.18 L)
		% TODO: add reference - used in 1.4g L? but no def?? not sure what my notebook note means...

		\litem{anubhāva} ``consequent'' (TODO REF)
		% TODO: add reference
		% TODO: rearrange order?

		\litem{vibhāva} ``determinant'' (TODO REF); ``object'' (1.1e L, quote from \textit{BhNŚ})
		% TODO: add reference - used in 1.4g L? but no def?? not sure what my notebook note means...
		\begin{enumerate}
			%\setlength{\itemsep}{-0.25em}
			\litem{ālambanavibhāva} ``objective determinant''
			% TODO: add reference

			\litem{uddīpanavibhāva} ``stimulative determinant''
			% TODO: add reference
		\end{enumerate}


		\litem{vyabhicāribhāva} ``transitory state'' (1.5 L), % TODO: state... of mind??? check REF 
		``transient state of mind'' (2.3 L), %maybe also (1.4g L)??
		``transient emotion'' (1.18 L, 2.3 L)

		\litem{vyabhicārin} See vyabhicāribhāva, \#~\ref{vyabhicāribhāva}
		% temporary/transient state of mind
		% NOTE: not quoted, because form B. Perry's sheet, not sure if directly form text...
		% TODO: add reference
		% TODO: rearrange order?
		% TODO: which of the two vyabhic... is more used? make that the primary def/label




		% TODO: see footnote 1.1 K.2 - bhākta is used closely with guṇavṛtti
		\litem{bhākta} ``associated meaning'' (1.1 K), ``associated sense'' (1.1 L), ``regularly fed by another, a dependent'' (M.W. p.751)
		% MW: (H1) [Printed book page 751,2]
		% 1. bhākta mf(ī)n. (fr. bhakta) regularly fed by another, a dependent, retainer, Pāṇ. iv, 4, 68 [ID=149534]
		% fit for food, ib., iv, 4, 100. [ID=149535]
		% (H2) [Printed book page 751,2]
		% 2. bhākta mf(ī)n., (fr. bhakti) inferior, secondary (opp. to mukhya), Śaṃk. ; ĀpŚr. Sch. [ID=149536]
		% bhākta m. pl. ‘the faithful ones’, N. of a Vaiṣṇava and Śaiva sect, W. [ID=149537]
		% oh, so 'fed by' another, or 'dependent' on something else (e.g. here in the 2nd alternative view it is saying dhvani is just created by some other functions, so just say it is those functions instead of making up sub-divisions of existent process/term)
		\begin{itemize}
			\litem{bhākti} ``associated meaning'' (1.18 A), ``associated usage'' (1.18 L)
		\end{itemize}
		% TODO: keep as separate or combine/explain?



		\litem{------1} ``outside the scope of speech''? (1.1 K) TODO: a term, or just often used phrase?
		% TODO: get term or definition. delete this? it is one of the alternate views more than a term really. treated in ... 1.1e A/L??



		\litem{guṇa} the ``qualities'' (1.1 L, 1.1a A, 2.7 K \& A), ``virtues'' (EC p. 145?)
		\begin{enumerate}
			%\setlength{\itemsep}{-0.25em}
			\litem{mādhurya} ``sweetness'' (2.7 K.1),
			``Sweetness has its seat in poetry that is full of this flavor [\textit{śṛṅgāra}].''

			\litem{ojas} ``force'' (2.7 K.1),
			``strength'' (2.9 K)

			\litem{prasāda} ``clarity'' (2.7 K.1),
			``perspicuity'' [K. Kris.\ 2.10 K?, p.???],
			NB: ``[clarity] is a quality common to all \textit{rasa}s''
			% TODO: get k kris ref
			% TODO: use common to all line??

			\item ``\dots these are the \textit{śabdaguṇa}s mentioned by Bhāmaha, Daṇḍin, and Vāmana. Ānanda\-var\-dhana completely altered the older teaching by bringing them under the system of \textit{rasa}s. For him the \textit{guṇa}s are the properties of the \textit{rasa}s; see 2.7 below. Instead of the ten \textit{guṇa}s mentioned by older writers, Ānanda accepts only the three mentioned here.'' (2.4 L.35)
		\end{enumerate}


		\litem{vṛtti} \textbf{1.}\ ``commentary'' used by Abhinavagupta to refer to Ānandavardhana commentary \{vs. Ānanda\-var\-dhana's Kārikā\};
		\textbf{2.}\ ``simple alliteration'' ---
		``The word vṛtti \dots bears two different technical meanings in this book, one derived from Udbhaṭa the other from \textit{BhNŚ}. The word is here used in Udbhaṭa's sense, who applies this term to the three varieties of simple alliteration, that is, what later writers call \textit{vṛttyanuprāsa}. He calls the three types \textit{paruṣā} (harsh), \textit{upanāgarikā} (polite), and \textit{grāmyā} (rustic or vulgar). He calls the third type also \textit{komalā} (soft).'' (1.1a A.4)
		% TODO: still use?: multiple meanings (1.1a A)
		% NOTE: some more words on this in (1.1a L.4)
		% and also vṛttikāra (1.1e L)????
		% TODO: clarify multiple meanings? or add some in that are used in text??

		\litem{rīti} ``style'' (1.1a A, 1.1a L)
		% also (1.1a L.8)?? also vṛttikāra (1.1e L)??
		% TODO: check definition

		\litem{sahṛdayāṇām} ``of sensitive reader'' (1.1e A, 1.1e L, also 1.1 L.1)
		% TODO: check definition, wording as well, add 'a' or 'the'??

		\litem{sahṛdaya} ``having their hearts \textit{with} it'' (1.1e L)

		\litem{sahṛdayatva} ``literary sensitivity'' (1.1e L.8)
		% TODO: keep these two sahṛday terms separate, or make sub items? or just make all one long line?


		\litem{prasiddha} 1.\ ``well known to all'' and ``ornamented'' (1.4 L) \{used in 1.4 K: ``well-known elements [of poetry]'' \& in 1.4 A: ``known, ornamented, elements [of poetry]''\},
		``external constituents'' [K. Kris.\ p.8]

		\litem{lakṣa} ``that by which something is recognized'' (1.1e L)

		\litem{lakṣaṇa} ``definition'' (1.1e L)
		% lakṣaṇa
		% (H2) [Printed book page 892,1]
		% lakṣaṇa mfn. indicating, expressing indirectly, Vedântas. [ID=180381]
		% lakṣaṇa m. Ardea Sibirica, L. [ID=180382]
		% N. of a man, Rājat. (often confounded with, lakṣmaṇa) [ID=180383]
		% (H2B) [Printed book page 892,1]
		% lakṣaṇā a f. See s.v. [ID=180384]
		% lakṣaṇa n. (ifc. f(ā). ) a mark, sign, symbol, token, characteristic, attribute, quality (ifc. = ‘marked or characterized by’, ‘possessed of’), Mn. ; MBh. &c. [ID=180385]
		% lakṣaṇa n. a stroke, line (esp. those drawn on the sacrificial ground), ŚBr. ; GṛŚrS. [ID=180386]
		% a lucky mark, favourable sign, GṛŚrS. ; Mn. ; MBh. &c. [ID=180387]
		% a symptom or indication of disease, Cat. [ID=180388]
		% a sexual organ, MBh. xiii, 2303 [ID=180389]
		% a spoon (?), Divyâv [ID=180390]
		% accurate description, definition, illustration, Mn. ; Sarvad. ; Suśr. [ID=180391]
		% settled rate, fixed tariff, Mn. viii, 406 [ID=180392]
		% a designation, appellation, name (ifc. = ‘named’, ‘called’), Mn. ; MBh. ; Kāv. [ID=180393]
		% a form, species, kind, sort (ifc.= ‘taking the form of’, ‘appearing as’), Mn. ; Śaṃk. ; BhP. [ID=180394]
		% the act of aiming at, aim, goal, scope, object (ifc. = ‘concerning’, ‘relating to’, ‘coming within the scope of’), APrāt. ; Yājñ. ; MBh. ; BhP. [ID=180395]
		% reference, quotation, Pāṇ. i, 4, 84 [ID=180396]
		% effect, operation, influence, ib. i, 1, 62 &c. [ID=180397]
		% cause, occasion, opportunity, R. ; Daś. [ID=180398]
		% observation, sight, seeing, W. [ID=180399]

		\litem{guṇavṛtti} ``secondary usage'' (1.18 K), ``indication'' [K. Kris.\ p.35],
		but K. Kris.: \textit{lakṣaṇayā} ``secondary usage'' (2.9 A, p51) which in IMP is ``metonymy''??
		% TODO see where metonymy used before in IMP.
		``secondary or associated meaning'' by Ānanda (1.1 K.2)
		Also see term \#~\ref{guṇa} and \#~\ref{vṛtti}
		% TODO: get other refs to metonymy - sort of used in 2.7 K.1, EC p145, used in 2.9 A & L
		% TODO: see footnote 1.1 K.2 - bhākta is used closely with guṇavṛtti
		% TODO: is it close to as simple as Ānanda uses guṇavṛtti and Abhinava uses lakṣaṇā ??


		% TODO CHECK ALL lakṣaṇā such words for long ā !!!
		% lakṣaṇā is correct, see 1.1 K.2
		% lākṣaṇika is correct, see 1.1 K.2
		% lakṣaṇayā ?
		% lakṣayatām ?

		% TODO: figure this mess out. see EC p.77 1.1 K.2...
		% TODO: see footnote 1.1 K.2 - bhākta is used closely with guṇavṛtti
		\begin{itemize}
			\item \textbf{Ānanda's two types of ``secondary\-/associated meaning''} (1.1 K.2)
			      \begin{enumerate}
				      \litem{upacāra} ``metaphorical''
				      \litem{lakṣaṇā} ``relational'', ``metonymy'' (2.9 A)
				      % (H2) [Printed book page 892,1]
				      % lakṣaṇā b f. aiming at, aim, object, view, Hariv. [ID=180434]
				      % [Printed book page 892,2]
				      % indication, elliptical expression, use of a word for another word with a cognate meaning (as of ‘head’ for ‘intellect’), indirect or figurative sense of a word (one of its three Arthas; the other two being abhidhā or proper sense, and vyañjanā or suggestive s°; with sāropā, the placing of a word in its figurative sense in apposition to another in its proper s°), Sāh. ; Kpr. ; Bhāṣāp. &c. [ID=180435]
				      % \item[upacāra] ``metaphorical type of secondary/associated meaning''
				      % \item[lakṣaṇā] ``relational type of secondary/associated meaning'' 
			      \end{enumerate}

			\item \textbf{Abhinava's two types of ``secondary\-/associated meaning''} (1.1 K.2)
			      % TODO: verify: ``secondary usage'' (1.18 L)
			      \begin{enumerate}
				      \litem{gauṇa} ``metaphorical'', ``qualitative'' (1.18 L)
				      \litem{lākṣaṇika} ``relational''
				      % \item[gauṇa] ``metaphorical type of secondary/associated meaning''
				      % \item[lākṣaṇika] ``relational type of secondary/associated meaning''
			      \end{enumerate}
		\end{itemize}

		\litem{bhāsa} true or proper correlate (4.5 A, assumed from term \#~\ref{ābhāsa}),
		``impression made on the mind'' (M.W. p.756)
		% (H2) [Printed book page 756,1]
		% bhāsa m. light, lustre, brightness (often ifc.), MBh. ; Hariv. ; Kathās. [ID=150601]
		% impression made on the mind, fancy, MW. [ID=150602]
		% TODO: do we get bhasa or bhāsa alone in the text, or did i just back-assume this def. from ābhāsa?

		\litem{ābhāsa} ``false or improper correlate'' (4.5 A),
		``semblance'' (from usage in compounds, e.g.\ term \#~\ref{rasābhāsa},~\ref{bhāvābhāsa}),
		``imitation'' (2.3 L)

		\litem{rasābhāsa} ``semblance of sentiment'' [K. Kris.] (2.3 K)
		Also see term \#~\ref{rasa} \& \#~\ref{ābhāsa}

		\litem{bhāvābhāsa} ``improper emotion'' (1.4g L);
		``semblance of mood'' [K. Kris.] (2.3 K)
		Also see term \#~\ref{bhāva} \& \#~\ref{ābhāsa}
		% TODO: two diff meanings?? or just diff wordings... 'semblance' being in line with 'it works for now until we realize later it was improper (around 1.4g L or footnote??)

		\litem{bhāvapraśānti} ``(rise and) cessation'' of mood/emotion [\textit{bhāva}, term \#~\ref{bhāva}] (2.3 K) Also see ninth rasa, \textit{śānta} (`peace')
		% TODO: is connection to peace correct?
		% praśānti
		% (H3) [Printed book page 695,2]
		% pra-śānti f. sinking to rest, rest, tranquillity (esp. of mind), calm, quiet, pacification, abatement, extinction, destruction, MBh. ; Kāv. &c. [ID=137533]
		%
		%
		% praśānta
		% (H3) [Printed book page 695,1]
		% pra-°śānta a mfn. tranquillized, calm, quiet, composed, indifferent, Up. ; Mn. ; MBh. &c. [ID=137511]
		% (in augury) auspicious, boni ominis, Var. [ID=137512]
		% extinguished, ceased, allayed, removed, destroyed, dead, MBh. ; Kāv. &c. [ID=137513]
		% (H1) [Printed book page 695,2]
		% pra-śānta b &c. See under pra-√śam. [ID=137549]
		%
		%
		% śānta
		% (H2) [Printed book page 1064,2]
		% 1. śānta mfn. (perhaps always w.r. for 1. śāta q.v.) = śānita, L. [ID=215264]
		% thin, slender, Hariv. ; R. (Sch.) [ID=215265]
		% (H1) [Printed book page 1064,2]
		% 2. śānta mfn. (fr. √1. śam) appeased, pacified, tranquil, calm, free from passions, undisturbed, Up. ; MBh. &c. [ID=215270]
		% soft, pliant, Hariv. [ID=215271]
		% gentle, mild, friendly, kind, auspicious (in augury; opp. to dīpta), AV. &c. &c. [ID=215272]
		% abated, subsided, ceased, stopped, extinguished, averted (śāntam or dhik śāntam or śāntam pāpam, may evil or sin be averted! may God forfend! Heaven forbid! not so!), ŚBr. ; MBh. ; Kāv. &c. [ID=215273]
		% rendered ineffective, innoxious, harmless (said of weapons), MBh. ; R. [ID=215274]
		% come to an end, gone to rest, deceased, departed, dead, died out, ib. ; Ragh. ; Rājat. [ID=215275]
		% purified, cleansed, W. [ID=215276]
		% śānta m. an ascetic whose passions are subdued, W. [ID=215277]
		% tranquillity, contentment (as one of the Rasas q.v.) [ID=215278]
		% N. of a son of Day, MBh. [ID=215279]
		% of a son of Manu Tāmasa, MārkP. [ID=215280]
		% of a son of Śambara, Hariv. [ID=215281]
		% of a son of Idhma-jihva, BhP. [ID=215282]
		% of a son of Āpa, VP. [ID=215283]
		% of a Devaputra, Lalit. [ID=215284]
		% (H1B) [Printed book page 1064,2]
		% śāntā f. (in music) a partic. Śruti, Saṃgīt. [ID=215285]
		% Emblica Officinalis, L. [ID=215286]
		% Prosopis Spicigera and another species, L. [ID=215287]
		% a kind of Dūrvā grass, L. [ID=215288]
		% a partic. drug (= reṇukā), L. [ID=215289]
		% N. of a daughter of Daśa-ratha (adopted daughter of Loma-pāda or Roma-pāda and wife of Ṛṣya-śṛṅga), MBh. ; Hariv. ; R. [ID=215290]
		% (with Jainas) of a goddess who executes the orders of the 7th Arhat, L. [ID=215291]
		% of a Śakti, MW. [ID=215292]
		% (H1B) [Printed book page 1064,2]
		% śānta n. tranquillity, peace of mind, BhP. [ID=215293]

		\litem{bhāvapraśama} ``termination'' of mood/emotion (1.5 L),
		``the cessation of an emotion'' (2.3 L),
		\{Seemingly used interchangeably with \textit{bhāvapraśānti} term \#~\ref{bhāvapraśānti}\}
		% TODO: nbsp ~ between 1.5 and L ??



		\litem{rasavadalaṅkāra} ``figurative sentiment'' [K. Kris.\ p.41] (?)
		See term \#~\ref{rasavat}
		% TODO: add reference

		\litem{rasavat} a ``figure of speech'' different from \textit{rasadhvani} (2.4 Intro. A),
		%``ornamental'' (2.4 Intro. A.1), % TODO: useful??
		``that which contains rasa in a subordinated position'' (2.3 L),
		``Daṇḍin says merely that it was a figure charming with \textit{rasa} (2.275).
		Bhāmaha says litte more: `The figure \textit{rasavat} is where the rise of a \textit{rasa} as śṛṅgāra is clearly exhibited. It is a locus of \textit{rasa}, \textit{sthāyibhāva}, \textit{sañcārin} (= \textit{vyabhicārin}), \textit{vibhāva}, and dramatic portrayal'\,'' (2.4 A.1)% TODO: useful?? or too much beyond 'diff than rasadhvani??
		% TODO: get term or definition

		\litem{---------} ``non-sequential type'' of \textit{dhvani} (2.3 K),
		``undiscerned sequentiality'' [K. Kris.\ p.41, 2.3 K],
		``without apparent sequence [from literal meaning to suggested meaning]'' (2.4 Intro. A)
		[same translation, K. Kris.\ p.41 last paragraph]% TODO: keep this?? extra/does'nt add anything?? ALSO CHECK SAME NOW WITH ADDTION OF [notes]
		% TODO: fix up references
		% TODO: get Sanskrit word??? or keep as english term? or put under dhvani???



		\litem{---4} ``pleasing to the ear'' (2.7 A), ``sound-harmony'' [K. Kris.\ p.51], Note: ``found alike in force [and sweetness]'' WHERE FROM?
		% TODO: where from?
		% TODO: get term or definition

		\litem{dīpti} ``excitement'' (2.9 K, A, L), ``fiery'' (1.1a L)
		% TODO: add something about ojas or move under it??

		% TODO: get def w/o ???
		\litem{racanā} ``structures''? (2.10 A), ``composition'' [K. Kris.\ p.???]
		% Partial MW Def: [Printed book page 861,1]
		% mostly f(ā). arrangement, disposition, management, accomplishment, performance, preparation, production, fabrication, MBh. ; Kāv. &c. [ID=173268]
		% a literary production, work, composition, VarBṛS. ; Sāh. [ID=173269]
		% style, Sāh. [ID=173270]
		% putting on, wearing (of a garment), Mṛcch. [ID=173271]
		% arrangement (of troops), array, Pañcat. [ID=173272]
		% contrivance, invention, Kathās. ; BhP. [ID=173273]
		% a creation of the mind, artificial image, Jaim. [ID=173274]



		% NOTE: below also from 1.1 K.1, footnore says to also see 1.13l A & L? assume not 1.131...
		\litem{śabdaḥ} (a): ``a word which gives rise to suggestion'' (1.13 L)
		% TODO: is this about the SOUND of a word, vs the MEANING of a word?? see 'meaning' in arthaḥ right below...

		\litem{arthaḥ} (b) ``a meaning which gives rise to a suggestion'' (1.13 L)

		\litem{vyāpāraḥ} (c) ``operation'' (1.5 L),
		% TODO: check 1.5 L def...
		``the operation, the suggestion of the implicit meaning'' (1.13 L)

		\litem{vyaṅgyam} (d) ``the suggested meaning itself'' (1.13 L)
		% TODO: Use this footnote below?????
		%``vācya is used in distinction from vyaṅgya'' (1.5 A.1)

		\litem{samudāyaḥ} (e) ``the group; or a poem which embodies all the above factors'' (a--d) (1.13 L) Note: (a--e) given by Abhinavagupta (1.1 K.1, EC p.76)

		\litem{samaya}  ``conventions by which words transmit meaning'' (1.1 L) \{Note: this is an alternative idea to dhvani, argued against in the Dhvanyāloka\}

		\litem{anubandha} ``pertinent point'' (1.1 L.1, EC p.79)
		% Partial MW Def: [Printed book page 36,2]
		% anu-bandha m. binding, connection, attachment [ID=6866]
		% uninterrupted succession [ID=6869]
		% sequence, consequence, result [ID=6870]
		% intention, design [ID=6871]
		% motive, cause [ID=6872]
		% (in phil.) an indispensable element of the Vedānta [ID=6880]
		\begin{enumerate}
			\litem{abhidheya} ``the subject to be treated''
			\litem{prayojana} ``purpose''
			\litem{sambandha} ``suppose that sambandha refers to the connection between the subject and the purpose''
			\litem{adhikāra} ``the qualificcation required of the reader''
		\end{enumerate}

		\litem{lakṣayatām} those ``who are noticing'' (1.1e A); ``describing it by means of a definition'' (1.1e L). Possibly used differently by Ānandavardhana and Abhinavagupta, see 1.1e A.2.

		\litem{ānanda} ``bliss'' (1.1e L), ``delight''? (1.1 K), also name of the author \{Ānandavardhana\}

		\litem{saṅghaṭanā} ``arrangement'' (1.1a L), ``texture'' (1.1a L), ``certain degrees of compounding'' (1.1a A.3)
		% other notes: 1.1a A.3 -> 3.5 K, 3.5 A.1


		%TODO: the 3 parts of poetry according to Bhaṭṭanāyaka, 1.1e L EC p.87

		\litem{carvaṇā} ``tasting'' (2.4 L)

		\litem{rasacarvaṇā} ``relishing of \textit{rasa}'' (1.1e L),
		described as ``it is the relishing of \textit{rasa} that gives [poetry] its life'' (1.1e L)

		\litem{rasanā} ``relishing (\textit{rasanā}) is a special kind of perception'' (2.4 L)
		% TODO: ``texture'' (4.5 A) - seems wrong...


		% TODO 'great poet' 1.4 L


		\litem{anurāga} ``stimulation'' (1.4a L)

		\litem{abhidhā} ``designation'' (1.4a L),
		``denotation'' (1.5 L);
		% TODO: add: ``direct denotation'' (2.4 L) ???
		``name, appellation'' or
		``the literal power or sense of a word'' or
		``a word, sound'' (M.W. p.63)
		% abhidhā (H2) [Printed book page 63,2]
		% 2. abhi-dhā f. name, appellation [ID=11507]
		% the literal power or sense of a word, Sāh. [ID=11508]
		% a word, sound, L. [ID=11509]

		\litem{bhāvanā} ``aesthetic efficacy'' (1.4a L)
		``i.e., the ability to create \textit{rasa}'' (2.4 L);
		% above also seen in (1.5 L.7) % use this???
		same translation for \textit{bhāvakatva} ``efficacy'' (2.4 L) % TODO: use this??? make seperate? explain ana katva etc??

		\litem{vastu} ``situation'' (1.4g L)

		\litem{āsvāda} ``relish'' (1.5 L.2),
		``relishing'' (2.4 L)

		\litem{āsvādyamāna} ``(the process of) being relished'' (1.4g L)
		% TODO: check on def vs () words

		\litem{hṛdayasaṃvāda} ``sympathetic response'' (1.5 L.2, 2.4 L)

		\litem{druti} ``melting of the mind''? \{a kind of relishing/aesthetic enjoyment\} (1.5 L.3, 2.4 L)

		\litem{vistara} ``expansion'' \{a kind of relishing/aesthetic enjoyment\} (1.5 L.3, 2.4 L)

		\litem{vikāsa} ``radiance'' \{a kind of relishing/aesthetic enjoyment\} (1.5 L.3, 2.4 L),
		``expansion'' (2.9 L)
		% TODO: group these three under āsvāda or something??

		\litem{bhoga} ``enjoyment'' (2.4 L),
		``aesthetic pleasure'' (2.4 L)

		\litem{bhogakṛttva} ``efficacy of aesthetic enjoyment'' or ``the power of aesthetic enjoyment'' (2.4 L) % TODO: use???
		% TODO: how connected with āsvāda??

		\litem{bhogīkaraṇa} ``causing aesthetic enjoyment'' (2.4 L)

		\litem{cittavṛtti} ``stable state of mind'' (1.4g L),
		``thought-trend'' (1.5 L), % TODO: is this confusing or disproved? see... later on i forget where...
		NB: the \textit{sthāyibhāva}s are a type of `state of mind' (2.3 L)

		\litem{parisphurati} ``it makes itself felt''?? (1.4g L)
		% parisphur
		% (H1) [Printed book page 604,2]
		% pari-√sphur P. -sphurati, to throb, quiver, vibrate, Kāv. ;
		% to glitter, gleam, BhP. ;
		% to burst forth, appear, Kull. [ID=119089]		

		\litem{sphurati} ``makes itself felt'' (1.4g L) TODO: diff from~\ref{parisphurati}??
		% sphur
		% (H1) [Printed book page 1270,3]
		% Whitney Roots links: sphṛ & Westergaard Dhatupatha links: 28.95
		% 1. sphur (cf. √sphar) cl. 6. P. (Dhātup. xxviii, 95 ) sphurati (mc. also °te; p. sphurat and sphuramāṇa [qq.vv.]; only in pres. base, but See apa-√sphur; Gr. also pf. pusphora, pusphure; fut. sphuritā, sphuriṣyati; aor. asphorīt; Prec. sphūryāt; inf. sphuritum),
		% to spurn, RV. ; AV. ;
		% to dart, bound, rebound, spring, RV. ; MBh. ; Kāv. ;
		% to tremble, throb, quiver, palpitate, twitch (as the nerves of the arm, Śak. ), struggle, Kauś. ; MBh. &c.;
		% to flash, glitter, gleam, glisten, twinkle, sparkle, MaitrUp. ; R. &c.;
		% to shine, be brilliant or distinguished, Rājat. ; Kathās. ; MārkP. ;
		% to break forth, burst out plainly or visibly, start into view, be evident or manifest, become displayed or expanded, NṛsUp. ; MBh. &c.;
		% to hurt, destroy, Naigh. ii, 19 :
		% Caus. sphorayati (aor. apusphurat or apuspharat), to stretch, draw or bend (a bow), Bhaṭṭ. ;
		% to adduce an argument, Śaṃk. Sch.;
		% to cause to shine, eulogize, praise excessively, Pañcad. ;
		% sphurayati, to fill with (instr.), Lalit. :
		% Desid. pusphuriṣati Gr.:
		% Intens. posphuryate, posphorti. [ID=256968]
		% sphur [cf. Gk. σπαίρω, σφυρόν; Lat. sperno; Lith. spírti; Germ. sporo, spor, Sporn; Eng. spur, spurn.] [ID=256968.1]
		% (H2) [Printed book page 1271,1]
		% 2. sphur (ifc.) quivering, trembling, throbbing, Śiś. ii, 14. [ID=256969]

		\litem{vāsanā} ``latent impressions'' (1.4g L) \{text has note: `see 2.4 L.6'\},
		``minds are characterized by a great variety of latent impressions (\textit{vāsanā}).'' (2.4 L),
		``proclivity'' (2.7 L)

		\litem{sāmarthya} ``force'': ``When the suggestion of \textit{rasa} is ascribed to a word, the force (\textit{sāmarthya}), that is, the cooperating force, viz., the \textit{vibhāva}s, etc., is the directly denoted meaning.'' (1.4g L)
		% TODO: delete or keep?

		\litem{śakti} ``force''?: ``When the suggestion of \textit{rasa} is ascribed to the directly denoted meaning \dots
		then the force (\textit{sāmarthya}, \textit{śakti}) of this meaning is the totality of denotative words arranged in their particular way.'' (1.4g L)
		% TODO: delete or keep?

		\litem{saṃsarga} ``syntax'' (1.4g L)
		% TODO: ok can prob delete this one...

		\litem{-------} ``memory elements'' (1.5 L)
		% TODO: get sanskrit word

		\litem{alaukika} ``super-normal'' (1.18 L)

		\litem{rasatā} ``aesthetic relish'' (1.18 L)

		\litem{rasapratīti} ``apprehension of \textit{rasa}'' (1.18 L)

		\litem{vijñā} ``(give) understanding''?? (1.18 L)

		\litem{anubhāvayati} ``experience'' (1.18 L)

		\litem{anubhavana} ``experiencing'' (1.18 L)
		% TODO: check no long a...

		\litem{saṃskāra} ``memory bank'' (1.18 L)

		\litem{rasyamāṇatā} ``a being tasted, a gustation, of beauty'' (1.18 L)

		\litem{svabhāvavacana} ``one's own nature' (2.3 L)

		\litem{prakāra} ``variety'' (2.3 L)

		\litem{pratipatti} ``apprehension'' (2.4 L)

		\litem{pratipattuḥ} ``audience's'' \{pratipattṛ is the root noun\} (2.3 L)
		% pratipattṛ
		% (H3) [Printed book page 667,1]
		% prati-°pattṛ mfn. one who perceives or hears, Sāh. [ID=132535]
		% [Printed book page 667,2]
		% one who comprehends or understands, Śaṃk. [ID=132536]
		% one who maintains or asserts, ĀpŚr. Sch. [ID=132537]
		% TODO: delete prob?

		\litem{asaṃlakṣyakramavyaṅgya} ``--------''?? (2.3 L)
		% TODO: look up. also: lakṣaṇā? vyaṅgya?

		\litem{pratīyate} ``perceived'' (2.4 L)

		\litem{gocara} ``direct object'' delete?? (2.4 L)

		\litem{avabhāsa} ``appearance (or semblance, \textit{avabhāsa}) of a stable emotion in the actor'' (2.4 L)
		% TODO: keep or delete? is this the acting or the emotion, vs a the semblance of a false emotion? or same-ish? diff sanskrit word...
		% avabhāsa
		% (H2) [Printed book page 101,2]
		% ava-bhāsa m. splendour, lustre, light [ID=17728]
		% appearance (especially ifc. with words expressing a colour), Jain. ; Suśr. [ID=17729]
		% (in Vedānta phil.) manifestation [ID=17730]
		% reach, compass, see, śravaṇāvabh°. [ID=17731]

		\litem{vyutpādana} ``educative effect'' (2.4 L)
		% TODO: prob delete. don't think comes up again, was just a cool word.

		\litem{rāga} ``passion'' (2.8 L) TODO: correct? keep?




	\end{enumerate}
\end{multicols}

% TODO: look through all of above terms and see which to remove (ok prob just comment out)
% TODO: rearrange above terms? or group some together?


\end{document}


% Delete this section later, used for now for an easy palce to copy letters from for updates to works
%
% ś
% 
% Ś
% 
% ṛ
% 
% ṅ
% 
% ṇ
%
% ṣ
% 
% ā
% 
% Ā
% 
% ī
% 
% ṭ
% 
% ṃ
%
% ḥ
