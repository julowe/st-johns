\documentclass[12pt]{article}

\usepackage{indentfirst}
\usepackage{enumitem}

\usepackage[top=1in, bottom=1.25in, left=1.25in, right=1.25in]{geometry}



% Don't print page numbers - assuming some overall page number is added for EC Manual printing
\pagenumbering{gobble}

% if set to zero, do not print numbers before section headers
\setcounter{secnumdepth}{0}

\title{Some Notes on the Dhvanyāloka}
\author{Bruce M. Perry}
\date{October 10, 2002}


\begin{document}

\maketitle

\section{Introduction}

One of the principal achievements of Indian aesthetics (\textit{alaṅkāra}) is its analysis of the nature of the poetic experience.
According to its two foremost theorists, poetry consists in the power of suggestion (\textit{dhvani}), and, conversely, produces in the audience an experience of relishing (\textit{rasa}) that is not emotional but transcendental (\textit{alaukika}).
The crucial texts of these theorists, however, do not lend themselves to a straightforward study.
Anandavardhana wrote verses (\textit{kārikā}), upon which he composed a commentary (\textit{vṛtti}), the \textit{Dhvanyāloka}.
Abhinavagupta subsequently composed a commentary on both the verses and commentary of Ānandavardhana, called the \textit{Locana}.
Given the length, complexity, and seeming randomness of topics taken up in these works, a simplifying approach seems in order.
To provide a roadmap, as it were, of the rasa-theory, a chapter from A. B. Keith's \textit{The Sanskrit Drama}, ``The Sentiments'', is first provided.
Select passages from Ānandavardhana and Abhinavagupta, from the excellent translation of Ingalls, Masson, and Patwardhan, are then presented; these selections contain the most direct, if not exhaustive, formulations or discussions of \textit{dhvani} and \textit{rasa}.

Please note that the scholarly footnotes to the selections are supplied for completeness' sake only: they are not crucial to understanding the texts.
The complete translation will be on reserve in the library.


Bruce M. Perry

10/23/02

% insert page break
\newpage


\section{Terms}

Some helpful items from Ingalls, Masson, and Patwardhan (IMP) (see further xerox pages 15--17):

\begin{description}
	\item[anubhāva] consequent

	\item[bhāva] emotion

	\item[rasa] ‘flavor’ / ‘sentiment’ -- based on the sthāyibhāvas respectively.
	      1) erotic (śṛṅgāra),
	      2) comic (hāsya),
	      3) tragic (karuṇa),
	      4) furious/cruel (raudra),
	      5) heroic (vīra),
	      6) fearsome/timorous (bhayānaka),
	      7) gruesome/loathsome (bībhatsā),
	      8) wondrous (adbhuta),
	      9) peace (śānta) [later addition]

	\item[vibhāva] ‘determinant’:
	      a) ‘objective determinant’ (ālambanavibhāva);
	      b) ‘stimulative determinant’ (uddīpanavibhāva)

	\item[vyabhicārin/vyabhicāribhāva] temporary/transient state of mind

	\item[sthāyibhāva] ‘abiding emotion’:
	      1) sexual desire (rati),
	      2) laughter (hāsa),
	      3) grief (śoka),
	      4) anger (krodha),
	      5) heroic energy (utsāha),
	      6) fear (bhaya),
	      7) disgust (jugupsā),
	      8) wonder/amazement (vismaya)

\end{description}


\subsection{abbreviations used}

\begin{description}
	\item[K] \textit{Kārikā} (verse) and\dots

	\item[A] \textit{Dhvanyāloka} (commentary: vṛtti) `Light on Suggestion', by Rājānaka \\ Ānandavardhana, Kashmiri [9th AD]

	\item[L] \textit{Locana} `The Eye' by Abhinavagupta, Kashmiri [10th AD]
\end{description}


\subsection{Some authors mentioned}

\begin{description}
	\item[Bhāmaha] author of works on literary criticism, largely concerned with the figures of speech [8th AD]

	\item[Bharata] (legendary) author of the Nāṭyaśāstra (\textit{BhNŚ}), an ancient manual on dramaturgy, source of first eight sthāyibhāva and rasa

	\item[Bhattanayaka] critic of Ānandavardhana, apparently a Mīmāṃsaka, answered by Abhinavagupta
\end{description}


\end{document}
