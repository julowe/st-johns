%Typeset with XeLaTex !!!!
\documentclass{article}


% Set page size and margins
% book is 7.75"x5" - so take away some to...

%NOTE: page height is twice width of book, need to fold in half
% \usepackage[showframe,height=9.75in,width=7.6in]{geometry}
\usepackage[height=9.75in,width=7.6in]{geometry}

% Useful packages
\usepackage{graphicx}
\usepackage{enumitem}
\usepackage{multicol}
% \usepackage{layout}


% if set to zero, do not print numbers before section headers
\setcounter{secnumdepth}{0}

% don't print page numbers
\pagenumbering{gobble}

\begin{document}

% \layout

%%%%%%%%%%%%%%%%%%%%%%%%%%%%%%%%%%%%%%%%%%
%%%
%%%
%%%
%%%
%%%      Translation Notes - damaged text conventions
%%%
%%%
%%%
%%%
%%%%%%%%%%%%%%%%%%%%%%%%%%%%%%%%%%%%%%%%%%

% \section{Conventions for Damaged Text}
\noindent \textbf{Conventions for Damaged Text}
% \vspace{0.15em}

% normal size font gts us to X pages ...  with 0.25em itemsep
\begin{small}% this gets us to X page with 0.15em itemsep
% \begin{footnotesize}% this only gets us to X pages, use above

% \noindent \textbf{\textit{Note:}} Some explanation is needed of the conventions that mark damaged text:
\vspace{-0.25em}

\begin{itemize}[
    label=,
    leftmargin=1.5em,
    % rightmargin=-1.5em,
    itemindent=-1.5em,
    % nosep,
    itemsep=0.25em,
    parsep=0em
]

    \item[] [Gilgamesh]: Square brackets enclose words that are restored in
passages where the tablet is broken. Small breaks can often be restored
with certainty from context and longer breaks can sometimes be filled
securely from parallel passages. 

    \item[] \textit{Gilgamesh}: Italics are used to indicate insecure decipherments and
uncertain renderings of words in the extant text.

    \item[] [\textit{Gilgamesh}]: Within square brackets, italics signal restorations
that are not certain or material that is simply conjectural, i.e.
supplied by the translator to fill in the context.

    \item[] \dots{}: An ellipsis marks a small gap that occurs where writing is
missing through damage or where the signs are present but cannot be
deciphered. Each ellipsis represents up to one quarter of a verse.

    \item[] \dots\dots\dots\dots: Where a full line is missing or undeciphered the
lacuna is marked by a sequence of four ellipses.

    \item[] * * *: Where a lacuna of more than one line is not signalled by an editorial note it is
marked by a succession of three asterisks.

% Note in addition the following convention:

    \item[] *Humbaba: In old material that has been interpolated into the standard
version of the epic some proper nouns are preceded by an asterisk. This
is to signify that for consistency's sake the name in question (e.g.
Huwawa) has been altered to its later form.

\end{itemize}

% \end{footnotesize}
\end{small}

% \end{multicols}


\vspace{-1.25em}
% \setlength{\columnsep}{5em}
\setlength{\columnwidth}{0.5\textwidth}
\begin{multicols}{3}

%%%%%%%%%%%%%%%%%%%%%%%%%%%%%%%%%%%%%%%%%%
%%%
%%%
%%%
%%%
%%%      Dramatis Personae - Shorter Listing
%%%
%%%
%%%
%%%
%%%%%%%%%%%%%%%%%%%%%%%%%%%%%%%%%%%%%%%%%%


% \section{Dramatis Personae}
\noindent \textbf{Dramatis Personae}
\vspace{0.15em}

% normal size font gts us to asdf pages ...  with 0.25em itemsep
\begin{small}% this gets us to asdf page with 0.15em itemsep
% \begin{footnotesize}% this only gets us to asdf pages, use above

\noindent 
\textbf{\textit{Note:}} This is the listing of Principal Characters from beginning of the Penguin edition.
An acute accent marks the vowel of a stressed syllable.
Where such a vowel falls in an open syllable it will often be long (e.g., Humbaaba).
In some names the position of the stress is conjectural.
\vspace{0.3em}

\begin{itemize}[
        label=,
        leftmargin=1.0em,
        % rightmargin=-1.5em,
        itemindent=-1.0em,
        nosep,
    ]

    \item \textbf{Gilgámesh}, king of the city-state of Úruk

    \item \textbf{Nínsun}, a goddess, his mother

    \item \textbf{Enkídu}, his friend and companion

    \item \textbf{Shámhat}, a prostitute of Uruk

    \item \textbf{Shámash}, the Sun God

    \item \textbf{Humbába}, the guardian of the Forest of Cedar

    \item \textbf{Íshtar}, the principal goddess of Uruk

    \item \textbf{Shidúri}, a minor goddess of wisdom

    \item \textbf{Ur-shanábi}, the ferryman of Uta-napishti

    \item \textbf{Úta-napíshti}, survivor of the Flood

\end{itemize}


% \end{footnotesize}
\end{small}

% \end{multicols}


%page break 
% \clearpage


%%%%%%%%%%%%%%%%%%%%%%%%%%%%%%%%%%%%%%%%%%
%%%
%%%
%%%
%%%
%%%      Glossary of Proper Nouns
%%%
%%%
%%%
%%%
%%%%%%%%%%%%%%%%%%%%%%%%%%%%%%%%%%%%%%%%%%


% \section{Glossary of Proper Nouns}
\vspace{0.5em}
\noindent \textbf{Glossary of Proper Nouns}
\vspace{0.2em}

% normal size font gts us to 3 pages plus 10 lines...
\begin{small}% this gets us to asdf pages
% \begin{footnotesize}% this only gets us to 2.25 pages, use above
\begin{itemize}[
        label=,
        leftmargin=1.0em,
        % rightmargin=-1.5em,
        itemindent=-1.0em,
        nosep,
        % itemsep=0.15em
    ]

    \item \textbf{ABZU}
    See Ocean Below.

    \item \textbf{ADAD}
    The Storm God, venerated as a supreme power especially in Syria and
    Lebanon, where in the epic he has a particular association with the
    Forest of Cedar.

    \item \textbf{AKKA} King of Kish, possibly Gilgamesh's nephew.

    \item \textbf{AN}
    `Sky': the name of heaven in Sumerian, equals Anu in Babylonian.

    \item \textbf{ANSHAN}
    An area of south-western Iran.

    \item \textbf{ANTU}
    The wife of Anu, and at Uruk the mother of Ishtar.

    \item \textbf{ANU}
    The father of the gods, the god of the sky, but also resident in Uruk,
    where he is Ishtar's father.

    \item \textbf{ANUNNA}
    See Anunnaki.

    \item \textbf{ANUNNAKI}
    A traditional name for one of the two divisions of the pantheon, in the
    later periods assigned to the gods of the Netherworld; see Igigi.

    \item \textbf{ARALLI}
    A name of the Netherworld.

    \item \textbf{ARATTA}
    A city-state far away in the highlands of Iran, traditionally a rival of
    Uruk.

    \item \textbf{ARURU}
    Another name for Belet-ili, the Mother Goddess.

    \item \textbf{ASAKKU}
    See Azag.

    \item \textbf{ATRA-HASIS}
    `Surpassing Wise': an epithet of Uta-napishti.

    \item \textbf{AYA}
    Goddess of dawn, the bride of Shamash.

    \item \textbf{AZAG}
    A demon.

    \item \textbf{BELET-ILI}
    `Lady of the Gods': the Mother Goddess, who created mankind with Ea,
    also known as Aruru. As Mother Earth she once enjoyed the attentions of
    Anu, the sky.

    \item \textbf{BELET-ṢERI}
    `Lady of the Desert': the scribe of the Netherworld, who keeps tally for
    Ereshkigal.

    \item \textbf{BIBBU}
    Ereshkigal's butcher and cook.

    \item \textbf{BIRHURTURRA}
    One of Gilgamesh's personal guard. The reading and meaning of the name
    are uncertain.

    \item \textbf{BITTI}
    Or Bidu, `He opens!'. The gate-keeper of the Netherworld.

    \item \textbf{DIMPIKUG}
    A Netherworld deity.

    \item \textbf{DUMUZI}
    `Steadfast Child': the Babylonian Tammuz, lover and husband of Ishtar,
    punished with annual death and descent to the Netherworld.

    \item \textbf{EA}
    The god of the
    freshwater Ocean Below (\emph{Apsû}). The wisest of the gods, he is
    adept in every skill and finds a solution to every problem. His
    expertise enabled the Mother Goddess to create mankind, whom he
    civilized and saved from the wrath of Enlil.

    \item \textbf{EANNA}
    `House of Heaven': the temple of the goddess Ishtar and the god Anu in
    the city of Uruk.

    \item \textbf{EBABBARA}
    `Shining House': the temple of Shamash at Larsa.

    \item \textbf{EBLA}
    A town in north Syria, now Tell Mardikh, south-west of Aleppo.

    \item \textbf{EKUR}
    `Mountain House': the temple of Enlil at Nippur.

    \item \textbf{ENDASHURIMMA}
    and NINDASHURIMMA Ancestors of Enlil, resident in the Netherworld.

    \item \textbf{ENDUKUGA}
    and NINDUKUGA Ancestors of Enlil, resident in the Netherworld.

    \item \textbf{ENGILUA}
    Or Idengilua, a waterway of Uruk. Perhaps a variant form of Idurungal,
    the principal eastern branch of the river Euphrates.

    \item \textbf{ENKI}
    The name of Ea in Sumerian.

    \item \textbf{ENKI}
    and NINKI Ancestors of Enlil, resident in the Netherworld.

    \item \textbf{ENKIDU}
    `Lord of the Pleasant Place': in the Babylonian tradition a wild man
    created by the gods as Gilgamesh's equal, in the Sumerian his favoured
    servant.

    \item \textbf{ENLIL}
    `Lord Air': the divine ruler of Earth and its human inhabitants. Aided
    by Anu, Ea and the Mother Goddess he governs the cosmos. His cult-centre
    was Nippur. His ancestors counted as `dead' gods, and dwelt in the
    Netherworld.

    \item \textbf{ENMEBARAGESI}
    Apparently the elder sister of Gilgamesh, but in history an early ruler
    of Kish, and assumed to be male.

    \item \textbf{ENMESHARRA}
    Uncle of Enlil, resident in the Netherworld.

    \item \textbf{ENMUL}
    and NINMUL Ancestors of Enlil, resident in the Netherworld.

    \item \textbf{ENUTILA}
    Ancestor of Enlil, resident in the Netherworld.

    \item \textbf{ERESHKIGAL}
    `Mistress of the Great Earth': the queen of the Netherworld.

    \item \textbf{ERIDU}
    An ancient city in the far south of Babylonia, the cult-centre of
    Enki-Ea. Now Tell Abu Shahrein, south-west of Nasiriyah.

    \item \textbf{ERRAKAL}
    A manifestation of Nergal as a god of wanton devastation.

    \item \textbf{ETANA}
    A legendary king of Kish, who rode an eagle to heaven, but remained a
    mortal. In the afterlife he was, like Gilgamesh, an officer in the court
    of the Netherworld.

    \item \textbf{GANZIR}
    The first of seven gates of the Netherworld.

    \item \textbf{GILGAMESH}
    A legendary king of Uruk, son of a goddess but doomed to die. In the
    afterlife he became a judge in the Netherworld.

    \item \textbf{GIPAR}
    The private chambers of Inanna in her temple Eanna.

    \item \textbf{GIRSU}
    A city-state of eastern Babylonia, now Telloh, north of Nasiriyah.

    \item \textbf{HAMRAN}
    A mountain of uncertain location, on the way to the Cedar Forest.

    \item \textbf{HUMBABA}
    The monstrous guardian of the Forest of Cedar, appointed by Enlil to
    protect its timber.

    \item \textbf{HUSHBISHA}
    `Its Fury is Fine': a member of Ereshkigal's court.

    \item \textbf{HUWAWA}
    An old form of the name Humbaba.

    \item \textbf{IGIGI}
    A traditional name for one of the two divisions of the pantheon, in the
    later periods assigned to the great gods of heaven; see Anunnaki.

    \item \textbf{INANNA}
    `Queen of Heaven':
    in Sumerian texts the name of Ishtar.

    \item \textbf{IRKALLA}
    `Great City': a name of the Netherworld, and also of its queen, the
    goddess Ereshkigal.

    \item \textbf{IRNINA}
    A name given to the goddess Ishtar, but also a deity of the Netherworld.

    \item \textbf{ISHTAR}
    Deity of the city of Uruk, the goddess of sexual love and war, daughter
    of Anu. Sometimes she is a mature woman, sometimes an impetuous young
    virgin. In heaven she is Venus, daughter of the Moon God.

    \item \textbf{ISHULLANU}
    `Shorty': a cultivator of dates, one of Ishtar's former lovers.

    \item \textbf{KISH}
    An ancient city-state, the early centre of power in northern Babylonia.
    It is modern Tell Uhaimir, east of Babylon.

    \item \textbf{KULLAB}
    Part of the city of Uruk.

    \item \textbf{LARSA}
    Modern Senkereh, a town between Uruk and Ur, which housed one of the
    cult-centres of Shamash.

    \item \textbf{LION-BIRD}
    The Babylonian Anzû, a mythical being imagined sometimes as a
    lion-headed eagle, sometimes as a flying stallion. It dwelt in the
    mountains, where it was defeated by Ninurta.

    \item \textbf{LUGALBANDA}
    `Little Lord': a mortal king of Uruk, later deified. In one tradition he
    was the father of Gilgamesh, in another his guardian deity.

    \item \textbf{LUGALGABANGAL}
    `Chest-endowed Lord': Gilgamesh's minstrel.

    \item \textbf{MAGAN}
    A land far away across the Gulf, possibly modern Oman.

    \item \textbf{MAMMITUM}
    A name of the Mother Goddess, creatress of mankind; see Belet-ili.

    \item \textbf{MARDUK} The god of Babylon, son of Ea, with expertise in exorcism.
    He became king of the gods in the theological reforms of the late second
    millennium, but in the pantheon of Uruk at the time the epic found its
    final form he was a minor figure.

    \item \textbf{MASHU}
    `Twin Peaks': the mountains where the sun rose and set.

    \item \textbf{NAMTAR}
    `Doom': the minister of Ereshkigal, and angel of Death.

    \item \textbf{NANNA} The name of Sîn, the Moon God, in Sumerian.

    \item \textbf{NERGAL}
    A god of plague and war, later the husband of Ereshkigal.

    \item \textbf{NIMUSH}
    A high peak of the Zagros mountains, probably Pir Omar Gudrun
    (Piramagrun) near Suleimaniyah in southern Kurdistan.

    \item \textbf{NINAZU}
    `Lord Doctor': the son of Ereshkigal.

    \item \textbf{NINGAL}
    `Great Lady': the wife of the Moon, the mother of the Sun.

    \item \textbf{NINGISHZIDA}
    `Lord of the True Tree': the chamberlain of the Netherworld, a senior
    figure in the court of Ereshkigal.

    \item \textbf{NINHURSANGA}
    `Lady of the Uplands': a name of the Mother Goddess; see Belet-ili.

    \item \textbf{NINSHULUHHAtumma}
    `Lady whose Hands are Fit for Cleansing': a member of Ereshkigal's
    court.

    \item \textbf{NINSUN}
    `Lady Wild Cow': a minor goddess who was Gilgamesh's mother.

    \item \textbf{NINTU}
    `Lady who Gives Birth': a name of the Mother Goddess; see Belet-ili

    \item \textbf{NINURTA}
    `Lord Earth': the son of Enlil, the epitome of youthful vigour, champion
    of the gods and god of agriculture.

    \item \textbf{NIPPUR}
    The cult-centre of Enlil, modern Nuffar, near `Afaq in central
    Babylonia.

    \item \textbf{NISSABA}
    A cereal goddess,
    patron of writing and accountancy.

    \item \textbf{NUDIMMUD}
    `Man-fashioner': a name of the god Ea which alludes to his part in the
    creation of mankind.

    \item \textbf{NUNGAL}
    A goddess, patron of jails and stewardess of Enlil.

    \item \textbf{OCEAN BELOW} A freshwater ocean below the earth, whence wells and springs
    take their water. The cosmic domain of Ea, known in Babylonian as the
    \emph{Apsû.}

    \item \textbf{PESHTUR}
    `Little Fig': the younger sister of Gilgamesh.

    \item \textbf{PUZUR-ENLIL}
    `Protected by Enlil': the boatbuilder who made Uta-napishti's ark. The
    name can also be read Puzur-Amurru.

    \item \textbf{SEVEN SAGES} Legendary figures in Babylonian mythology, who were sent by the
    god Ea at the beginning of history to civilize mankind.

    \item \textbf{SHAKKAN}
    The god of gazelles, wild asses and other animals.

    \item \textbf{SHAMASH}
    The Sun God, arbiter of justice and patron of travellers, so responsible
    for Gilgamesh's welfare on his adventures. His chief cult-centres were
    Sippar and Larsa.

    \item \textbf{SHAMHAT}
    A cultic prostitute from Uruk, whose task was to entice Enkidu into the
    ways of man. The name means something between `Good-looking' and
    `Well-endowed'.

    \item \textbf{SHAMKATUM}
    An older form of the name Shamhat.

    \item \textbf{SHIDURI}
    `She is my Rampart': a wise goddess who kept an alehouse at the edge of
    the world.

    \item \textbf{SHULPAE}
    `Manifest Hero': a minor deity, husband of the Mother Goddess.

    \item \textbf{SHURUPPAK}
    Modern Fara, an ancient city between Nippur and Uruk.

    \item \textbf{SILILI}
    The mythical mother of all horses, attested only in the Gilgamesh epic.

    \item \textbf{SîN}
    The Moon God. His cult-centre was Ur.

    \item \textbf{SIRION}
    The Anti-Lebanon mountain range, including Mount Hermon.

    \item \textbf{SURSUNABU}
    A variant form of the name Ur-shanabi.

    \item \textbf{ULAY}
    The river Karun in Khuzistan, classical Eulaeus.

    \item \textbf{UR}
    A city in southern Babylonia, modern Tell al-Muqayyar, west of
    Nasiriyah.

    \item \textbf{URLUGAL}
    Gilgamesh's son.

    \item \textbf{UR-SHANABI}
    The ferryman of Uta-napishti.

    \item \textbf{URUK}
    A very ancient city in southern Babylonia, the modern site of Warka,
    east of Samawah.

    \item \textbf{UTA-NAISHTIM}
    A variant form of the name Uta-napishti.

    \item \textbf{UTA-NAPISHTI}
    `I Found Life': the Babylonian Noah, a legendary king of Shuruppak who
    survived the Deluge and was made immortal.

    \item \textbf{UTU}
    The name of the Sun God in Sumerian, equated with Babylonian Shamash.

    \item \textbf{WER} `Violent One': a name of Adad, particularly in the west.

    \item \textbf{ZIUSUDRA}
    `Life of Distant Days': the Sumerian name for Uta-napishti.




\end{itemize}


% \end{footnotesize}
\end{small}

%%%%%%%%%%%%%%%%%%%%%%%%%%%%%%%%%%%%%%%%%%
%%%
%%%
%%%
%%%
%%%      Map - in MultiCol
%%%
%%%
%%%
%%%
%%%%%%%%%%%%%%%%%%%%%%%%%%%%%%%%%%%%%%%%%%


\noindent 
\centering
\includegraphics[height=\linewidth,angle=90,origin=c]{map-cropped.png}
% \includegraphics[height=7cm]{map.png}

\end{multicols}


%%%%%%%%%%%%%%%%%%%%%%%%%%%%%%%%%%%%%%%%%%
%%%
%%%
%%%
%%%
%%%      Map
%%%
%%%
%%%
%%%
%%%%%%%%%%%%%%%%%%%%%%%%%%%%%%%%%%%%%%%%%%


% \noindent 
% \centering
% % \includegraphics[height=\linewidth,angle=90,origin=c]{map.png}
% \includegraphics[height=7cm]{map.png}


%%%%%%%%%%%%%%%%%%%%%%%%%%%%%%%%%%%%%%%%%%
%%%
%%%
%%%
%%%
%%%      Credits
%%%
%%%
%%%
%%%
%%%%%%%%%%%%%%%%%%%%%%%%%%%%%%%%%%%%%%%%%%

\vspace{-0.3cm}
\begin{footnotesize}
    Copied from the Penguin Edition of \textit{The Epic of Gilgamesh}, translated by Andrew George. PDF (and \TeX) available at:\\ https://github.com/julowe/st-johns/blob/main/ref-gilgamesh/gilgamesh-penguin-info.pdf
\end{footnotesize}

%%%%%%%%%%%%%%%%%%%%%%%%%%%%%%%%%%%%%%%%%%
%%%
%%%
%%%
%%%
%%%      Time Chart in Table Format
%%%
%%%
%%%
%%%
%%%%%%%%%%%%%%%%%%%%%%%%%%%%%%%%%%%%%%%%%%

% % \begin{minipage}{0.5\textwidth}
%   \centering
%   \includegraphics[height=3cm]{time-chart.png}
% %   \includegraphics[width=1.4\textwidth,angle=90,origin=c]{time-chart.png}
% % \end{minipage}


%page break 
% \clearpage

% NOTE: generated by pandoc, does not work, did not try to fix.

% \begin{minipage}[t]{0.45\linewidth}
% \begin{longtable}[]{@{}
%   >{\raggedright\arraybackslash}p{(\columnwidth - 10\tabcolsep) * \real{0.1667}}
%   >{\raggedright\arraybackslash}p{(\columnwidth - 10\tabcolsep) * \real{0.1667}}
%   >{\raggedright\arraybackslash}p{(\columnwidth - 10\tabcolsep) * \real{0.1667}}
%   >{\raggedright\arraybackslash}p{(\columnwidth - 10\tabcolsep) * \real{0.1667}}
%   >{\raggedright\arraybackslash}p{(\columnwidth - 10\tabcolsep) * \real{0.1667}}
%   >{\raggedright\arraybackslash}p{(\columnwidth - 10\tabcolsep) * \real{0.1667}}@{}}
% \caption{}\label{hidden1234}\tabularnewline
% \toprule\noalign{}
% \begin{minipage}[b]{\linewidth}\raggedright
% \end{minipage} & \begin{minipage}[b]{\linewidth}\raggedright
% {POLITICAL HISTORY}
% \end{minipage} & \begin{minipage}[b]{\linewidth}\raggedright
% {IN INTELLECTUAL LIFE}
% \end{minipage} & \begin{minipage}[b]{\linewidth}\raggedright
% {GILGAMESH}
% \end{minipage} & \begin{minipage}[b]{\linewidth}\raggedright
% \end{minipage} & \begin{minipage}[b]{\linewidth}\raggedright
% \end{minipage} \\
% \midrule\noalign{}
% \endfirsthead
% \toprule\noalign{}
% \begin{minipage}[b]{\linewidth}\raggedright
% \end{minipage} & \begin{minipage}[b]{\linewidth}\raggedright
% {POLITICAL HISTORY}
% \end{minipage} & \begin{minipage}[b]{\linewidth}\raggedright
% {IN INTELLECTUAL LIFE}
% \end{minipage} & \begin{minipage}[b]{\linewidth}\raggedright
% {GILGAMESH}
% \end{minipage} & \begin{minipage}[b]{\linewidth}\raggedright
% \end{minipage} & \begin{minipage}[b]{\linewidth}\raggedright
% \end{minipage} \\
% \midrule\noalign{}
% \endhead
% \bottomrule\noalign{}
% \endlastfoot
% 3000 & \begin{minipage}[t]{\linewidth}\raggedright
% \textbf{Early city states of Sumer}\\
% {↓}\strut
% \end{minipage} & \begin{minipage}[t]{\linewidth}\raggedright
% Invention of writing\\
% Earliest Sumerian tablets\\
% {↓}\strut
% \end{minipage} & & 3000 & \\
% & & & & & \\
% 2800 & Gilgamesh, king of Uruk & {↓} & Gilgamesh, king of Uruk & 2800
% & \\
% 2600 & \begin{minipage}[t]{\linewidth}\raggedright
% \textbf{Early city states of Sumer}\\
% {↓}\strut
% \end{minipage} & Early Sumerian literature &
% \begin{minipage}[t]{\linewidth}\raggedright
% Gilgamesh deified\\
% in god lists\\
% {↓}\strut
% \end{minipage} & 2600 & \\
% & & & & & \\
% 2400 & & & Gilgamesh worshipped in cult & 2400 & \\
% 2300 & \begin{minipage}[t]{\linewidth}\raggedright
% \textbf{Old Akkadian Empire}\\
% Sargon of Akkade\\
% Naram-Sîn\\
% {↓}\strut
% \end{minipage} & \begin{minipage}[t]{\linewidth}\raggedright
% Akkadian as language of empire\\
% {↓}\strut
% \end{minipage} & & 2300 & \\
% 2200 & & \begin{minipage}[t]{\linewidth}\raggedright
% \textbf{Oral Gilgamesh poems\\
% in Sumerian}and Akkadian?\strut
% \end{minipage} & & 2200 & \\
% & \begin{minipage}[t]{\linewidth}\raggedright
% \textbf{Third Dynasty of Ur}\\
% {↓}\strut
% \end{minipage} & & & & \\
% 2100 & & \begin{minipage}[t]{\linewidth}\raggedright
% Sumerian renaissance\\
% Royal Tablet Houses\\
% Sumerian court literature\\
% Spoken Sumerian dying out\strut
% \end{minipage} & & 2100 & \\
% 2000 & \begin{minipage}[t]{\linewidth}\raggedright
% Shulgi\\
% Fall of Ur\strut
% \end{minipage} & & \begin{minipage}[t]{\linewidth}\raggedright
% \textbf{Oldest copy of a Sumerian\\
% Gilgamesh poem}\strut
% \end{minipage} & 2000 & \\
% 1900 & \begin{minipage}[t]{\linewidth}\raggedright
% \textbf{Isin Dynasty}\\
% {↓}\strut
% \end{minipage} & & & 1900 & \\
% & \begin{minipage}[t]{\linewidth}\raggedright
% \textbf{Larsa Dynasty}\\
% {↓}\strut
% \end{minipage} & & & & \\
% 1800 & \begin{minipage}[t]{\linewidth}\raggedright
% \textbf{Old Babylonian Kingdom\\
% Hammurapi of Babylon}\strut
% \end{minipage} & \begin{minipage}[t]{\linewidth}\raggedright
% Scribal schools\\
% at Ur and\\
% Nippur\strut
% \end{minipage} & \begin{minipage}[t]{\linewidth}\raggedright
% \textbf{Many copies of\\
% Sumerian Gilgamesh\\
% poems}\strut
% \end{minipage} & 1800 & \\
% 1750 & & & & 1750 & \\
% 1700 & Decline of southern Babylonia &
% \begin{minipage}[t]{\linewidth}\raggedright
% Literary creativity\\
% in Akkadian\strut
% \end{minipage} & \begin{minipage}[t]{\linewidth}\raggedright
% Akkadian fragments\\
% \textbf{`Surpassing all other kings'}\\
% \textbf{Old Babylonian Gilgamesh epic}\strut
% \end{minipage} & 1700 & \\
% & {↓} & & & & \\
% 1600 & \begin{minipage}[t]{\linewidth}\raggedright
% Sack of Babylon by the Hittites\\
% \textbf{Kassite Dynasty}\strut
% \end{minipage} & \begin{minipage}[t]{\linewidth}\raggedright
% Very few tablets extant\\
% from this period\strut
% \end{minipage} & The Sealand tablet & 1600 & \\
% 1500 & {↓} & & & 1500 & \\
% 1400 & Amarna Age & \begin{minipage}[t]{\linewidth}\raggedright
% Akkadian as \emph{lingua franca}\\
% Spread of Babylonian texts\\
% to the West\strut
% \end{minipage} & \begin{minipage}[t]{\linewidth}\raggedright
% \textbf{Middle Babylonian versions}\\
% \textbf{of the Gilgamesh epic}\\
% copied in Anatolia,\\
% Palestine, Syria and\\
% Babylonia\strut
% \end{minipage} & 1400 & \\
% 1300 & & & & 1300 & \\
% 1200 & & \begin{minipage}[t]{\linewidth}\raggedright
% Organization and editing of\\
% Babylonian literature\\
% {↓}\strut
% \end{minipage} & \begin{minipage}[t]{\linewidth}\raggedright
% Sîn-leqi-unninni edits\\
% the Babylonian epic into\\
% \textbf{the standard version}\\
% \textbf{`He who saw the Deep'}\\
% {↓}\strut
% \end{minipage} & 1200 & \\
% 1100 & Tiglath-pileser I of Assyria & & & 1100 & \\
% 1000 & & \begin{minipage}[t]{\linewidth}\raggedright
% Very few tablets extant\\
% from this period\\
% {↓}\strut
% \end{minipage} & & 1000 & \\
% 900 & \begin{minipage}[t]{\linewidth}\raggedright
% \textbf{Neo-Assyrian Empire}\\
% {↓}\strut
% \end{minipage} & & & 900 & \\
% 800 & & \begin{minipage}[t]{\linewidth}\raggedright
% Spread of Aramaic in Assyria\\
% and Babylonia\strut
% \end{minipage} & & 800 & \\
% 700 & \begin{minipage}[t]{\linewidth}\raggedright
% Sargon II\\
% Sennacherib\\
% Esarhaddon\strut
% \end{minipage} & & & 700 & \\
% 650 & \begin{minipage}[t]{\linewidth}\raggedright
% Ashurbanipal\\
% Fall of Nineveh\strut
% \end{minipage} & \begin{minipage}[t]{\linewidth}\raggedright
% Royal libraries\\
% at Nineveh\strut
% \end{minipage} & \begin{minipage}[t]{\linewidth}\raggedright
% \textbf{Copies of the\\
% Gilgamesh epic\\
% from Assyria}\strut
% \end{minipage} & 650 & \\
% 600 & \begin{minipage}[t]{\linewidth}\raggedright
% \textbf{Neo-Babylonian Empire}\\
% Nebuchadnezzar II\strut
% \end{minipage} & Spoken Akkadian dying out & & 600 & \\
% 500 & \begin{minipage}[t]{\linewidth}\raggedright
% \textbf{Persian Empire}\\
% Darius, Xerxes\\
% {↓}\strut
% \end{minipage} & & & 500 & \\
% 400 & & \begin{minipage}[t]{\linewidth}\raggedright
% Babylonian literature\\
% copied out and\\
% preserved in\\
% libraries of\\
% temples\\
% and\\
% scholars\\
% {↓}\strut
% \end{minipage} & \begin{minipage}[t]{\linewidth}\raggedright
% \textbf{Copies of\\
% the Gilgamesh epic\\
% from Uruk\\
% and Babylon}\\
% {↓}\strut
% \end{minipage} & 400 & \\
% 300 & \begin{minipage}[t]{\linewidth}\raggedright
% Alexander the Great\\
% \textbf{Greek (Hellenistic) period}\\
% {↓}\strut
% \end{minipage} & & & 300 & \\
% 200 & & & & 200 & \\
% 100 {BC} & \begin{minipage}[t]{\linewidth}\raggedright
% \textbf{Parthian period}\\
% {↓}\strut
% \end{minipage} & & \textbf{Last copies of the Gilgamesh epic} & 100 {BC}
% & \\
% {BC--AD} & \begin{minipage}[t]{\linewidth}\raggedright
% Decline of Babylon\\
% {↓}\strut
% \end{minipage} & & & {BC--AD} & \\
% {AD} 100 & Roman wars & Last cuneiform tablets & & {AD} 100 & \\
% \multicolumn{6}{@{}l@{}}{%
% *N.B. Dates before 1100 {BC} are approximate.} \\
% \end{longtable}
% \end{minipage}


\end{document}
