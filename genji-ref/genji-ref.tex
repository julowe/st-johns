%Typeset with LuLaTex !!!!
\documentclass{article}

% Language setting
% Replace `english' with e.g. `spanish' to change the document language
\usepackage[english]{babel}
%\usepackage{CJKutf8}%pdflatex

%\usepackage{xeCJK}
%\setCJKmainfont{Noto Serif CJK JP}%fonts not found
%\setCJKsansfont{Noto Sans CJK JP}
%\setCJKmonofont{Noto Sans Mono CJK JP}

\usepackage{luatexja}

% Set page size and margins
% Replace `letterpaper' with `a4paper' for UK/EU standard size
\usepackage[letterpaper,top=1.0cm,bottom=1.0cm,left=2.0cm,right=2.0cm,marginparwidth=1.75cm]{geometry}

% Useful packages
\usepackage{graphicx}
\usepackage{enumitem}

\usepackage{multicol}



\title{Your Paper}
\author{You}

\begin{document}
\pagenumbering{gobble}

%%%%%%%%%%%%%%%%%%%%%%%%%%%%%%%%%%%%%%%%%%
%%%
%%%
%%%
%%%
%%%      Genealogical Chart
%%%
%%%
%%%
%%%
%%%%%%%%%%%%%%%%%%%%%%%%%%%%%%%%%%%%%%%%%%


\begin{figure}
	%\setlength\lineskip{-0.5in}
	\makebox[\textwidth][c]{\includegraphics[angle=90,origin=c,width=1.1\linewidth]{genji-genealogical-chart.png}}%
	%\includegraphics[angle=90,origin=c,width=1.05\textwidth]{genji-genealogical-chart.png}%
	%\caption{Caption}
	%\label{fig:key}
\end{figure}


%page break 
\clearpage

%%%%%%%%%%%%%%%%%%%%%%%%%%%%%%%%%%%%%%%%%%
%%%
%%%
%%%
%%%
%%%      Characters by name - Shorter Seidensticker Listing
%%%
%%%
%%%
%%%
%%%%%%%%%%%%%%%%%%%%%%%%%%%%%%%%%%%%%%%%%%


%\section{Characters in The Tale of Genji}%Don't really need this header, as it is obvious

\noindent \textbf{\textit{Note:}} This is the listing of Principal Characters from the Everyman edition, \textit{trans. Seidensticker}

\setlength{\columnsep}{5em}
\begin{multicols}{2}

	%\vspace*{-0.4cm}
	% normal size font gts us to 3 pages plus 10 lines...
	%\begin{small}% this gets us to ~2.75 pages
	%\begin{footnotesize}% this only gets us to 2.25 pages, use above
	\begin{itemize}[
			label=,
			leftmargin=0em,
			rightmargin=-1.5em,
			itemindent=-2em,
			nosep,
		]
		\setlength{\itemsep}{0.25em}

		\item \textbf{Akashi Empress}. Also the Akashi girl, the Akashi princess. Genji's daughter by the Akashi lady. Consort of the emperor regnant at the end of the tale. (Ch. 14, 18, 19, 22, 23, 25, 28, 32--37, 40, 42, 47, 49, 52, 53)

		\item \textbf{Akashi Lady}. Daughter of a former governor of Akashi. Installed by Genji in the northwest quarter at Rokujō. Mother of the Akashi empress. (Ch. 5, 12--14, 18, 19, 23, 25, 28, 33--35, 40, 41)

		\item \textbf{Akikonomu}. Daughter of a former crown prince and the Rokujō lady. Consort of the Reizei emperor. First cousin of Genji and Asagao. (Ch. 9, 10, 14, 16, 19, 21, 24, 28, 32, 34--36, 38, 40, 42)

		\item \textbf{Aoi}. Genji's first wife. Daughter of the Minister of the Left and Princess Omiya. Sister of Tō no Chūjō. Mother of Yūgiri. (Ch. 1, 2, 5, 7, 9)

		\item \textbf{Asagao}. Genji's first cousin. Daughter of a brother of his father. (Ch. 2, 9, 10, 20, 21, 32)

		\item \textbf{Bennokimi}. In attendance upon Kashiwagi and later the Uji princesses. (Ch. 45--52)

		\item \textbf{Eighth Prince}. Brother of Genji and father of the Uji princesses, Oigimi and Nakanokimi, and of Ukifune. (Ch. 45, 46)

		\item \textbf{Emperor}, (1) Genji's father, whose abdication is announced at the beginning of Chapter 9. (Ch. 1, 2, 7--10, 12--14)
		      (2) \textit{see also, Suzaku emperor}, Genji's brother, who succeeds to the throne at the beginning of Chapter 9 and abdicates in Chapter 14. (Ch. 1, 7, 8, 10, 12--14, 17, 21, 33--37, 39)
		      (3) \textit{see also, Reizei emperor}. Thought by the world to be Genji's brother, but in fact his son by Fujitsubo. Succeeds to the throne in Chapter 14 and abdicates in Chapter 35. (Ch. 7, 9, 10, 12--14, 17--19, 21, 29, 31, 33--35, 38, 44, 45)
		      (4) A son of the Suzaku emperor who succeeds to the throne in Chapter 35 and is still reigning at the end of the tale. (Ch. 14, 32, 34--36, 42, 44, 45, 49)

		\item \textbf{Evening Faces, Lady Of The}. A lady of undistinguished lineage who is loved by Tō no Chūjō and bears his daughter Tamakazura. (Ch. 4)

		\item \textbf{Fujitsubo}. Daughter of a former emperor, consort of Genji's father, and mother of the Reizei emperor. (Ch. 1, 5, 7--10, 12--14, 17, 19, 20)

		\item \textbf{Genji}. Son of the emperor regnant at the beginning of the tale.

		\item \textbf{Higekuro}. Son of a Minister of the Right, husband of Murasaki's sister and of Tamakazura, and uncle of the emperor regnant at the end of the tale. (Ch. 24, 29--31, 35)

		\item \textbf{Hotaru, Prince}. Genji's brother. Husband of Makibashira. (Ch. 12, 17, 21, 24, 25, 29--32, 34, 35, 38, 41)

		\item \textbf{Hyōbu, Prince}. Brother of Fujitsubo and father of Murasaki. (Ch. 1, 5, 7, 10, 12, 14, 17, 21, 31, 34, 35)

		\item \textbf{Kaoru}. Thought by the world to be Genji's son, but really Kashiwagi's. (Ch. 36, 37, 42--54)

		\item \textbf{Kashiwagi}. Son of Tō no Chūjō and father of Kaoru. Married to the Second Princess, daughter of the Suzaku emperor. (Ch. 24--27, 29--37)

		\item \textbf{Kōbai}. Younger brother of Kashiwagi. (Ch. 10, 14, 23, 26, 27, 29, 32, 33, 43, 44, 49)

		\item \textbf{Kojijū}. A woman in attendance upon the Third Princess. (Ch. 34--36)

		\item \textbf{Kokiden}. Daughter of the Minister of the Right. Wife of Genji's father, sister of Oborozukiyo, and mother of the Suzaku emperor. (Ch. 1, 7--10, 13, 14, 21)

		\item \textbf{Koremitsu}. Genji's servant and confidant. (Ch. 4, 5, 8, 9, 11, 12, 14, 15, 18, 21)

		\item \textbf{Kumoinokari}. Tō no Chūjō's daughter and Yūgiri's wife. (Ch. 21, 26, 28, 32, 33, 37, 39, 44)

		\item \textbf{Locust Shell, Lady Of The}. Wife of a governor of Iyo. Installed by Genji among his lesser ladies at Nijō. (Ch. 2--4, 16, 23)

		\item \textbf{Makibashira}. Daughter of Higekuro. Wife successively of Prince Hotaru and Kōbai. (Ch. 31, 35, 43, 44)

		\item \textbf{Minister Of The Left}. Several characters. Initially, the husband of Princess Omiya and the father of Aoi and Tō no Chūjō (Ch. 1, 2, 5--10, 12, 14, 19).

		\item \textbf{Minister Of The Right}. Several characters. Initially, the father of Kokiden and Oborozukiyo and the grandfather of the Suzaku emperor (Ch. 1, 8--10, 12, 13).

		\item \textbf{Murasaki}. Daughter of Prince Hyōbu, niece of Fujitsubo, and granddaughter of a former emperor. (Ch. 5--10, 12--14, 18--25, 28, 29, 31--35, 39, 40)

		\item \textbf{Nakanokimi}. Second daughter of the Eighth Prince. (Ch. 45--51)

		\item \textbf{Niou, Prince}. Son of the emperor regnant at the end of the tale and of the Akashi empress. (Ch. 37, 40--43, 45--52)

		\item \textbf{Oborozukiyo}. Sister of Kokiden. (Ch. 8--10, 12, 14, 21, 34, 35)

		\item \textbf{Oigimi}. Oldest daughter of the Eighth Prince. (Ch. 45--47)
		
		\item \textbf{Omi, Lady Of}. Long-lost daughter of Tō no Chūjō. (Ch. 26, 29, 31)

		\item \textbf{Omiya, Princess}. Genji's paternal aunt. Mother of Aoi and Tō no Chūjō. (Ch. 1, 9, 12, 21, 28, 29)

		\item \textbf{Ono, Nun Of}. Ukifune's protector. (?Ch. 51, 53)

		\item \textbf{Orange Blossoms, Lady Of The}. Sister of a lesser concubine of Genji's father. Installed by Genji in the northeast quarter at Rokujō. (Ch. 11--14, 18, 19, 22, 23, 25, 28, 35, 39--41)

		\item \textbf{Reizei Emperor}. Thought by the world to be Genji's brother, but really his son by Fujitsubo. Reigns from Chapter 14 to Chapter 35. (Ch. 7, 9, 10, 12--14, 17--19, 21, 29, 31, 33--35, 38, 44, 45)

		\item \textbf{Rokujō Lady}. Widow of a former crown prince, Genji's uncle. Mother of Akikonomu. (Ch. 4, 9, 10, 12, 14, 35, 36)

		\item \textbf{Rokunokimi}. Daughter of Yūgiri and wife of Niou. (Ch. 42, 49)

		\item \textbf{Safflower Lady}. Impoverished, but of royal origins. Installed by Genji among his lesser ladies at Nijō. (?Ch. 6, 15, 22, 23)

		\item \textbf{Second Princess}. 
		(1) Daughter of the Suzaku emperor and wife of Kashiwagi. (Ch. 35--37, 39, 42, 49)
		(2) Daughter of the emperor regnant at the end of the tale and wife of Kaoru. (Ch. 49, 52)

		\item \textbf{Suzaku Emperor}. Genji's brother. Reigns from Chapter 9 to Chapter 14. (Ch. 1, 7, 8, 10, 12--14, 17, 21, 33--37, 39)

		\item \textbf{Tamakazura}. Daughter of Tō no Chūjō and the lady of the evening faces. (Ch. 4, 22--31, 34, 35, 44)

		\item \textbf{Third Princess}. Daughter of the Suzaku emperor, wife of Genji, and mother of Kaoru. (Ch. 34--38, 41, 42, 45, 49)

		\item \textbf{Tō no Chūjō}. Son of a Minister of the Left and Princess Omiya and brother of Aoi. Father of Kashiwagi, Kōbai, Kumoinokari, Tamakazura, and the Omi lady. (Ch. 1, 2, 4--10, 12, 14, 17, 19, 21, 25, 26, 28--37, 39, 40)

		\item \textbf{Ukifune}. Unrecognized daughter of the Eighth Prince. Half sister of Oigimi and Nakanokimi. (Ch. 49--51, 53, 54)

		\item \textbf{Ukon}. In attendance upon Ukifune. (?Ch. 52)

		\item \textbf{Yokawa, Bishop Of}. Brother of the nun of Ono. (Ch. 53, 54)

		\item \textbf{Yūgiri}. Son of Genji and Aoi. (Ch. 9, 12, 21--44, 47--49, 51)

	\end{itemize}


	%\end{footnotesize}
	%\end{small}

\end{multicols}


%page break 
\clearpage


%%%%%%%%%%%%%%%%%%%%%%%%%%%%%%%%%%%%%%%%%%
%%%
%%%
%%%
%%%
%%%      Chapters Names for Different Translations
%%%
%%%
%%%
%%%
%%%%%%%%%%%%%%%%%%%%%%%%%%%%%%%%%%%%%%%%%%


\begin{table}
	\centering
	\begin{tabular}{llll}
		\textbf{\#} & \textbf{Japanese}      & \textbf{Seidensticker}              & \textbf{Tyler}                \\
		\hline%
		\noalign{\vskip 1mm}% so the text below is not too close to this horizontal rule
		1           & Kiritsubo (桐壺)         & The Paulownia Court                 & The Paulownia Pavilion        \\
		2           & Hahakigi (帚木)          & The Broom Tree                      & The Broom Tree                \\
		3           & Utsusemi (空蝉)          & The Shell of the Locust             & The Cicada Shell              \\
		4           & Yūgao (夕顔)             & Evening Faces                       & The Twilight Beauty           \\
		5           & Waka Murasaki (若紫)     & Lavender                            & Young Murasaki                \\
		6           & Suetsumu Hana (末摘花)    & The Safflower                       & The Safflower                 \\
		7           & Momiji no ga (紅葉賀)     & An Autumn Excursion                 & Beneath the Autumn Leaves     \\
		8           & Hana no En (花宴)        & The Festival of the Cherry Blossoms & Under the Cherry Blossoms     \\
		%8          & Hana no En                & The Festival of the                 & Under the Cherry Blossoms     \\
		%           &                           & Cherry Blossoms                     &                               \\
		9           & Aoi (葵)                & Heartvine                           & Heart-to-Heart                \\
		10          & Sakaki (賢木)            & The Sacred Tree                     & The Green Branch              \\
		11          & Hanachiru Sato (花散里)   & The Orange Blossoms                 & Falling Flowers               \\
		12          & Suma (須磨)              & Suma                                & Suma                          \\
		13          & Akashi (明石)            & Akashi                              & Akashi                        \\
		14          & Miotsukushi (澪標)       & Channel Buoys                       & The Pilgrimage to Sumiyoshi   \\
		15          & Yomogiu (蓬生)           & The Wormwood Patch                  & A Waste of Weeds              \\
		16          & Sekiya (関屋)            & The Gatehouse                       & At the Pass                   \\
		17          & E-Awase (絵合)           & The Picture Contest                 & The Picture Contest           \\
		18          & Matsukaze (松風)         & The Wind in the Pines               & Wind in the Pines             \\
		19          & Usugumo (薄雲)           & A Rack of Cloud                     & Wisps of Cloud                \\
		20          & Asagao (朝顔)            & The Morning Glory                   & The Bluebell                  \\
		21          & Otome (乙女)             & The Maiden                          & The Maidens                   \\
		22          & Tamakazura (玉鬘)        & The Jeweled Chaplet                 & The Tendril Wreath            \\
		23          & Hatsune (初音)           & The First Warbler                   & The Warbler's First Song      \\
		24          & Kochō (胡蝶)             & Butterflies                         & Butterflies                   \\
		25          & Hotaru (螢)             & Fireflies                           & The Fireflies                 \\
		26          & Tokonatsu (常夏)         & Wild Carnations                     & The Pink                      \\
		27          & Kagaribi (篝火)          & Flares                              & The Cressets                  \\
		28          & Nowaki (野分)            & The Typhoon                         & The Typhoon                   \\
		29          & Miyuki (行幸)            & The Royal Outing                    & The Imperial Progress         \\
		30          & Fujibakama (藤袴)        & Purple Trousers                     & Thoroughwort Flowers          \\
		31          & Makibashira (真木柱)      & The Cypress Pillar                  & The Handsome Pillar           \\
		32          & Umagae (梅枝)            & A Branch of Plum                    & The Plum Tree Branch          \\
		33          & Fuji no Uraba (藤裏葉)    & Wisteria Leaves                     & New Wisteria Leaves           \\
		34          & Wakana: Jō (若菜上)       & New Herbs, Part I                   & Spring Shoots I               \\
		35          & Wakana: Ge (若菜下)       & New Herbs, Part II                  & Spring Shoots II              \\
		36          & Kashiwagi (柏木)         & The Oak Tree                        & The Oak Tree                  \\
		37          & Yokobue (横笛)           & The Flute                           & The Flute                     \\
		38          & Suzumushi (鈴虫)         & The Bell Cricket                    & The Bell Cricket              \\
		39          & Yūgiri (夕霧)            & Evening Mist                        & Evening Mist                  \\
		40          & Minori (御法)            & The Rites                           & The Law                       \\
		41          & Maboroshi (幻)          & The Wizard                          & The Seer                      \\
		X           & Kumogakure (雲隠)        &                                     & Vanished into the Clouds      \\
		42          & Niou miya (匂宮)         & His Perfumed Highness               & The Perfumed Prince           \\
		43          & Kōbai (紅梅)             & The Rose Plum                       & Red Plum Blossoms             \\
		44          & Takekawa (竹河)          & Bamboo River                        & Bamboo River                  \\
		45          & Hashihime (橋姫)         & The Lady at the Bridge              & The Maiden of the Bridge      \\
		46          & Shii ga Moto (椎本)      & Beneath the Oak                     & Beneath the Oak               \\
		47          & Agemaki (総角)           & Trefoil Knots                       & Trefoil Knots                 \\
		48          & Sawarabi (早蕨)          & Early Ferns                         & Bracken Shoots                \\
		49          & Yadorigi (宿木)          & The Ivy                             & The Ivy                       \\
		50          & Azumaya (東屋)           & The Eastern Cottage                 & The Eastern Cottage           \\
		51          & Ukifune (浮舟)           & A Boat upon the Waters              & A Drifting Boat               \\
		52          & Kagerō (蜻蛉)            & The Drake Fly                       & The Mayfly                    \\
		53          & Tenarai (手習)           & At Writing Practice                 & Writing Practice              \\
		54          & Yume no Ukihashi (夢浮橋) & The Floating Bridge of Dreams       & The Floating Bridge of Dreams \\
	\end{tabular}
	%\caption{Caption}
	%\label{tab:my_label}
\end{table}



%page break 
\clearpage

%%%%%%%%%%%%%%%%%%%%%%%%%%%%%%%%%%%%%%%%%%
%%%
%%%
%%%
%%%
%%%      Characters by name
%%%
%%%
%%%
%%%
%%%%%%%%%%%%%%%%%%%%%%%%%%%%%%%%%%%%%%%%%%


%\section{Characters in The Tale of Genji}%Don't really need this header, as it is obvious

\noindent \textbf{\textit{Note:}} Characters are listed wherever possible by Japanese designation and identified, with their English appellations and the chapters in which they appear. \textit{This is from the Penguin Edition, trans. Tyler}

\setlength{\columnsep}{5em}
\begin{multicols}{2}

	%\vspace*{-0.4cm}
	% normal size font gts us to 3 pages plus 10 lines...
	%\begin{small}% this gets us to ~2.75 pages
	%\begin{footnotesize}% this only gets us to 2.25 pages, use above
	\begin{itemize}[
			label=,
			leftmargin=0em,
			rightmargin=-1.5em,
			itemindent=-2em,
			nosep,
		]

		\setlength{\itemsep}{0.25em}
		\item \textbf{Akashi no Amagimi}, Akashi no Nyūdō's wife, Akashi no Kimi's mother The Akashi Nun, 12;
		      mother of Akashi no Kimi, 13;
		      the Nun, 18, 19, 34, 35

		\item \textbf{Akashi no Himegimi} (\textit{Akashi no Nyōgo, Akashi no Chūgū}), daughter of Genji and Akashi no Kimi (Born), 14;
		      the young lady, 18, 19, 22, 23, 25, 28, 32;
		      the Consort (\textit{of the Heir Apparent}), 33;
		      the Heir Apparent's Kiritsubo Consort, then Haven, 34;
		      the Kiritsubo Consort, the Consort, 35;
		      the Consort, 36, 37;
		      Her Majesty, the Empress, 40, 42, 47, 49, 52, 53

		\item \textbf{Akashi no Kimi}, Akashi no Nyūdōs daughter, the mother of Genji's daughter 5, 12, 13;
		      the lady from Akashi, 14, 18, 25, 28, 35, 41;
		      the lady at Ōi, 19;
		      Akashi, 23, 33, 34, 40

		\item \textbf{Akashi no Nyūdō}, a former Governor of Harima, father of Akashi no Kimi The (\textit{Akashi}) Novice, 12, 13, 14, 18, 19, 34

		\item \textbf{Akikonomu}, Rokujō no Miyasudokoro's daughter High Priestess of Ise, 9, 10, 14;
		      Her Highness, the former Ise Priestess, the Ise Consort, 16;
		      the Ise Consort, 19;
		      the Ise Consort, Her Majesty, 21;
		      Her Majesty, 24, 28, 32, 34, 35, 36, 38, 40, 42

		\item \textbf{Aoi}, Genji's first wife, Sadaijin's daughter 1,2,5,7,9

		\item \textbf{Asagao}, Shikibukyō no Miya's daughter Daughter of His Highness of Ceremonial, 2;
		      the lady of the bluebells, Her Highness, 9;
		      the lady of the bluebells, the High Priestess of the Kamo Shrine, 10;
		      Her Highness, the Former Kamo Priestess, 20, 21, 32

		\item \textbf{Chūjō}, the son-in-law of Yokawa no Sōzu's sister The Captain, 53

		\item \textbf{Chūjō no Kimi}, Ukifune's mother, Hitachi no Kami's wife 49, 50, 51, 52

		\item \textbf{Dazai no Shōni}, husband of Yūgao's nurse The Dazaifu Assistant, 22

		\item \textbf{Fujitsubo}, Kiritsubo no Mikado's Empress, Reizei's mother Daughter of an earlier Emperor, 1;
		      Her Highness, 5;
		      Her Highness, then Her Majesty, 7;
		      Her Majesty, 8, 9;
		      Her Majesty, then Her Cloistered Eminence, 10, 12, 13, 14, 17, 19;
		      a dream phantom, 20

		\item \textbf{Genji} (\textit{Birth through 12}), 1;
		      a Captain in the Palace Guards, 2, 3, 4, 5, 6;
		      a Captain in the Palace Guards, then a Consultant, 7;
		      a Consultant, 8;
		      the Commander of the Right, 9, 10, 11;
		      no rank, 12;
		      no rank, then promoted to Acting Grand Counselor, 13;
		      becomes Palace Minister, 14;
		      the Commander, then the Acting Grand Counselor, 15;
		      the Palace Minister, 16, 17;
		      His Grace, the Palace Minister, 18, 19, 20;
		      His Grace, the Chancellor, 21, 22, 23, 24, 25, 26, 27, 28, 29, 30, 31, 32;
		      His Grace, the Chancellor, then the Honorary Retired Emperor, 33;
		      His Grace, the Honorary Retired Emperor, 34, 35, 36, 37, 38, 39, 40, 41

		\item \textbf{Gen no Naishi}, a randy old woman The Dame of Staff, 7, 9;
		      the former Dame of Staff, 20

		\item \textbf{Gentlewomen}, nurses, page girls Ateki (\textit{serves Aoi}), 9,
		      Ateki, later called Hyōbu (\textit{daughter of Dazai no Daini; serves Tamakazura}), 22, 24;
		      Azechi (\textit{serves Onna San no Miya}), 49;
		      Ben (\textit{Shōnagon's daughter; serves Murasaki}), 9;
		      Ben (\textit{serves Fujitsubo}), 10;
		      Ben (\textit{serves Ōigimi and Naka no Kimi, then a nun}), 45, 46, 47, (as a nun) 48, 49, 50, 51, 52;
		      Ben (\textit{serves Empress}), 52;
		      Chūjō (\textit{serves Utsusemi}), 2;
		      Chūjō (\textit{serves Rokujō no Miyasudokoro}), 4;
		      Chūjō (\textit{serves Asagao}), 10;
		      Chūjō (\textit{serves Genji, then Murasaki, then Genji}), 12, 19, 23, 41;
		      Chūjō (\textit{serves Higekuro's first wife}), 31;
		      Chūjō (\textit{serves Tamakazura's elder daughter}), 44;
		      Chūjō (\textit{serves Onna Ichi no Miya}), 52;
		      Chūnagon (\textit{serves Sadaijin}), 2, 9, 12;
		      Chūnagon (\textit{serves Oborozukiyo}), 10, 12, 34;
		      Chūnagon (\textit{serves Kokiden no Nyōgo II}), 26;
		      Dainagon (\textit{serves Onna Ichi no Miya}), 52;
		      Gosechi (\textit{serves Ōmi no Kimi}), 26;
		      Inuki (\textit{serves Murasaki}), 5;
		      Jijū (\textit{serves Suetsumuhana, also her foster sister}), 6, 15;
		      Jijū (\textit{serves Ukifune}), 50, 51, 52;
		      Kojijō (\textit{serves Onna San no Miya, also her foster sister}), 34, 35, 36;
		      Komoki (\textit{serves Ukifune}), 53;
		      Koshōshō (\textit{serves Ochiba no Miya}), 39;
		      Kozaishō (\textit{Kaoru's lover, serves Onna Ichi no Miya}), 52, 53;
		      Miruko (\textit{serves Tamakazura}), 24;
		      Miya no Kimi (\textit{serves Empress}), 52;
		      Moku (\textit{serves Higekuro}), 31;
		      Nakatsukasa (\textit{serves Sadaijin}), 2, 6;
		      Nakatsukasa (\textit{serves Genji}), 12, 35;
		      Nareki (\textit{serves Tamakazura's elder daughter}), 44;
		      Nurse (\textit{Akashi no Himegimi's}), 14, 18, 19;
		      Nurse (\textit{Yūgao's}), 22;
		      Nurse (\textit{Onna San no Miya's}), 34;
		      Nurse (\textit{Ukifune's}), 50, 51, 52, 53;
		      Omyōbu (\textit{serves Fujitsubo}), 5, 7, 10, 12;
		      Saemon (\textit{serves Yokawa no Sōzu's sister, as a nun}), 53;
		      Saishō (\textit{Yūgiri's nurse}), 12, 21, 33;
		      Saishō (\textit{serves Tamakazura}), 24, 30, 44;
		      Sanjō (\textit{serves Tamakazura}), 22;
		      Senji (\textit{serves Asagao}), 20, 21;
		      Shōnagon (\textit{serves Murasaki}), 4, 7, 9, 10, 12;
		      Shōshō (\textit{serves Suetsumuhana}), 15;
		      Shōshō (\textit{serves Naka no Kimi}), 49, 50, 51;
		      Shōshō (\textit{serves Yokawa no Sōzu's sister, as a nun}), 53;
		      Taifu (\textit{serves Suetsumuhana}), 6;
		      Taifu (\textit{Kumoi no Kari's nurse}), 33;
		      Taifu (\textit{serves Tamakazura's younger daughter}), 44;
		      Taifu (\textit{serves Naka no Kimi}), 48, 49, 50;
		      Ukon (\textit{Yūgao's nurse, then serves Murasaki}), 4, 22;
		      Ukon (\textit{serves Naka no Kimi, also Taifu's daughter}), 50, 51;
		      Ukon (\textit{serves Ukifune at Uji, a daughter of Ukifune's nurse}), 52;
		      Yugei no Myōbu (\textit{serves Kiritsubo no Mikado}), 1

		\item \textbf{Gosechi}, a young woman favored by Genji The Gosechi Dancer, 12, 13, 14

		\item \textbf{Hachi no Miya}, Genji and Suzaku's half brother His Highness, the Eighth Prince, 45, 46

		\item \textbf{Hanachirusato} Younger sister of the Reikeiden Consort, 11;
		      the lady of the falling flowers, 12, 14;
		      the lady of the village of falling flowers, 13;
		      the lady of Falling Flowers, 18, 40;
		      the lady in Genji's east pavilion, 19;
		      the lady of the northeast quarter of Rokujō, the lady of summer, 22;
		      the lady in/of the northeast quarter, 23, 25, 28, 39;
		      the lady of summer, mistress of the northeast quarter, 35;
		      the lady of summer, 41

		\item \textbf{Higekuro}, Tamakazura's husband The Commander (\textit{of the Right}), 24, 29, 30, 31;
		      the Left Commander, then Minster of the Right, 35

		\item \textbf{Hitachi no Kami}, Chūjō no Kimi's husband, Ukifune's stepfather The Governor of Hitachi, 50, 52

		\item \textbf{Hotaru}, Genji's half brother His Highness, the Viceroy Prince (\textit{Sochi no Miya}), 12, 17;
		      His Highness of War (\textit{Hyōbukyō no Miya}), 21, 24, 25, 29, 30, 31, 32, 34, 35, 38, 41

		\item \textbf{Hyōbukyō no Miya}, Fujitsubo's elder brother, Murasaki's father; becomes Shikibukyō no Miya His Highness of War, His Highness, 1, 5, 7, 10, 12, 14, 17;
		      the Lord of Ceremonial, 21;
		      His Highness of Ceremonial, 31, 34, 35

		\item \textbf{Hyōtōda}, Dazai no Shōni's eldest son The Bungo Deputy, 22

		\item \textbf{Ichijō no Miyasudokoro}, Ochiba no Miya's mother The Haven, 35, 36, 37, 39

		\item \textbf{Iyo no Suke}, father of Ki no Kami I, Utsusemi's husband, later Hitachi no Suke The Iyo Deputy, 2, 4;
		      the Deputy Governor of Hitachi, 16

		\item \textbf{Kaoru}, son of Onna San no Miya and Kashiwagi (\textit{Born}), 36;
		      the little boy, 37;
		      the Consultant Captain, 42;
		      the Minamoto Counselor, 43;
		      the Minamoto Adviser, then Consultant Captain, then Counselor, 44;
		      the Consultant Captain, 45;
		      the Consultant Captain, then Counselor, 46;
		      the Counselor, 47, 48;
		      the Counselor, then the Commander, 49;
		      the Commander, 50, 51, 52, 53, 54

		\item \textbf{Kashiwagi}, Tō no Chūjō's eldest son The Captain, 24, 29;
		      the Right Captain, 25, 26;
		      the Secretary Captain, 27, 30, 31, 32, 33;
		      the Intendant of the Right Gate Watch, 34;
		      the Intendant of the Right Gate Watch, then also Counselor, 35;
		      the Intendant of the Right Gate Watch, then Acting Grand Counselor, 36 (\textit{as a phantom}), 37

		\item \textbf{Kinjō}, ``the reigning Emperor'' (\textit{unnamed}) after Reizei, son of Suzaku and the Shōkyōden Consort The Shōkyōden Prince, named Heir Apparent, 14;
		      His Highness, the Heir Apparent, 32, 34;
		      His Highness, the Heir Apparent, then His Majesty, the Emperor, 35;
		      His Majesty, 36, 42, 44, 45, 49

		\item \textbf{Ki no Kami I}, the Governor of Kii, then of Kawachi; Utsusemi's stepson The Governor of Kii 2;
		      the Governor of Kawachi, 16

		\item \textbf{Ki no Kami II}, grandson of Yokawa no Sōzu's mother The Governor of Kii, 53

		\item \textbf{Kiritsubo no Kōi}, Genji's mother The Kiritsubo Intimate, the Haven, 1

		\item \textbf{Kiritsubo no Mikado}, the Kiritsubo Emperor, Genji's father His Majesty, 1, 2, 7, 8;
		      His Eminence (\textit{the Retired Emperor}), 9;
		      His Eminence, His Late Eminence, 10;
		      His Late Eminence, 12, 13, 14

		\item \textbf{Kōbai}, Tō no Chūjō's second son 10, 14;
		      the Controller Lieutenant, 23, 26, 27, 29, 32, 33;
		      the Inspector Grand Counselor, 43, 49;
		      the Grand Counselor, then Minister of the Right, 44

		\item \textbf{Kogimi I}, Utsusemi's younger brother 2, 3;
		      the Second of the Right Gate Watch, 16

		\item \textbf{Kogimi II}, Ukifune's younger half brother 54

		\item \textbf{Kokiden no Nyōgo I}, Udaijin's daughter, Suzaku's mother The Kokiden Consort, 1, 7, 8, 9, 10;
		      the Empress Mother, 13, 14;
		      Her Majesty, the Empress Mother, 21

		\item \textbf{Kokiden no Nyōgo II}, Tō no Chūj's daughter, Reizei's Consort The (\textit{Kokiden}) Consort, 14, 17, 21, 26, 29, 44

		\item \textbf{Koremitsu}, Genji's foster brother and confidant Koremitsu, 4, 5, 8, 9, 11, 14, 15, 18;
		      the Commissioner of Civil Affairs, 12;
		      the Governor of Tsu and Left City Commissioner, 21

		\item \textbf{Kumoi no Kari}, Tō no Chūjōs daughter, later Yūgiri's wife 21, 26, 28, 32;
		      the young lady, 33;
		      the Commanders wife, 37, 39, 44

		\item \textbf{Kurōdo no Shōshō}, son of Yūgiri and Kumoi no Kari The Chamberlain Lieutenant, then Third Rank Captain, then Consultant, 44

		\item \textbf{Makibashira}, Higekuro's daughter, later Kōbai's wife 31;
		      the daughter of the Left Commander, 35;
		      Kōbai's wife, 43, 44

		\item \textbf{Makibashira's daughter}, by Hotaru Her Highness, 43

		\item \textbf{Michisada}, Niou's retainer The Chief Clerk and Deputy Commissioner of Ceremonial, 51

		\item \textbf{Murasaki} A little girl, 5, 6;
		      Genji's young lady, 7, 8, 9;
		      the mistress of Genji's west wing, 10, 12, 21;
		      Genji's lady at Nijō, 13, 14;
		      Genji's lady, 18, 22, 23, 25, 29;
		      the lady in Genji's west wing, 19, 20;
		      Genji's love, 23;
		      the mistress of the southeast quarter, 24, 28, 29, 32;
		      the lady of spring, 31;
		      the mistress of Genji's east wing, 33;
		      the mistress of the east wing, 34, 35;
		      Lady Murasaki, 35, 39, 40

		\item \textbf{Naishi no Suke}, Koremitsu's daughter A Gosechi dancer, 21;
		      the Fujiwara Dame of Staff, 33;
		      the Dame of Staff, 39

		\item \textbf{Nakanobu}, Kaoru's retainer, Michisada's father-in-law Nakanobu, the Treasury Commissioner, 51, 52

		\item \textbf{Naka no Kimi}, second daughter of Hachi no Miya, eventually wife of Niou 45, 46, 47, 48;
		      the lady in the wing at Nijō, 49;
		      the wife of His Highness of War, 50;
		      Her Highness, the wife of His Highness of War, 51, 52

		\item \textbf{Niou}, Akashi no Himegimi's son, Murasaki's favorite, Naka no Kimi's husband, Ukifune's lover His Highness, the Third Prince, 37, 40, 41;
		      His Highness of the Bureau of War, His Highness of War, 42, 43, 45, 46, 47, 48, 49, 50, 51, 52

		\item \textbf{Nokiba no Ogi}, Ki no Kami's sister, Iyo no Suke's daughter The lady from the west wing, 3;
		      the daughter of the Iyo Deputy, 4

		\item \textbf{Oborozukiyo}, Udaijin's sixth daughter, sister of Kokiden no Nyōgo I The lady of the misty moon, 8;
		      the Mistress of the Wardrobe, 9;
		      the Mistress of the Wardrobe, then the Mistress of Staff, 10;
		      the Mistress of Staff, 12, 14, 21, 34;
		      the Nijō Mistress of Staff, 35

		\item \textbf{Ochiba no Miya}, Suzaku's second daughter, Roku no Kimi's adoptive mother Her Highness, the Second Princess, 35, 36, 37;
		      Her Highness at/of Ichijō, 39, 42;
		      Her Highness, 49
		      
		\item \textbf{Ōigimi}, Hachi no Miya's elder daughter 45, 46;
		      Her Highness, 47
		      
		\item \textbf{Ōmi no Kimi}, Tō no Chūjō's lost daughter 26; 
		      the girl from Omi, 29, 31

		\item \textbf{Omiya}, Kiritsubo no Mikado's sister, Sadaijin's wife, Aoi and Tō no Chūjō's mother, Yūgiri and Kumoi no Kari's grandmother The Princess, 1;
		      Her Highness, 9, 12, 28, 29;
		      Her Highness, the Third Princess, 21

		\item \textbf{Onna Go no Miya}, Asagao's aunt Her Highness, the Fifth Princess, 20, 21

		\item \textbf{Onna Ichi no Miya}, daughter of Reizei and the Kokiden Consort II His Eminence's daughter, 42;
		      the First Princess, 52

		\item \textbf{Onna Ni no Miya}, Kaoru's wife, daughter of Kinjō and the Fujitsubo Consort Her Highness, the Second Princess, 49, 52

		\item \textbf{Onna San no Miya}, Suzaku's favorite daughter, Genji's wife, Kaoru's mother Her Highness, the Third Princess, 34, 35;
		      Her Highness, the Third Princess, Her Cloisered Highness, 36;
		      Her Cloistered Highness, the Third Princess, 37, 38, 41, 42, 45, 49

		\item \textbf{Reizei}, Emperor, son of Genji and Fujitsubo (\textit{Born}), 7;
		      the Heir Apparent, 9, 10, 12, 13;
		      His Highness, the Heir Apparent, then His Majesty, 14;
		      His Majesty, 17, 18, 19, 21, 29, 31, 33, 34;
		      His Majesty, then His Eminence, Retired Emperor, 35;
		      His Eminence, 38, 44, 45

		\item \textbf{Risshi His Reverence}, the Master of Discipline, 39

		\item \textbf{Rokujō no Miyasudokoro}, widow of a former Heir Apparent The Rokujō Haven, 4, 9, 10, 12, 14; (\textit{as a spirit}), 35, 36

		\item \textbf{Roku no Kimi}, Yūgiri's sixth daughter (\textit{by Koremitsu's daughter}), later Niou's wife His Excellency's sixth daughter, 42, 49

		\item \textbf{Sachūben}, Onna San no Miya's retainer, also serves Genji The Left Controller, 34

		\item \textbf{Sadaijin}, the Minister of the Left (\textit{resigns}), then Chancellor; Aoi's father His Excellency, 1, 2, 5, 6, 7, 8, 9, 10; (resigns), 12, 14;
		      becomes Chancellor, 19

		\item \textbf{Sakon no Shōshō}, marries Hitachi no Kami's favorite daughter The Lieutenant of the Left Palace Guards, 50

		\item \textbf{Shikibukyō no Miya}, Asagao's father His Highness of Ceremonial, 9, 19

		\item \textbf{Shi no Kimi}, Tō no Chūjō's wife, Kashiwagi's mother 4, 35, 36

		\item \textbf{Suetsumuhana}, daughter of the Hitachi Prince Her Highness, daughter of His Highness of Hitachi, 6, 15;
		      the safflower, 22;
		      Her Highness of Hitachi, 23

		\item \textbf{Suetsumuhana's aunt} 15

		\item \textbf{Suzaku}, Emperor, then Retired Emperor; Kiritsubo no Mikado's eldest son Appointed Heir Apparent, 1;
		      the Heir Apparent, 7, 8;
		      His Majesty, 10, 12, 13;
		      His Majesty, the Emperor, then His Eminence, 14;
		      His Eminence, 17, 21, 33, 34;
		      His Cloistered Eminence, 35, 36, 37, 39

		\item \textbf{Taifu no Gen}, pursues Tamakazura in Tsukushi The Audit Commissioner, 22

		\item \textbf{Tamakazura}, Yūgao and Tō no Chūjō's daughter, later Higekuro's wife The pink, 4;
		      the young lady, 22;
		      the (\textit{young}) lady in the west wing, 23, 24, 25, 26, 27, 28, 29;
		      the lady, 30;
		      the Mistress of Staff, 31, 34, 35, 44

		\item \textbf{Tamakazura's daughters} The Haven (\textit{the elder}), the Mistress of Staff (the younger), 44

		\item \textbf{Tamakazura's sons} The Left Palace Guards Captain, the Right Controller, the Fujiwara Adviser, 44

		\item \textbf{Tokikata}, Niou's retainer The Deputy Governor of Izumo, 51, 52

		\item \textbf{Tō no Chūjō}, the Minister of the Left's eldest son, Aoi's brother The Chamberlain Lieutenant, 1;
		      the Secretary Captain, 2, 4, 5, 6, 7, 8;
		      the Third Rank Captain, 9;
		      the Captain, 9, 10;
		      the Captain, also appointed a Consultant, 12;
		      the Consultant Captain, then the Acting Counselor, 14;
		      the Acting Counselor, 17;
		      the Acting Counselor, then Grand Counselor and Commander of the Right, 19;
		      the Commander, then His Excellency, the Palace Minister, 21;
		      His Excellency, the Palace Minister, 25, 26, 28, 29, 30, 31, 32;
		      His Excellency, the Palace Minister, then Chancellor, 33;
		      His Excellency, the Chancellor, 34;
		      His Excellency, the Chancellor, then His Retired Excellency, 35;
		      His Retired Excellency, 36, 37;
		      His Excellency, 39, 40

		\item \textbf{Udaijin}, the Minister of the Right, then Chancellor; father of Kokiden no Nyōgo I, Suzaku's grandfather The Minister of the Right, 1, 9, 10;
		      the Minister of the Right, His Excellency, 8, 12;
		      the Chancellor, 13

		\item \textbf{Uji no Ajari}, Hachi no Miya's spiritual adviser The Adept, 45, 46, 47, 48, 49;
		      the Master of Discipline, 52

		\item \textbf{Ukifune}, Hachi no Miya's unrecognized daughter A young woman, 49, 50, 51, 53, 54

		\item \textbf{Utsusemi}, stepmother of Ki no Kami I, the Iyo Deputy's wife 2, 3;
		      the lady of the cicada shell, 4, 16;
		      the cicada shell, 23

		\item \textbf{Yamato no Kami}, Ochiba no Miya's cousin and Koshōshō's brother The Governor of Yamato, 39

		\item \textbf{Yoshikiyo}, Genji's retainer, son of the Governor of Harima 5;
		      Yoshikiyo, 8, 12;
		      the Minamoto Junior Counselor, 13;
		      the Governor of Ōmi, 21

		\item \textbf{Yokawa no Sōzu} His Reverence, the Prelate of Yokawa, 53, 54

		\item \textbf{Yokawa no Sōzu's} mother An old nun, 51, 53, 54

		\item \textbf{Yokawa no Sōzu's} sister A nun, 51, 53

		\item \textbf{Yūgao}, mother of Tamakazura 4

		\item \textbf{Yūggiri}, son of Genji and Aoi (\textit{Born}), 9, 12, 14;
		      a student at the Academy, the Adviser, 21;
		      the Captain, 22, 23, 24, 25, 26, 27, 28, 29;
		      the Consultant Captain, 30, 31, 32;
		      the Consultant Captain, then Counselor, 33;
		      the Counselor, then Right Commander, 34;
		      the Right Commander, then also Grand Counselor, 35;
		      the Right Commander, 36;
		      the Commander, 37, 38, 39, 40, 41;
		      His Excellency, the Minister of the Right, the Commander of the Left Palace Guards, 42;
		      His Excellency, the Minister of the Right, 43;
		      His Excellency, the Minister of the Right, then of the Left, 44;
		      His Excellency, the Minister of the Right, 47, 48, 49, 51

	\end{itemize}


	%\end{footnotesize}
	%\end{small}

\end{multicols}

%page break 
\clearpage

%%%%%%%%%%%%%%%%%%%%%%%%%%%%%%%%%%%%%%%%%%
%%%
%%%
%%%
%%%
%%%      Clothing and Colors
%%%
%%%
%%%
%%%
%%%%%%%%%%%%%%%%%%%%%%%%%%%%%%%%%%%%%%%%%%


%\section{Clothing and Color}%Don't really need this header, as it is obvious

\noindent \textbf{\textit{Note:}} Words for clothing are impossible to translate except in general terms that convey at best only vague impressions. The colors and color combinations (\textit{“layerings”}) listed here are more evocative, but they are not necessarily more precise. The range of colors current in the time of the tale was too wide to permit usefully precise translation, and in any case much about their names remains uncertain. Moreover, a good many terms for single colors refer more to the dye source (safflower, cloves, sappanwood, dayflower, gardenia seeds, and so on) than to the resulting hue, which in practice could vary widely. Therefore these color terms, too, are no more than distant approximations.


%\newcommand\Item[1]{\item\begin{minipage}[t]{\linewidth}#1\end{minipage}}

\setlength{\columnsep}{5em}
\begin{multicols*}{2}
	%\vspace*{-0.4cm}
	% normal size font gts us to 3 pages plus 10 lines...
	%\begin{small}% this gets us to ~2.75 pages
	%\begin{footnotesize}% this only gets us to 2.25 pages, use above
	\begin{itemize}[
			label=,
			leftmargin=0em,
			rightmargin=-1.5em,
			itemindent=-2em,
			nosep,
		]
		\setlength{\itemsep}{0.25em}

		%\Item{\textbf{apron} \textit{shibira, uwamo} --- Worn by a gentlewoman when serving her mistress.}%NB: the captial I of Item is a diff command, see comment above. minipage needs hangin indent, don't want to deal with right now.

		\item \textbf{apron} \textit{shibira, uwamo} --- Worn by a gentlewoman when serving her mistress.

		\item \textbf{ash green} \textit{aoji}.

		\item \textbf{aster layering} \textit{shion (\textit{kasane})} --- A layering of colors (perhaps pale gray-violet [usuiro] over blue or green [ao]) that gave an impression of violet-blue, like this flower.

		\item \textbf{autumn green layering} \textit{aokuchiba} --- Fabric woven of leaf green (\textit{ao}) warp and yellow weft threads, over leaf green.

		\item \textbf{azure} \textit{hanada} --- A medium, morning glory blue from indigo.

		\item \textbf{bead tree layering} \textit{ōchi} --- Possibly purple lined with a lighter shade of the same color.
		      Ōchi is the old name for the Japanese bead tree
		      (\textit{sendan, Melia azedarach, var.\ subtripinnata}),
		      which reaches about twenty-six feet in height and in spring bears light purple, five-petaled flowers.

		\item \textbf{beaten silk} \textit{uchimono} --- Silk beaten on a fulling block (\textit{kinuta}) to bring out its luster.

		\item \textbf{blue} \textit{koki hanada}.

		\item \textbf{blue-gray} \textit{aonibi} --- May also be visualized in the green range (\textit{gray-green}), since ao in practice covers both ranges.

		\item \textbf{Cathay tendril pattern} \textit{karakusa} --- A family of textile patterns consisting of arabesque-like leafy tendrils and sometimes flowers.

		\item \textbf{cherry blossom layering} \textit{sakura gasane} --- White over scarlet (\textit{kurenai}) or, if worn by a young man, violet (\textit{futaai}).

		\item \textbf{Chinese jacket} \textit{karaginu} --- A short jacket, longer in front than in back, that formed the outer-most layer of a woman's formal dress.

		\item \textbf{clove-dyed} \textit{chōjizome, kōzome} --- A warm tan.

		\item \textbf{court dress} \textit{nōshi sugata} --- The ordinary costume worn by a nobleman at the palace or when dressed up at home. The level of formality could be varied: with a formal cap (\textit{kanmuri}) it was more formal than with a hat (\textit{eboshi}). The dress cloak (\textit{nōshi}) was tied on and worn over a gown (\textit{uchiki, akome, onzo, kinu}) and shift (\textit{hitoe}), with gathered trousers (\textit{sashinuki}).

		\item \textbf{cover} \textit{kinu} --- A gown used for cover at night.

		\item \textbf{cypress bark} \textit{hiwada iro} --- The color of the bark of the Japanese cypress (\textit{hinoki}), a dark red-brown.

		\item \textbf{damask} \textit{aya} --- More properly figured twill. Twill weaving was originally Chinese.

		\item \textbf{dark blue} \textit{kon} --- A color associated in the tale with ruri (\textit{lapis lazuli or glass}).

		\item \textbf{dark gray} \textit{tsurubami}.

		\item \textbf{dayflower} \textit{tsuyukusa, tsukikusa} --- A fugitive blue dye from the sky blue flowers of the dayflower, a common wild plant in Japan.

		\item \textbf{deep blue} \textit{komayaka naru} --- A deep shade (\textit{komayaka}) of azure (\textit{hanada}).

		\item \textbf{deep blue-gray} \textit{koki aonibi} --- See blue-gray.

		\item \textbf{deep green} \textit{midori} --- A color range that actually extends from gray to blue-green and deep blue.

		\item \textbf{deep hat} \textit{tsubo sōzoku} --- The attire for a respectable woman outdoors. She draped an unlined gown over her head and hair, then put on a deep, broad-brimmed hat. She also hitched up her skirts a little for walking.

		\item \textbf{deep red-violet} \textit{koki iro}.

		\item \textbf{deep scarlet layering} \textit{koki hitokasane} --- Two very dark scarlet shifts, one over the other.

		\item \textbf{deutzia layering} \textit{unohana} --- White over grass green (\textit{moegi}). The deutzia flowers in the fourth lunar month, in long clusters of small white blossoms.

		\item \textbf{dress cloak} \textit{nōshi} --- The outer garment ordinarily worn by a courtier at court or fully dressed at home. The dress cloak and the formal cloak (\textit{hō}) were ample, straight garments tied at the neck and with the front and back joined at the hem by a circular band of cloth (\textit{ran}), pressed flat. However, the color of a dress cloak was not determined by the wearer's rank. It could be made of a single layer of sheer, dark cloth in summer and of light-colored, lined cloth in winter. It was generally worn with gathered trousers (\textit{sashinuki}).

		\item \textbf{dress gown} \textit{kouchiki} --- A gown of fancy stuff of the same shape as a gown (\textit{uchiki}) but somewhat shorter, worn by a woman at home when some formality was desired. Also kazami --- A long garment worn on top, especially by page girls in formal dress.

		\item \textbf{earth green} \textit{aoni}.

		\item \textbf{fallen chestnut} \textit{ochiguri} --- Thought to be a deep, reddish brown.

		\item \textbf{formal cap} \textit{kanmuri} --- A small cap with various attachments worn with full civil dress and with more formal court dress.

		\item \textbf{formal cloak} \textit{hō} --- The outer garment worn by men on official business and when participating in court ceremonies. The formal cloak was not a layering (\textit{kasane}) but was made either of an opaque cloth with a figure worked into it in the same or nearly the same color, or of a single layer of sheer cloth. Its color matched the wearer's rank in the time of the tale.

		\item \textbf{formal dress} --- The full-dress costume worn by women at court or by gentlewomen in an aristocratic household. The Chinese jacket (\textit{karaginu}) was worn over a train (\textit{mo}) tied at the waist over an outer gown (\textit{uwagi}) that was the most elaborate of a layer of gowns of identical shape (\textit{uchiki, kinu, onzo}) worn over a shift (\textit{hitoe}) and long, ample trousers (\textit{hakama}) tied at the waist with a sash. The layers of gowns were cut smaller as they reached the outside so that the edges of the underlayers could be seen.

		\item \textbf{full civil dress} \textit{sokutai} --- The costume worn by noblemen on official business and when participating in court ceremonies. The formal cloak (\textit{hō}), in the color appropriate to the wearer's rank, was worn with a sword and a stone belt (\textit{sekitai}) likewise matched to the wearer's rank. It was worn over a train-robe (\textit{shitagasane}), which was a midthigh-length garment with a train (\textit{kyo}), worn in turn over gowns (\textit{akome, onzo, kinu, uchiki}) and a shift (\textit{hitoe}), with two pair of wide, open-legged trousers (\textit{hakama}). The costume was completed by the formal cap (\textit{kanmuri}) and baton (\textit{shaku}). At least in later times, for somewhat less formal occasions, including ceremonies at home, the formality of this costume could be lowered by wearing gathered trousers (\textit{sashinuki}) instead of trousers (\textit{hakama}) and lowered still further (\textit{ikan}) by omitting the belt and sword and using a narrow sash instead, and by carrying a fan instead of a baton.

		\item \textbf{gathered trousers} \textit{sashinuki} --- Ample trousers gathered around the ankles and worn with a dress cloak (\textit{nōshi}) or a hunting cloak (\textit{kariginu}).

		\item \textbf{golden yellow} \textit{yamabuki} --- The color of kerria rose flowers.

		\item \textbf{gossamer silk} \textit{suzushi} --- A thin, raw silk for an unlined garment.

		\item \textbf{gown} (\textit{1}) \textit{uchiki, onzo, kinu} --- Any woman's gown worn, often in several layers, between dress gown (\textit{uwagi}) and shift (\textit{hitoe}). (2) \textit{akome, onzo, kinu, uchiki} --- A man's robe worn over the shift (\textit{hitoe}) but under the train-robe (\textit{shitagasane}) or the dress cloak. (3) \textit{akome} --- A garment worn by young girls between dress gown and shift. However, \textit{akome} is sometimes translated “jacket” because page girls could wear their \textit{akome} on top.

		\item \textbf{grape} (\textit{colored}) \textit{ebi, ebizome} --- Grape purple to reddish brown.

		\item \textbf{grape layering} \textit{ebizome kasane} --- A winter layering of sappan (\textit{suō}) over azure (\textit{hanada}).

		\item \textbf{grass green} \textit{moegi}.

		\item \textbf{gray} \textit{nibiiro} (\textit{light gray, \textit{usunibi} dark gray, \textit{tsurubami}}).

		\item \textbf{green} \textit{asamidori}.

		\item \textbf{hail pattern} \textit{arare-ji} --- A check pattern of dark and light squares. Also called ishidatami, “paving stones.”

		\item \textbf{hat} \textit{eboshi} --- A tall hat worn with court dress (\textit{nōshi sugata}) and with informal dress (\textit{kariginu sugata}).

		\item \textbf{hunting cloak} \textit{kari no onzo, kariginu} --- An outer garment originally worn for hunting and then adopted by the nobles as everyday informal wear. The hunting cloak had cords laced through the sleeves to allow them to be gathered at the wrist, and it had sleeves semidetached from the body of the garment, for ease of movement. It was worn with gathered trousers (\textit{sashinuki}) or hunting trousers (\textit{karibakama}) and a hat (\textit{eboshi}). A gentleman normally wore a hunting cloak while traveling, and his attendants would wear it even when he was in court dress (\textit{nōshi sugata}). He also might wear it as a disguise.

		\item \textbf{indigo} \textit{ai} --- A plant that yields a blue dye.

		\item \textbf{jacket} \textit{akome} --- A type of gown worn as an outer garment by page girls.

		\item \textbf{kerria rose} \textit{yamabuki} --- The color, also listed as “golden yellow,” obtained from gardenia-seed dye.

		\item \textbf{kerria rose layering} \textit{yamabuki kasane} --- Ocher (\textit{kuchiba}) over yellow (\textit{kuchinashi}).

		\item \textbf{layering} \textit{kasane} --- Usually a combination of two garments of different colors, or of garment and lining, so that one color could be seen through the other. Layerings had names, but the colors in them changed over time, and they are often uncertain. The name often refers to the overall effect. For example, “cherry blossom,” white over scarlet, produced a pale cherry blossom pink. Many layerings were seasonal and were also affected by the wearer's age and rank.

		\item \textbf{leaf gold} \textit{kanzō} iro.

		\item \textbf{leaf green} \textit{ao, ao-iro} --- A light leaf green veering toward yellow. The color also known as kikujin (\textit{although the word does not appear in the tale}) and favored by the Emperor for daily wear.

		\item \textbf{light blue} \textit{asagi}.

		\item \textbf{light gray} \textit{usunibi, usuki nibi}.

		\item \textbf{light russet} \textit{akakuchiba}.

		\item \textbf{light silk twill} \textit{ki}.

		\item \textbf{long dress} \textit{hosonaga} --- A lady's outer garment, divided in front and with long panels that trailed behind on either side.

		\item \textbf{madder red} \textit{hiiro} --- The color dyed with akane.

		\item \textbf{maidenflower layering} \textit{ominaeshi (\textit{kasane})} --- A layering that recalls the valerian family flower of that name. The outer layer has a leaf green (\textit{ao}) warp and a yellow (\textit{ki}) weft, and the underlayer is leaf green.

		\item \textbf{mauve-gray (\textit{paper})} \textit{murasaki no nibameru (\textit{kami})}.

		\item \textbf{mourning weeds} \textit{fujigoromo} --- Mourning robes figuratively made (\textit{as perhaps they really were in ancient times}) from the bark of wild fuji (wisteria) vines.

		\item \textbf{\textit{murasaki}} --- A plant (\textit{Lithospermum erythrorhizon}) the roots of which yield a purple dye; also, the dye and its color. The color stands for relationship and lasting passion.

		\item \textbf{night service wear} \textit{tonoi sugata} --- A simple costume, perhaps just a white gown, worn by a page girl on night attendance.

		\item \textbf{ocher} \textit{kuchiba} --- The Japanese word means “dead leaves.”

		\item \textbf{pale gray-violet} \textit{usuiro}.

		\item \textbf{petal blue} \textit{hana (\textit{iro})} --- A pale indigo.

		\item \textbf{pink layering} \textit{nadeshiko (\textit{kasane})} --- A layering that suggests the pink, or gillyflower. According to some authorities, dark pink over light purple; others cite plum pink (\textit{kōbai}) over leaf green (\textit{ao}).

		\item \textbf{plum pink} \textit{kōbai} --- A pink veering toward violet, reminiscent of plum blossoms.

		\item \textbf{plum red} \textit{imayō}.

		\item \textbf{pure raiment} \textit{jōe} --- Robes worn by priests during a rite; blue-black, yellow, red, white, gray, or brown, depending on the deity to whom the rite was addressed.

		\item \textbf{purple} \textit{murasaki} --- The color from the roots of the murasaki plant, a common field plant with white flowers. In poetry murasaki stands for close relationship. The color figures prominently in the tale as the color of enduring love.

		\item \textbf{red} \textit{aka}.

		\item \textbf{red plum blossom layering} \textit{kōbai kasane} --- Scarlet (\textit{kurenai}) over purple (\textit{murasaki}) or sappan (\textit{suō}).

		\item \textbf{rouge red} \textit{beniiro} --- A scarlet color produced with safflower (\textit{benibana}) dye, the source of scarlet (\textit{kurenai}).

		\item \textbf{sanctioned rose} \textit{yurushiiro} --- A pale scarlet (\textit{kurenai}). The counterpart “forbidden color” (\textit{kinjiki}), not mentioned in the tale, was a deep shade of dark red or purple allowed only to the Emperor and the senior nobles.

		\item \textbf{sappan layering} \textit{suō gasane} --- A winter layering of sappan over dark sappan (\textit{a dull reddish purple}).

		\item \textbf{sappan (\textit{wood})} \textit{suō} --- A red from the wood of the sappan tree, imported from Southeast Asia.

		\item \textbf{sash} \textit{obi}.

		\item \textbf{scarlet} \textit{kurenai} --- The color from the carthame dye, a fugitive red lake made from safflowers (\textit{Carthamus tinctorius}). The dye has both red and yellow components, but with some difficulty the yellow can be eliminated. The pigment from the flowers can also be used to produce a makeup rouge.

		\item \textbf{scarlet layering} \textit{kaineri kasane} --- Scarlet (\textit{kurenai}) over scarlet.

		\item \textbf{seaside} \textit{kaifu} --- A textile pattern showing waves, beachside pines, seaweed, shells, and so on.

		\item \textbf{service dress} \textit{tonoi sugata} --- The costume worn by men in regular service at the palace, characterized especially by the dress cloak (\textit{nōshi}).

		\item \textbf{shift} \textit{hitoe} --- An unlined garment worn by men or women under the outer layers of clothing and cut larger. Like all other garments listed in this glossary, it was open in front, like a jacket, so that “shift” is only an expedient approximation.

		\item \textbf{shoulder cords} \textit{obi, or, more properly, kakeobi} --- Red silk cords passed over each shoulder and tied together in the back, worn by a woman on pilgrimage or when performing religious devotions.

		\item \textbf{silk gauze} \textit{usumono} --- Very thinly woven silk, used especially for summer wear.

		\item \textbf{sky blue} \textit{asahanada}.

		\item \textbf{softened silk} \textit{kaineri} --- Silk boiled with lye to soften it.

		\item \textbf{spring green layering} \textit{wakanae kasane} --- Light grass green (\textit{moegi}) over light grass green.

		\item \textbf{stone belt} \textit{sekitai, shakutai} --- A broad black leather belt worn with the formal cloak (\textit{hō}), with a row of squares or circles made of stone, jade, or horn set in it so as to show at the wearer's back. (A fold of the hō covered the front.) The “stone” varied according to the wearer's rank.

		\item \textbf{sunshade band} \textit{hikage} --- A band of dangling white or blue-green braided threads (\textit{originally club moss fronds}) worn by a Gosechi dancer as well as by other women, such as the Kamo Priestess, engaged in certain sacred functions.

		\item \textbf{sweet-flag layering} \textit{ayame gasane} --- Probably green over plum pink.

		\item \textbf{tail} \textit{ei} --- A long, narrow, springy appendage to a man's court cap, made of lacquered cloth. Usually straight, but rolled as a sign of mourning.

		\item \textbf{tan} \textit{kurumi iro, kō-iro} --- A yellowish tan from cloves.

		\item \textbf{train} \textit{kyo, mo} --- The man's train (\textit{kyo}) was a long, rectangular piece of cloth extending from the train-robe (\textit{shitagasane}) worn with full civil dress (\textit{sokutai}). The woman's train (\textit{mo}) was a long, sheer, decorated piece of cloth pleated into a sash tied at the front, at the waist, over the gown (\textit{uchiki, uwagi}), and under the Chinese jacket (\textit{karaginu}). A woman wore a train in service or on formal occasions to indicate a subservient position.

		\item \textbf{train-robe} \textit{shitagasane} --- The man's garment worn in full civil dress (\textit{sokutai}) over the mid-robe and under the formal cloak (\textit{hō}); it was of midthigh length and had a train (\textit{kyo}).

		\item \textbf{trousers} \textit{hakama, nagabakama} --- Men wore two pair of wide-legged, ankle-length trousers (\textit{an inner one and an outer ue no hakama}) with full civil dress, and a different sort of under-trousers under gathered trousers (\textit{sashinuki}) in less formal costume. Women wore long trousers (\textit{nagabakama}), with legs that extended well beyond their feet.

		\item \textbf{twill} \textit{aya} --- See damask.

		\item \textbf{violet} \textit{futaai} --- A “double-dyed” (\textit{futaai}) color produced by dyeing cloth in safflower (the source of scarlet, kurenai) dye and then in indigo (\textit{ai}). The actual hue varied. Brighter, deeper shades with little blue were worn by young men, while duller, paler tints with little red were worn by older men.

		\item \textbf{violet-blue} \textit{shion} --- A color reminiscent of the aster (\textit{shion}).

		\item \textbf{white layering} \textit{shiragasane} --- White over white.

		\item \textbf{wild indigo} \textit{yamaai} --- The green color derived from the wild indigo plant.

		\item \textbf{willow} \textit{yanagi} --- White weft and pale green warp threads.

		\item \textbf{willow layering} \textit{yanagi kasane} --- White over green.

		\item \textbf{wisteria layering} \textit{fuji gasane} --- Violet over green.

		\item \textbf{yellow} \textit{ki}, a general term; or \textit{kuchinashi} --- A yellow from gardenia-seed dye. The color is that of kerria rose (\textit{yamabuki}) flowers. (The color specified in the text as yamabuki is translated “golden yellow.”)

		\item \textbf{young pink leaves} \textit{nadeshiko no wakaba no iro} --- A light yellow-green (\textit{usumoegi}).




	\end{itemize}


	%\end{footnotesize}
	%\end{small}

\end{multicols*}

%page break 
\clearpage

%%%%%%%%%%%%%%%%%%%%%%%%%%%%%%%%%%%%%%%%%%
%%%
%%%
%%%
%%%
%%%      Titles, with associated characters
%%%
%%%
%%%
%%%
%%%%%%%%%%%%%%%%%%%%%%%%%%%%%%%%%%%%%%%%%%

%\section{Characters in The Tale of Genji}%Don't really need this header, as it is obvious

\noindent \textbf{\textit{Note:}} This glossary lists all the official or customary titles that appear in \textit{The Tale of Genji}, explains their meaning, and in most cases indicates the numbered rank corresponding to the office in question.

\setlength{\columnsep}{4.5em}
\begin{multicols}{2}

	%\vspace*{-0.4cm}
	% normal size font gts us to 3
	\begin{small}% this gets us to 
		%\begin{footnotesize}% this only gets us to 
		\begin{itemize}[
				label=,
				leftmargin=0em,
				rightmargin=-1.5em,
				itemindent=-2em,
				%nosep,
			]
			\setlength{\itemsep}{0.075em}

			\item \textbf{Abbot (\textit{of the Mountain})} (Yama no) Zasu --- The superior of the entire Mount Hiei temple complex.

			\item \textbf{Acting \(\ldots\)} Gon\(\ldots\)  --- A prefix to the title of certain male officials, indicating that the appointment is in excess of the normal number of incumbents in that post.

			\item \textbf{Acting Captain} Gon Chūjō.

			\item \textbf{Acting Grand Counselor} Gon Dainagon.

			\item \textbf{Adept} Azari, Ajari --- An imperially conferred title held by a distinguished practitioner monk, one expert in healing and other rituals to avert illness and disaster and summon good fortune.

			\item \textbf{Adviser} Jijū --- A junior official (\textit{junior fifth rank, lower grade}) under the Bureau of Central Affairs, who acted as an assistant to the Emperor. There were generally eight of them.

			\item \textbf{Adviser Consultant} Jijū no Saishō --- A dual appointment as Adviser and Consultant.

			\item \textbf{Aide} Zō, Jō --- A third-level appointee in the bureaus and in some guards units (\textit{Gate Watch, Watch}), but a fourth-level officer in the Palace Guards.

			\item \textbf{Aide of Ceremonial} Shikibu no Jō --- A third-level post in the Bureau of Ceremonial (\textit{sixth-rank range}).

			\item \textbf{Aide of the Gate Watch} Yugei no Jō (\textit{junior sixth rank}).

			\item \textbf{Aide of the Right Palace Guards} Ukon no Zō (\textit{sixth rank, lower grade}).

			\item \textbf{Aide of the Watch} Hyōe no Zō (\textit{seventh rank}).

			\item \textbf{Audit Commissioner} Taifu no Gen --- Auditor (\textit{Gen}) was a post in a provincial administration, responsible for discovering and correcting various irregularities. The most senior incumbent could attain the fifth rank, lower grade, in which case “Commissioner” (Taifu) was added to the title.

			\item \textbf{Bath Nurse} Mukaeyu --- The “assistant” role in the bathing of a newborn child of high birth.

			\item \textbf{Bungo Deputy} Bungo no Suke --- In principle, Deputy Governor of Bungo (\textit{now Ōita Prefecture}), a post in the sixth-rank range. However, it is unclear just what weight this title has where it occurs in the tale (“The Tendril Wreath”).

			\item \textbf{Bureau of Central Affairs} Nakatsukasa Shō --- The bureau that administered the palace. The senior among the eight major government bureaus, it was always headed by a Prince.

			\item \textbf{Bureau of Ceremonial} Shikibu Shō --- One of the eight major government bureaus, in charge of ceremonies, appointments, and awards.

			\item \textbf{Bureau of Civil Affairs} Minbu ShōOne --- of the eight major government bureaus, in charge of population registers, corvée, and taxation.

			\item \textbf{Bureau of the Treasury} Ōkura Shō --- One of the eight major government bureaus, in charge of managing the tax goods collected from the provinces.

			\item \textbf{Bureau of War} Hyōbu Shō --- One of the eight major government bureaus, in charge of military affairs and equipment.

			\item \textbf{Captain} Chūjō --- The second-level officer in the Palace Guards of Left or Right (\textit{junior fourth rank, lower grade}).

			\item \textbf{Captain of the Left Palace Guards} Sakon no Chūjō.

			\item \textbf{Chamberlain} Kurōdo --- An official of fifth or sixth rank, responsible to the Chamberlains' Office (\textit{Kurōdodokoro}) and under the supervision of two Secretaries (Kurōdo no Tō) of somewhat higher rank. A Chamberlain was admitted to the privy chamber and had direct access to the Emperor; he was also allowed to wear colors and fabrics normally forbidden to a man of his rank.

			\item \textbf{Chamberlain Aide of the Left Gate Watch} Kurōdo no Saemon no Jō --- A dual appointment as a Chamberlain and either a third-level (\textit{Daijō}) or fourth-level (Shōjō) officer in the Left Gate Watch.

			\item \textbf{Chamberlain Aide of the Right Palace Guards} Ukon no Zō no Kurōdo --- A dual appointment as a Chamberlain and a fourth-level officer (\textit{sixth rank, upper grade}) in the Right Palace Guards.

			\item \textbf{Chamberlain Controller} Kurōdo no Ben --- A dual appointment as a Chamberlain and a Controller.

			\item \textbf{Chamberlain Lieutenant} Kurōdo no Shōshō --- A dual appointment as a Chamberlain and a Lieutenant in the Palace Guards.

			\item \textbf{Chamberlain Second of the Watch} Kurodo Hyoe no Suke --- A dual appointment as a Chamberlain and a Second of the Watch.

			\item \textbf{Chamberlains' Office} Kurōdodokoro --- An office that functioned as an imperial secretariat, serving the Emperor and carrying messages and imperial orders. It was staffed by two higher-ranking Secretaries (\textit{Kurōdo no Tō}), by Chamberlains (Kurōdo) of the fifth and sixth ranks, and by a number of lesser figures. A Chamberlain moved in circles above his official rank, served the Emperor directly, and held privileges such as the right to wear colors normally not allowed a man of his rank. The Chamberlains' Office also looked after the Emperor's falcons and took care of musical instruments, books, coins (metal currency), and clothing.

			\item \textbf{Chancellor} Okiotodo, Daijōdaijin --- The highest possible civil post (\textit{first rank or junior first rank}), one not provided for in the government's nominal table of organization. In theory it was filled only by a candidate able to serve as an example of virtue, and the incumbent was to be above actual administration.

			\item \textbf{chaplain} \textit{inori no shi} --- The monk who regularly performed prayer rituals for great a lord or lady.

			\item \textbf{Chief Clerk} Dainaiki --- A functionary (\textit{sixth rank, upper grade}) in the Bureau of Central Affairs, charged with composing imperial rescripts, maintaining court records, and so on.

			\item \textbf{Chief Equerry} Kami --- The senior officer (\textit{junior fifth rank, upper grade}) in charge of the Left (Sama no Kami) or Right (Uma no Kami) Imperial Stables (Meryō, Uma no Tsukasa). The incumbent held the junior fifth rank, upper grade. In “The Broom Tree” also Chief Left Equerry.

			\item \textbf{Chief Lady in Waiting} Naishi --- A translation devised to suit the context in “The Pilgrimage to Sumiyoshi.” The precise nature of the office is unclear.

			\item \textbf{Cloistered Eminence} --- See \textbf{Eminence}.

			\item \textbf{Commander} Taishō --- The commanding officer (\textit{third rank, lower grade}) of the Right (Udaishō) or Left (Sadaishō) Palace Guards.

			\item \textbf{Commissioner} Taifu or Daibu --- A title held by the head of some government or quasigovernment organs, such as the Office of Upkeep or the Empress's Household; and by the second-level official in others, such as the Bureau of War. The title properly carried the fifth rank, upper or lower grade.

			\item \textbf{Commissioner of Ceremonial} Shikibu no Taifu --- The second-ranking officer (\textit{fifth rank, lower grade}) in the Bureau of Ceremonial.

			\item \textbf{Commissioner of Civil Affairs} Minbu no Taifu --- The second-ranking officer (\textit{fifth rank, lower grade}) in the Bureau of Civil Affairs.

			\item \textbf{Commissioner of the Household} Daibu (\textit{Chūgū no Daibu}) --- The chief administrator of the Empress's household.

			\item \textbf{Commissioner of War} Hyōbu no Taifu --- The second-ranking officer (\textit{fifth rank, lower grade}) in the Bureau of War.

			\item \textbf{Consort} Nyōgo --- An imperial wife whose father was at least a Minister or a Prince. An Empress was normally chosen from among the Consorts.

			\item \textbf{Constable} Udoneri --- One of about a hundred men affiliated with the Bureau of Central Affairs and selected from the families of men of the fourth and fifth ranks. Assigned to guard the highest nobles, Constables could be arrogant and rough.

			\item \textbf{Consultant} Sangi, Saishō --- The junior post (\textit{fourth rank, lower grade}) in the Council of State, below Counselor and Minister. There were normally eight.

			\item \textbf{Consultant Captain} Saishō no Chūjō --- A dual appointment as Consultant and as Captain in the Palace Guards.

			\item \textbf{Controller} Ben --- One of a body of officials under the Council of State. The Controllers were attached to the eight major government bureaus and were divided into Left and Right (\textit{four bureaus each}). There were three grades: Grand Controller (Daiben, junior fourth rank, upper grade), Controller (Chūben, fifth rank, upper grade), and Minor Controller (Shōben, fifth rank, lower grade).

			\item \textbf{Controller Chamberlain} Kurōdo no Ben --- A dual appointment as a Controller and as a fifth-rank Chamberlain.

			\item \textbf{Controller Lieutenant} Ben no ShōshōA dual appointment as a Controller and as a Lieutenant in the Palace Guards.

			\item \textbf{Council of State} Daijōkan --- Stood above the eight major bureaus as the highest organ of government. Its members were the three Ministers (\textit{Left, Right, and Palace}); the Counselors (Counselor, Grand Counselor); and the Consultants. The executive office of the Council of State employed Junior Counselors and Controllers, among other lesser officials.

			\item \textbf{Counselor} Chūnagon --- A middle-level post (\textit{junior third rank}) in the Council of State.

			\item \textbf{Court Ritualists} Gishikikan --- Officers of various ranks charged with conducting court ceremonials. In their formal stance they held their elbows stiffly out to either side as they held their batons.

			\item \textbf{Dame of Staff} Naishi no Suke --- One of four women officials (\textit{junior fourth rank, upper or lower grade}) under the Mistress of Staff, in the Office of Staff.
			      (\textit{Dazaifu}) Assistant (Dazai no) Shōni --- The assistant (fifth rank, upper grade) to the (Dazaifu) Deputy.
			      (\textit{Dazaifu}) Deputy (Dazai no) Daini --- The Deputy (junior fourth rank, lower grade) who represented the court at Dazaifu, in Kyushu. His senior was the Viceroy, whose post was a sinecure; the incumbent, a Prince, did not leave the City.

			\item \textbf{Deputy Commissioner of Ceremonial} Shikibu no Shō --- A third-level official (\textit{junior fifth rank, lower grade}) in the Bureau of Ceremonial.

			\item \textbf{Deputy (\textit{Governor})} Suke --- The deputy to a provincial Governor. In the case of Hitachi, Kazusa, and Shimōsa, the titular Governor was a Prince, but since the post was a sinecure, only the Deputy Governor actually went to the province. (The Governor of Hitachi who figures in “The Ivy” and succeeding chapters is actually such a Deputy.) The rank of a Deputy Governor, like that of a Governor, depended on the standing of his province, but it was in the sixth-rank range.

			\item \textbf{Director of Reckoning} Kazoe no Kami --- The director (\textit{junior fifth rank, upper grade}) of an office within the Bureau of Civil Affairs, charged with calculating and allocating certain types of tax revenue.

			\item \textbf{Director of Upkeep} Suri no Kami --- The head (\textit{junior fourth rank, lower grade}) of the Office of Upkeep.

			\item \textbf{Doctor} Hakase --- A senior scholar engaged to teach in the Academy, typically Chinese language (\textit{written}), literature, history, law, and so on. The appointment was in the junior fifth-rank range. Also Doctor of Letters (Monjō Hakase).

			\item \textbf{Doctor of the Almanac} Koyomi no Hakase --- A calendar or almanac specialist from the Yin-Yang Office.

			\item \textbf{Eminence (\textit{His, His Cloistered, Her Cloistered})} --- An honorific term, devised for the purposes of translation, for a former Empress or a Retired Emperor (In). If the figure has taken the vows of a monk or a nun, he or she is also called Cloistered, although this term is omitted where possible. The only “Her Cloistered Eminence” (Nyūdō Kisai no Miya) in the tale is Fujitsubo; there is no “Her Eminence.”

			\item \textbf{Empress} Chūgū, Kisaki --- The Emperor's highest-ranking wife. There could be only one. She was normally appointed from among the Consorts.

			\item \textbf{Empress Mother} Ōkisai no Miya, Ōkisaki --- The mother of an Emperor. She had not necessarily held the title of Empress under the previous reign.

			\item \textbf{Excellency} --- See \textbf{His Excellency}.

			\item \textbf{Fourth Rank Lieutenant} Shii no Shōshō --- A Lieutenant (\textit{normally in the fifth-rank range}) who exceptionally holds the fourth rank.

			\item \textbf{Fujiwara Aide of Ceremonial} Tō Shikibu no Jō.

			\item \textbf{Fujiwara Consultant} Tō Saishō.

			\item \textbf{Fujiwara Grand Counselor} Tō Dainagon.

			\item \textbf{Fujiwara Lieutenant} Tō Shōshō.

			\item \textbf{Gate Watch} Emonfu, Yugei --- The corps of guards who guarded the gates of the palace compound. They were divided into Left Gate Watch (\textit{Saemon}) and Right Gate Watch (Uemon). The chief officer on each side was the Intendant (junior fourth rank, lower grade), followed by Deputy (junior fifth rank, upper grade) and Aide (junior sixth rank, upper grade).

			\item \textbf{Governor} Kami --- The official appointed by the Emperor to govern a province. His rank, which depended on the standing of his province (\textit{the provinces were classified as great, major, medium, or minor}) could vary from junior fifth rank, upper grade, down to junior sixth rank, lower grade. The term was sometimes used not only for a Governor proper but also for a Deputy Governor, in cases where only the Deputy actually went to the province. Governors in general were also referred to as Zuryō (“Grant Holder”).

			\item \textbf{Grace} --- See \textbf{His Grace}.

			\item \textbf{Grand Counselor} Dainagon --- The office (\textit{third rank}) below Minister in the Council of State.

			\item \textbf{Haven} Miyasudokoro --- In the tale an unofficial title for a woman (\textit{especially an Intimate or a Consort}) who had borne a child to an Heir Apparent, an Emperor, or a Retired Emperor. The Japanese term suggests either “place (person) in whom the august affection found rest” or “place (person) in whom the august seed found rest.” Examples in the tale include Genji's mother after Genji's birth; the Rokujō Haven, whose daughter is by a deceased Heir Apparent; Genji's daughter after she bears the Heir Apparent a son; the mother of Ochiba, Emperor Suzaku's daughter, and Tamakazura's elder daughter, who bears a child to Retired Emperor Reizei.

			\item \textbf{Heir Apparent} Bō, Tōgū --- The formally designated successor to the reigning Emperor. He was not necessarily the Emperor's firstborn son.

			\item \textbf{Highness (\textit{His, Her})} --- An honorific term of address, used in translation for a Prince or Princess.

			\item \textbf{High Priestess (\textit{of Ise})} --- See Ise Priestess.

			\item \textbf{High Priestess of the Kamo Shrine} --- See Kamo Priestess.

			\item \textbf{His Eminence} --- See \textbf{Eminence}.

			\item \textbf{His Excellency} Ōitono, Otodo --- Refers to a Minister or a Chancellor. The use of “the Minister” rather than “His Excellency” implies a greater distance between that figure and the narrator (\textit{or the side with which her sympathies and those of her audience lie}). An example is the Minister of the Right, as distinguished from His Excellency (of the Left) in the chapters leading up to Genji's exile.

			\item \textbf{His Grace} --- An honorific term, devised for the purposes of translation, for Genji in “The Pilgrimage to Sumiyoshi” and after, following his return from exile.

			\item \textbf{His Highness of Central Affairs} Nakatsukasa no Miko --- The Prince who headed the Bureau of Central Affairs.

			\item \textbf{His Highness of Ceremonial} Shikibukyō no Miya --- The Prince who was the titular head (\textit{fourth rank, lower grade}) of the Bureau of Ceremonial.

			\item \textbf{His Highness of Kanzuke} Kanzuke no Miko --- The Governor of the province of Kanzuke (\textit{also Kamutsuke or Kōzuke, roughly present Gumma Prefecture}), like that of Hitachi and Kazusa, was a Prince, but the post was a sinecure, and the province was actually administered by a deputy.

			\item \textbf{His Highness of War} Hyōbukyō no Miya --- The Prince who was the titular head (\textit{fourth rank, lower grade}) of the Bureau of War.

			\item \textbf{His Reverence} --- See Prelate.

			\item \textbf{Honorary Deputy Governor} Yōmei no Suke --- A sinecure post bought from a high-ranking nobleman.

			\item \textbf{Honorary Retired Emperor} Jundaijōtennō --- An extraordinary title awarded Genji (\textit{in “New Wisteria Leaves”}) by his secret son, Emperor Reizei.

			\item \textbf{Household Deputy} Suke --- The second-level officer (\textit{junior fifth rank, lower grade}) in charge of the Empress's household.

			\item \textbf{Inspector} Azechi --- A high-level Inspector appointed to review the administration of the provinces. By Heian times the post survived only for the northernmost provinces, and it was mainly honorary.

			\item \textbf{Inspector Grand Counselor} Azechi no Dainagon --- A dual appointment as Inspector and Grand Counselor.

			\item \textbf{Intendant of the (\textit{Left, Right}) Gate Watch} Emon (Saemon, Uemon) no Kami --- The senior officer of the Gate Watch (junior fourth rank, lower grade).

			\item \textbf{Intendant of the (\textit{Left, Right}) Watch} Hyōe (Sahyōe, Uhyōe) no Kami --- The senior officer of the Watch.

			\item \textbf{Intimate} Kōi --- An imperial wife of lower standing than a Consort; her father was at most a Grand Counselor. The word kōi refers literally to someone who dresses the Emperor.

			\item \textbf{Ise Consort} Saikū no Nyōgo --- Literally ``(\textit{Ise}) Priestess Consort'' Akikonomu's appellation as Consort, since she had been the High Priestess of Ise.

			\item \textbf{Ise Priestess} Saikū --- An unmarried Princess who represented the Emperor as the chief priestess of the Ise Shrine, where the ancestral deity of the imperial line was enshrined.

			\item \textbf{Junior Counselor} Shōnagon --- A junior official (\textit{junior fifth rank, upper grade}) attached to the Council of State.

			\item \textbf{Kamo Priestess} Saiin --- The chief priestess of the Upper and Lower Kamo Shrines, just north of the City. Like the Ise Priestess, she was a Princess.

			\item \textbf{Lecturer} Kōji --- The officiating priest at certain major Buddhist rituals.

			\item \textbf{Left City Commissioner} Sakyō no Daibu --- The chief officer (\textit{junior fourth rank, lower grade}) charged with population registration, tax collection, legal appeals, security, and so on in the left (east) sector of the City.

			\item \textbf{Left Controller} Sachūben.

			\item \textbf{Left Gate Watch} --- See \textbf{Gate Watch}.

			\item \textbf{Left Grand Controller} Sadaiben --- See \textbf{Controller}.

			\item \textbf{Left Lieutenant} Sashōshō --- A Lieutenant in the Left Palace Guards (\textit{fifth rank, lower grade}).

			\item \textbf{Left Palace Guards} Captain Sakon no Chūjō.

			\item \textbf{Lieutenant} Shōshō --- A third-level officer (\textit{fifth rank, lower grade}) in the Palace Guards, below Commander and Captain.

			\item \textbf{Lord of Ceremonial} Shikibukyō --- The head (\textit{fourth rank, lower grade}) of the Bureau of Ceremonial. The holder of this title was a Prince.

			\item \textbf{Lord of Civil Affairs} Minbukyō --- The head (\textit{fourth rank, lower grade}) of the Bureau of Civil Affairs.

			\item \textbf{Lord of the Palace Bureau} Kunaikyō --- The head (\textit{fourth rank, lower grade}) of the one among the eight major government bureaus that was concerned with all matters affecting the Emperor's household.

			\item \textbf{Lord of the Treasury} Ōkurakyō --- The head (\textit{fourth rank, lower grade}) of the Bureau of the Treasury.

			\item \textbf{Majesty (\textit{His, Her})} --- Used for the Emperor and Empress.

			\item \textbf{Master of Discipline} Risshi, Rishi --- The lowest on the ladder of ecclesiastical ranks accessible to elite, fully ordained priests. In the time of the tale it is still a distinguished appointment --- more so than in later times. “Discipline” means the body of Buddhist monastic discipline.

			\item \textbf{Master (\textit{of the Household})} (Saburai no) Betō --- The chief administrator of the household of an imperial family member, such as a Prince or a Retired Emperor.

			\item \textbf{Master of Spells} Jugonshi --- A specialist in performing spells (\textit{majinai}) as healing magic, employed by the Office of Medicine (Ten'yaku Ryō).

			\item \textbf{Minister} Otodo --- The highest nonimperial office (\textit{second rank}) provided for in the government's formal table of organization, as the office of Chancellor was not; however, the post of Palace Minister (Naidaijin, Uchi no Otodo) was also a later addition. The Minister of the Left (Sadaijin, Hidari no Otodo) was normally but not necessarily senior to the Minister of the Right (Udaijin, Migi no Otodo), and the Palace Minister was somewhat junior in standing.

			\item \textbf{Mistress of Staff} Naishi no Kami --- The senior woman official (\textit{third rank}) in the Office of Staff. In principle, the incumbent supervised female palace staff, palace ceremonies, and the transmission of petitions and decrees. In practice, she was a junior wife to the Emperor.

			\item \textbf{Mistress of the Household} Nyobettō --- The ranking female official in a great lady's household.

			\item \textbf{Mistress of the Wardrobe} Mikushigedono --- The woman official in charge of the palace office that made the Emperor's clothing.

			\item \textbf{Mother of the Realm} Kuni no Haha --- An expression or title used to refer to an Empress or an Empress Mother.

			\item \textbf{Myōbu} --- A title borne in palace service by middle-ranking gentlewomen (\textit{fifth rank or above}) or by the wives of gentlemen of those ranks. Since a number of gentlewomen bore this title at the same time, people distinguished one from another by attaching to her title the name of the major office associated with her husband, father, or brother.

			\item \textbf{Novice} Nyūdō --- A man or woman of noble birth who had taken preliminary vows as a monk or nun. A Novice did not join a monastic community but pursued Buddhist practice at home.
			      Ōmyōbu --- A Myōbu (\textit{palace gentlewoman}) of imperial birth.

			\item \textbf{page, page girl} warawa --- A boy or girl of good family, in service in a noble household. Particularly on the male side there were warawa of mature years as well, as a kind of long-term servant, but these hardly figure in the tale. See also privy page.

			\item \textbf{Palace Guards} Konoefu --- The double (\textit{Left and Right, Sakon and Ukon}) corps of guards assigned to protect the palace proper and stationed in its innermost areas. The Palace Guards had precedence over the Watch and the Gate Watch. Their two Commanders (third rank, lower grade) outranked the Intendants of those units (fourth rank, lower grade). A second-level officer was a Captain (Chūjō fourth rank, lower grade), a third-level officer a Lieutenant (Shōshō; fifth rank, lower grade), and a fourth-level officer an Aide (Zō; sixth rank, upper grade).

			\item \textbf{Palace Minister} Naidaijin, Uchi no Otodo --- Normally the junior among the three Ministers who constituted the senior level of the Council of State.

			\item \textbf{Prelate (\textit{His Reverence})} Sōzu --- The highest ecclesiastical rank mentioned in the tale. Two higher ranks existed, but at the time of the tale (unlike later) they were rarely filled.

			\item \textbf{Prince (\textit{His Highness})} Miya --- An imperial son appointed to this title by his father. (Genji is therefore not a Prince.) Historically, most Princes were ranked in four grades and received an imperial stipend accordingly, but some were “unranked” (muhon). The tale says nothing about according this kind of status to an imperial grandson.

			\item \textbf{Princess (\textit{Her Highness})} Miya --- An imperial daughter appointed to this title by her father, or the recognized granddaughter of an Emperor in the male line. Suetsumuhana, whose father was a Prince, is therefore a Princess. In contrast, Aoi, whose mother is a Princess, is not one herself; the narration treats her purely as a commoner. Ukifune, the daughter of a Prince, is not a Princess because her father did not recognize her.

			\item \textbf{privy gentleman} uebito, tenjōbito --- A gentleman individually authorized by the Emperor to enter the privy chamber. The term referred more specifically to those gentlemen of the fourth and fifth ranks, together with Chamberlains (\textit{Kurōdo}) of the sixth rank, who would not otherwise have enjoyed the privilege automatically granted the top three ranks. The number of privy gentlemen varied but usually fell below one hundred and was sometimes less than a third of that.

			\item \textbf{privy page} tenjō warawa --- A boy of good family, not yet of age, who served in the privy chamber in order to learn court customs and manners.

			\item \textbf{Reader} Kōji --- The official charged with reading out the Chinese poems composed at a festive gathering.

			\item \textbf{Regent} Sesshō --- A high-ranking, nonimperial nobleman appointed to act for the Emperor while the Emperor was a minor. (\textit{The title “Kanpaku,” also translated “Regent” and held by someone who acted similarly for an adult Emperor, does not appear in the tale.})

			\item \textbf{Retired Emperor (\textit{His [Cloistered] Eminence})} In --- An Emperor who has abdicated and who now resides in a separate palace. Such a figure appears most often in this translation as His Eminence or, if he has taken Buddhist vows, as His Cloistered Eminence.

			\item \textbf{Right Captain} Uchūjō --- See Captain.

			\item \textbf{Right City Commissoner} Ukyō no Kami (\textit{Daibu}) --- The chief officer (junor fourth rank, lower grade) charged with population registration, tax collection, legal appeals, security, and so on in the right (west) sector of the City.

			\item \textbf{Right Controller} Uchūben.

			\item \textbf{Right Deputy} Migi no Suke --- A second-level officer in the Right Gate Watch (\textit{fifth rank, lower grade}).

			\item \textbf{Right Gate Watch} --- See Gate Watch.

			\item \textbf{Right Grand Controller} Udaiben --- See Controller.

			\item \textbf{Right Guards Commissioner} Ukon no Taifu ---  An Aide of the Right Palace Guards, exceptionally promoted to the fifth rank, so that he bears the fifth-rank title of Commissioner.

			\item \textbf{Second Equerry} Uma no Suke --- A second-level officer (\textit{sixth rank, lower grade}) of the imperial stables Left or Right.

			\item \textbf{Second of the Left Gate Watch} Saemon no Taifu --- The second-ranking officer (\textit{junior fifth rank, upper grade}) in the Left Gate Watch. (The title “Taifu” acknowledges the fifth-rank appointment.)

			\item \textbf{Second of the Watch} Hyōe no Suke --- A second-level officer in the Watch (\textit{junior fifth rank, upper grade}).

			\item \textbf{Secretary} Kurōdo no Tō --- A senior appointee in the Chamberlains' Office. Of the two Secretaries, one was concurrently a Controller (\textit{fifth rank, upper grade}) and the other normally a Captain (junior fourth rank, lower grade).

			\item \textbf{Secretary Captain} Tō no Chūjō --- A dual appointment as a Secretary and as a Captain in the Palace Guards.

			\item \textbf{Secretary Controller} Tō no Ben --- A dual appointment as a Secretary and as a Controller.

			\item \textbf{Secretary Lieutenant} Tō no Shōshō --- A dual appointment as a Secretary and as a Lieutenant (\textit{fifth rank, lower grade}) in the Palace Guards.

			\item \textbf{senior noble} kandachime --- A noble of at least the third rank (\textit{sanmi}) and holding a post at least at the level of Consultant (Sangi).

			\item \textbf{Treasury Commissioner} Okura no Taifu --- The second level (\textit{fifth rank, lower grade}) in the Office of the Treasury.

			\item \textbf{Upkeep Consultant} Suri no Saishō --- A dual appointment as Director of Upkeep and Consultant.

			\item \textbf{Viceroy} Dazai no Sochi --- The senior appointee to Dazaifu, the government outpost in Kyushu that was particularly responsible for such foreign relations as Japan had at the time. The post (\textit{junior third rank}) was held by a Prince. Since it was a sinecure and the incumbent never actually went to Kyushu, the real government representative there was the Dazaifu Deputy.

			\item \textbf{Watch} Hyōefu --- The corps (\textit{divided into Left [Sahyōefu] and Right [Uhyōefu]}) of guards charged with maintaining general security in the palace compound and in the City at large. The senior officer was the Intendant (junior fourth rank, lower grade), followed by the Second (junior fifth rank, upper grade).

			\item \textbf{yin-yang master}  --- An expert in yin-yang lore, connected with the Yin-Yang Office (\textit{Onmyō Ryō}), an organ of the Bureau of Central Affairs in charge of matters pertaining to astrology, weather, the calendar, timekeeping, and divination.

		\end{itemize}

		%\end{footnotesize}
	\end{small}

\end{multicols}

\end{document}
